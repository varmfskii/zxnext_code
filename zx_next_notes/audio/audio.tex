\chapter{Audio}
\section{Internal Speaker}

The baseline sound of the ZX Spectrum was produced by toggling the Ear
bit (bit 4) of \$fe (254) The ULA port to produce 1-bit audio.  It is
enabled by bit 4 of Next register \$08 (8).  While this does work on
the ZX Spectrum Next, there are other much better methods and this is
only supported for backward compatibility.

Code:
\begin{verbatim}
;; enable internal speaker
ld bc,$243B
ld a,$08
out (c),a
ld bc,$253B
in a,(c)
or $10
out (c),a
\end{verbatim}

\section{Spectdrum/Convox}

This is 8-bit D/A audio.  It is enabled by setting bit 3 of Next
register \$08 (8).  After that audio can be controlled by writing
linear 8-bit unsigned values to port \$df (223).

Code:
\begin{verbatim}
;; enable SpecDrum/Convox audio
ld bc,$243B
ld a,$08
out (c),a
ld bc,$253B
in a,(c)
or $08
out (c),a
\end{verbatim}

\section{Turbosound}

TurboSound consists of the implementation of three AY-3-8912 chips. To
enable TurboSound set bit 1 of Next Register \$08 (8). Once enabled
the sound chips and registers of the sound chips are selected using
port \$fffd (65533) TurboSound Next Control while the registers are
accessed using \$bffd () Sound Chip Register Access.  To enable access
to a particular chip write 111111xx to the control register where
01=AY1, 10=AY2, and 11=AY3.  Access to particular registers of the
selected chip is selected by writing the register number to the
control register. You can then access a chip register using the access
port.

Code:
\begin{verbatim}
;; enable TurboSound audio
ld bc,$243B
ld a,$08
out (c),a
ld bc,$253B
in a,(c)
or $02
out (c),a
\end{verbatim}

Each of the three AY chips has three channels, A, B, and C whose
mapping is controlled by bit 5 of Next register 0x08 (8).

\begin{itemize}
\item (R/W) 0x00 (0) $\Rightarrow$ Channel A fine tune
\item (R/W) 0x01 (1) $\Rightarrow$ Channel A coarse tune (4 bits)
\item (R/W) 0x02 (2) $\Rightarrow$ Channel B fine tune
\item (R/W) 0x03 (3) $\Rightarrow$ Channel B coarse tune (4 bits)
\item (R/W) 0x04 (4) $\Rightarrow$ Channel C fine tune
\item (R/W) 0x05 (5) $\Rightarrow$ Channel C coarse tune (4 bits)
\item (R/W) 0x06 (6) $\Rightarrow$ Noise period (5 bits)
\item (R/W) 0x07 (7) $\Rightarrow$ Tone Enable
  \begin{itemize}
  \item bit 5: Channel C tone enable (0=enable, 1=disable)
  \item bit 4: Channel B tone enable (0=enable, 1=disable)
  \item bit 3: Channel A tone enable (0=enable, 1=disable)
  \item bit 2: Channel C noise enable (0=enable, 1=disable)
  \item bit 1: Channel B noise enable (0=enable, 1=disable)
  \item bit 0: Channel A noise enable (0=enable, 1=disable)
  \end{itemize}
\item (R/W) 0x08 (8) $\Rightarrow$ Channel A amplitude
  \begin{itemize}
  \item bit 4: 0=fixed amplitude, 1=use envelope generator (bits 0-3 ignored)
  \item bits 0-3: value of fixed amplitude
  \end{itemize}
\item (R/W) 0x09 (9) $\Rightarrow$ Channel B amplitude
  \begin{itemize}
  \item bit 4: 0=fixed amplitude, 1=use envelope generator (bits 0-3 ignored)
  \item bits 0-3: value of fixed amplitude
  \end{itemize}
\item (R/W) 0x0A (10) $\Rightarrow$ Channel C amplitude
  \begin{itemize}
  \item bit 4: 0=fixed amplitude, 1=use envelope generator (bits 0-3 ignored)
  \item bits 0-3: value of fixed amplitude
  \end{itemize}
\item (R/W) 0x0B (11) $\Rightarrow$ Envelope period fine
\item (R/W) 0x0C (12) $\Rightarrow$ Envelope period coarse
\item (R/W) 0x0D (13) $\Rightarrow$ Envelope shape
  \begin{itemize}
  \item bit 3: Continue: 0=drop to amplitude 0 after 1 cycle, 1=use ‘Hold’ value
  \item bit 2: Attack: 0=generator counts down, 1=generator counts up
  \item bit 1: Alternate:
    \begin{itemize}
    \item hold=0: 0=generator resets after each cycle, 1=generator
      reverses direction each cycle
    \item hold=1: 0=hold final value, 1=hold initial value
    \end{itemize}
  \item bit 0: Hold: 0=cycle continuously, 1=perform one cycle and hold
  \end{itemize}
\end{itemize}

