\chapter{Audio}
\section{ZX Spectrum 1-bit}
\begin{multicols}{2}
The baseline sound of the ZX Spectrum was produced by toggling the Ear
bit (bit 4) of \$fe (254) The ULA port to produce 1-bit audio.  It is
enabled by bit 4 of Next register \$08 (8).  While this does work on
the ZX Spectrum Next, there are other much better methods and this is
only supported for backward compatibility.

\sinset
Code:
\begin{verbatim}
;; enable internal speaker
ld bc,$243B
ld a,$08
out (c),a
ld bc,$253B
in a,(c)
or $10
out (c),a
\end{verbatim}
\einset

\section{Sampled 8-bit}

The ZX Next has four 8-bit D/A audio channels connected to provide
sampled stereo sound. Channels A and B are the left channels, while C
and D are the right channels. In order use 8-bit sound, it must first
be enabled by setting bit 3 on nextreg \$08. In order to emulate
legacy hardware there are a number of ports that can be used to
control the four channels additionally these are mirrored to three
nextregs to enable driving audio using the copper.  Channel A is
mapped to ports \$0f, \$3f, and \$f1; channel B to ports \$1f and \$f3
and nextreg \$2C; channel C to ports \$4f, and \$f9 and nextreg \$2E;
and channel D to: \$5f and \$fb; with port \$df connected to both
channel A and C and nextreg \$2D connected to both channel A and D.

\sinset
Code:
\begin{verbatim}
;; enable SpecDrum/Convox audio
ld bc,$243B
ld a,$08
out (c),a
ld bc,$253B
in a,(c)
or $08
out (c),a
\end{verbatim}
\einset

\section{Turbosound}

TurboSound consists of the implementation of three AY-3-8912 chips. To
enable TurboSound set bit 1 of Next Register \$08 (8). Once enabled
the sound chips and registers of the sound chips are selected using
port \$fffd (65533) TurboSound Next Control while the registers are
accessed using \$bffd () Sound Chip Register Access.  To enable access
to a particular chip write 111111xx to the control register where
01=AY1, 10=AY2, and 11=AY3.  Access to particular registers of the
selected chip is selected by writing the register number to the
control register. You can then access a chip register using the access
port.

\sinset
Code:
\begin{verbatim}
;; enable TurboSound audio
ld bc,$243B
ld a,$08
out (c),a
ld bc,$253B
in a,(c)
or $02
out (c),a
\end{verbatim}
\einset

Each of the three AY chips has three channels, A, B, and C whose
mapping is controlled by bit 5 of Next register 0x08 (8).

\register{R}{00}{Machine ID}
\begin{itemize}
\item 00000001 = DE1A
\item 00000010 = DE2A
\item 00000101 = FBLABS
\item 00000110 = VTRUCCO
\item 00000111 = WXEDA
\item 00001000 = EMULATORS
\item 00001010 = ZX Spectrum Next
\item 00001011 = Multicore
\item 10101010 = ZX Spectrum Next Core on unAmiga
\item 10111010 = ZX Spectrum Next Core on SiDi
\item 11001010 = ZX Spectrum Next Core on MIST
\item 11011010 = ZX Spectrum Next Core on MiSTer
\item 11011010 = ZX Spectrum Next Core on unAmiga Reloaded
\item 11101010 = ZX Spectrum Next Core on ZX-DOS
\item 11111010 = ZX Spectrum Next Anti-brick
\end{itemize}


\register{R}{01}{Core Version}
\begin{itemize}
\item bits 7-4 = Major version number
\item bits 3-0 = Minor version number
\item[] See register \$0E for sub minor version number
\end{itemize}


\register{R/W}{02}{Reset}\\
Read
\begin{itemize}
\item bit 7 = Expansion bus \textoverline{RESET} Asserted
\item bits 6-4 = Multiface \textoverline{NMI} generated by I/O trap
(experimental) (3.01.10)
\begin{itemize}
\item[] 011 = port \$3ffd write
\item[] 010 = port \$3ffd read
\item[] 001 = port \$2ffd read
\item[] 000 = none
\end{itemize}
\item bit 3 = Indicates multiface \textoverline{NMI} was generated by this
nextreg (3.01.09)
\item bit 2 = Indicates divmmc \textoverline{NMI} was generated by this
nextreg (3.01.09)
\item bit 1 = Last reset was Hard reset
\item bit 0 = Last reset was Soft reset
\end{itemize}
Write
\begin{itemize}
\item bit 7 = Hold Expansion bus and ESP \textoverline{RESET}
\item bits 6-5 = Reserved, must be 0
\item bit 4 = clear I/O trap (experimental)(3.01.10)
\item bit 3 = Generate multiface \textoverline{NMI} (write zero to clear)
(3.01.09)
\item bit 2 = Generate divmmc \textoverline{NMI} (write zero to clear)
(3.01.09)
\item bit 1 = generate Hard reset (reboot)
\item bit 0 = generate Soft reset
\end{itemize}


\register{R/W}{03}{Machine Type}\\
\begin{itemize}
\item bit 7 (R) = nextreg \$44 second byte indicator
\item bit 7 (W) = allow changes to bits 6-4 (0 on hard reset)
\item bits 6-4 = Display Timing
  \begin{itemize}
  \item 000 = Internal use
  \item 001 = ZX 48k
  \item 010 = ZX 128k/2+
  \item 011 = ZX +2A/+2B/+3
  \item 100 = Pentagon
  \end{itemize}
\item bit 3 = Display Timing user lock control (0 on hard reset)
\item bits 2-0 = Machine type (write on config mode only) determines ROMs loaded
  \begin{itemize}
  \item 000 = Configuration mode
  \item 001 = ZX 48k
  \item 010 = ZX 128k/+2 (Grey)
  \item 011 = ZX +2A/+2BB/+3/Next Native
  \item 100 = Pentagon
  \end{itemize}
\end{itemize}


\register{R/W}{04}{Channel C fine tune}


\register{R/W}{05}{Peripheral 1 Settings}
\begin{itemize}
\item bits 7-6 = joystick 1 mode (MSB)
\item bits 5-4 = joystick 2 mode (MSB)
\item bit 3 = joystick 1 mode (LSB)
\item bit 2 = 50/60 Hz mode (0 = 50Hz, 1 = 60Hz)
\item bit 1 = joystick 2 mode (LSB)
\item bit 0 = Enable Scandoubler
\end{itemize}
Joystick modes
\begin{itemize}
\item 000 = Sinclair 2 (67890)
\item 001 = Kempston 2 (port \$37)
\item 010 = Kempston 1 (port \$1F)
\item 011 = Megadrive 1 (port \$1F)
\item 100 = Cursor
\item 101 = Megadrive 2 (port \$37)
\item 110 = Sinclair 1 (12345)
\item 111 = I/O Mode (3.01.04)
Both joysticks are places in I/O Mode if either is set to I/O
Mode. The underlying joystick type is not changed and reads of this
register will continue to return the last joystick type. Ehether the
joystick is in io mode or not is invisible but this state can be
cleared either through reset or by re-writing the gegister with
joystick type not equal to 111. Recovery time for a normal joystick
read after leaving I/O Mode is at most 64 scan lines.
\end{itemize}


\register{R/W}{06}{Peripheral 2 Settings}
\begin{itemize}
\item bit 7 = Enable F8 cpu speed hotkey (soft reset = 1)
\item bit 6 = Divert BEEP only to internal speaker (hard reset = 0)
\item bit 5 = Enable F3 50/60 Hz hotkey (soft reset = 1)
\item bit 4 = Enable divmmc nmi by DRIVE button (hard reset = 0)
\item bit 3 = Enable multiface nmi by M1 button (hard reset = 0)
\item bit 2 = PS/2 mode (config mode only)
\begin{itemize}
\item 0 = keyboard primary
\item 1 = mouse primary
\end{itemize}
\item bits 1-0 = Audio chip mode
\begin{itemize}
\item 00 = YM
\item 01 = AY
\item 11 = Hold all AY in reset
\end{itemize}
\end{itemize}


\register{R/W}{07}{Turbo mode}\\
Read
\begin{itemize}
\item bits 7-6 = Reserved
\item bits 5-4 = Current Actual CPU Speed
\item bits 3-2 = Reserved
\item bits 1-0 = Current Selected CPU Speed (00 on reset)
\end{itemize}
Write
\begin{itemize}
\item bits 7-2 = Reserved, must be 0
\item bits 1-0 = Select CPU Speed
\end{itemize}
CPU Speeds
\begin{itemize}
\item 00 = 3.5MHz
\item 01 = 7MHz
\item 10 = 14MHz
\item 11 = 28MHz
\end{itemize}


\register{R/W}{08}{Peripheral 3 Settings}
\begin{itemize}
\item bit 7 = 128K Banking Unlock (inverse of port \$7FFD, bit 5) (0
  on reset)
\item bit 6 = Disable RAM and Port Contention (0 on reset)
\item bit 5 = PSG Stereo Mode Control (0 = ABC, 1 = ACB) (0 on hard
  reset)
\item bit 4 = Enable internal speaker (1 on hard reset)
\item bit 3 = Enable DACs (0 on hard reset)
\item bit 2 = Enable read of port \$FF (Timex) (0 on hard reset)
\item bit 1 = Enable Multiple PSGs (0 on hard reset)
\item bit 0 = Enable Issue 2 Keyboard
\end{itemize}


\register{R/W}{09}{Peripheral 4 setting:}
\begin{itemize}
\item bit 7 = PSG 2 Mono Enable (0 on hard reset)
\item bit 6 = PSG 1 Mono Enable (0 on hard reset)
\item bit 5 = PSG 0 Mono Enable (0 on hard reset)
\item bit 4 = Sprite ID lockstep enable (1 = Nextreg \$34 and IO Port
  \$303B are in lockstep, 0 on reset)
\item bit 3 = divMMC mapRAM bit Control (reset bit 7 of port \$E3)
\item bit 2 = HDMI audio mute (0 on hard reset)
\item bits 1-0 = scanlines
  \begin{itemize}
  \item 00 = scanlines off
  \item 01 = scanlines 75\%
  \item 10 = scanlines 50\%
  \item 11 = scanlines 25\%
  \end{itemize}
\item[] In Sprite lockstep, NextREG \$34 and Port \$303B are in
  lockstep
\end{itemize}


\register{R/W}{0A}{Peripheral 5 setting:}
\begin{itemize}
\item bits 7-6 = Multiface type (00 on hard reset)
  \begin{itemize}
  \item 00 = Multiface +3 (enable port 0x3F, disable port 0xBF)
  \item 01 = Multiface 128 v87.2 (enable port 0xBF, disable port 0x3F)
  \item 10 = Multiface 128 v87.12 (enable port 0x9F, disable port 0x1F)
  \item 11 = Multiface 1 (enable port 0x9F, disable port 0x1F)
  \end{itemize}
\item bit 5 = Reserved, must be zero
\item bit 4 = Enable divmmc automap (hard reset = 0) (3.01.10)
\item bit 3 = 1 to reverse left and right mouse buttons (3.01.07)
\item bit 2 = Reserved, must be 0
\item bits 1-0 = mouse dpi (00 on hard reset) (3.01.05)
  \begin{itemize}
  \item 00 = low dpi
  \item 01 = default
  \item 10 = medium dpi
  \item 11 = high dpi
  \end{itemize}
\end{itemize}


\register{R/W}{0B}{Envelope period fine}


\register{R/W}{0C}{Envelope period coarse}


\register{R/W}{0D}{Envelope shape}
\begin{itemize}
\item bit 3 = Continue
  \begin{itemize}
  \item 0 = drop to amplitude 0 after 1 cycle
  \item 1 = use ‘Hold’ value
  \end{itemize}
\item bit 2 = Attack
  \begin{itemize}
  \item 0 = generator counts down
  \item 1 = generator counts up
  \end{itemize}
\item bit 1 = Alternate\\
  hold = 0
  \begin{itemize}
  \item 0 = generator resets after each cycle
  \item 1=generator reverses direction each cycle
  \end{itemize}
  hold=1
  \begin{itemize}
  \item 0 = hold final value
  \item 1 = hold initial value
  \end{itemize}
\item bit 0 = Hold
  \begin{itemize}
  \item 0 = cycle continuously
  \item 1 = perform one cycle and hold
  \end{itemize}
\end{itemize}



\subsection{Pi Audio}
If connected the Pi Zero is configured to use the ZX Next as a
soundcard over an \iis interface making the Raspberry Pi a fully
configurable audio source for the ZX Spectrum Next.
\end{multicols}
