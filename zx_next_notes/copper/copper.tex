\chapter{Copper and Display Timing}
From: KevB (aka 9bitcolour)

\paragraph{Introduction}

The ZX Spectrum Next includes a co-processor named ``COPPER''. It
functions in a similar way to the Copper found in the Commodore Amiga
Agnus custom chip.  It's role is to free the Z80 of tasks that require
the writing of hardware registers at precise pixel co-ordinates.

\paragraph{Overview}

The ZX Spectrum Next COPPER has three instructions: NOOP, MOVE, WAIT. 

NOOP is used to fine tune timing. MOVE writes data to a specific range
of hardware registers. WAIT waits for a pixel position on the video
display.

These instructions are stored in 2k (2048 BYTES) of dedicated
write-only program RAM also known as a ``Copper list''.

Each instruction is 16 bits (WORD) in size allowing for a maximum of
1024 instructions to be stored in the program RAM. The COPPER uses an
internal 10 bit program counter (PC) which wraps to zero at the end of
the list. The PC can be reset to zero, this is the default value after
a hard/soft reset.

The instructions are stored in big endian format and transferred to
the 2k program RAM using the Z80 or DMA (bits 15..8 followed by bits
7..0).

Three write-only hardware registers control access to the program RAM
as well as the operating modes.

System performance is not affected when the COPPER is executing instructions.

The hardware registers and COPPER program RAM are not connected to the
main memory BUS. The overall design of this system together with the
use of alternate clock edges means that contention between the COPPER,
Z80 and DMA has been eliminated.

The COPPER has a base clock speed of 13.5Mhz for HDMI and 14Mhz for VGA.

The bandwidth is around 14 million single cycle NOOP/WAIT instructions
and 7 million two cycle MOVE instructions per second.

\section{Timing}

To fully understand the COPPER, you must first understand the display
timing for each of the machines and video modes found in the ZX
Spectrum Next.

There are several display timing configurations due to the four
machine types, two refresh rates, two video systems (VGA/HDMI) and
Timex HIRES mode.

Details of these timings are outlined in this chapter.

\paragraph{Machines}

The ZX Spectrum Next has four machine types (48k, 128k, Pentagon, and
HDMI). The machine timing and HDMI determine the number of T-states
per line which determines the base dot clock frequency and Z80/DMA
clock speed.

This guide groups machine types by their timing for convenience. The
HDMI video mode overrides the default machine timing so it is included
as an extra machine type which does not exist in the official
documentation.

\paragraph{Display}

The ZX Spectrum Next doesn't have video modes based on resolution that
you would expect to find on graphics card based hardware. There is one
fixed resolution of $256\times192$ which can be doubled to
$512\times192$ in Timex HIRES mode. What it does have is the ability
to set the refresh rate from 50Hz to 60Hz and horizontal dot
clock. This in turn together with the VGA and HDMI timing affects the
vertical line count giving several combinations in total.

VGA modes 0..6 are included as one single VGA mode as the internal
machine timing is constant across those seven refresh rate steps.

More details can be found in Video modes.

\paragraph{Resolution}

There are two main horizontal resolutions: standard $256\times192$ and
Timex HIRES $512\times192$. Details of LORES $128\times96$ are not
included to simplify this guide.

The frame buffer height is fixed at 192 pixels and surrounded by a
large border and overscan as well as horizontal and vertical blanking
periods.

There are five vertical line counts: 261, 262, 311, 312, 320. Several
pixels are hidden in the overscan and blanking periods beyond the
visible border.

The result is $256\times192$ and $512\times192$ pixel resolutions with
a large border.

The colour of the visible border beyond the frame buffer can be
manipulated. Visual changes will not show during the overscan and
blanking periods.

\paragraph{Dot Clock}

The dot clock on the ZX Spectrum Next runs at 13.5Mhz for HDMI and
around 14Mhz for VGA. The COPPER clock runs at the same frequency as
the dot clock. For v3.00 the copper runs at twice the frequency of the
dot clock.

The number of dot clocks per line is calculated by multiplying the
number of 3.5Mhz Z80 T-states per line by four. Example: 228Ts * 4 =
912 dot clocks.

The number of dot clocks per second is calculated by the following:

T-states per line * 4 * line count * refresh rate

In standard $256\times192$ resolution the duration of one pixel is two
dot clocks. In Timex HIRES $512\times192$ resolution the duration of
one pixel is one dot clock.

Details of the dot clock counts can be found in tables 5.1 and 5.2.

\begin{table}[h]\centering
  \caption{Vertical Line Counts and Dot Clock Combinations}
  \csvautotabular{copper/perline.csv}
\end{table}

\begin{table}[h]\centering
  \caption{Dot Clocks per Second}
  \csvautotabular{copper/persec.csv}
\end{table}

\paragraph{Coordinates}

The top left pixel of the frame buffer is line 0 and horizontal dot
clock 0. This is also known as ``0,0''.

The bottom right pixel of the frame buffer in standard $256\times192$
resolution is line 191 and horizontal dot clocks 510+511.

The bottom right pixel of the frame buffer in Timex HIRES
$512\times192$ resolution is line 191 and horizontal dot clock 511.

The line one pixel above the frame buffer is the last line of the
video frame and equal to the total line count minus one (312-1 for
example).

The line one pixel below the frame buffer is line 192.

The COPPER horizontal dot clock compare is locked to every eight
pixels in standard $256\times192$ resolution and every sixteen pixels
in Timex HIRES $512\times192$ resolution. The NOOP instruction can be
used to fine tune timing in single dot clock steps.

\paragraph{Compare}

The COPPER uses a 9 bit vertical line compare allowing it to handle
the various line counts.

The COPPER horizontal compare is 6 bits meaning that it can wait for
64 positions across each line. The range of this value is limited by
the machine timing as that determines the number of dot clocks per
line.

\begin{table}[h]\centering
  \caption{Maximum Horizontal COPPER Compare}
  \csvautotabular{copper/maxhcmp.csv}
\end{table}

Each horizontal compare is in steps of 16 dot clocks to cover the full
range across a raster line.

16 dot clocks = 4 pixels in lo $128\times96$ resolution

16 dot clocks = 8 pixels in standard $256\times192$ resolution

16 dot clocks = 16 pixels in high $512\times192$ resolution

There is some slack to consider after the maximum horizontal compare
value. The slack is calculated using the following:

dot clocks per line - maximum horizontal compare * 16
          
\begin{table}[h]\centering
  \caption{Slack Dot Clocks After Maximum Compare}
  \csvautotabular{copper/aftermaxcmp.csv}
\end{table}

Table 5.5 provides details of the horizontal display, left/right
border, blanking and COPPER dot clock/pixel position compare values:

\begin{table}[h]\centering
  \caption{Horizontal Timing}
  \csvautotabular{copper/htime.csv}

  \raggedright -- Dot clock compare is out of range.
\end{table}

Table 5.6 provides a detailed list of vertical display, top/bottom
border and blanking as well as maximum COPPER line compare. It also
provides the ULA VBLANK interrupt line number.

\begin{table}[h]\centering\tiny
  \caption{Vertical Timing}
  \csvautotabular{copper/vtime.csv}

  \raggedright -- Line compare is out of range

  * ULA VBLANK interrupt.
\end{table}

Note: The HDMI overscan and blanking period is larger than that of a
VGA monitor which can auto-adjust alignment. The following data is
based on visible results from various monitors thus subject to
refinement.

Pixels are visible during DISPLAY/BORDER and hidden during BLANKING.

\paragraph{Overscan}

The visible area of the display can extend to resolutions exceeding
$256\times192$.

The 50/60 Hz refresh rate mode dictates the vertical limit.

VGA and HDMI differ with VGA providing more visible pixels beyond the
range of HDMI. Table 5.7 provides ideal
extended pixel resolutions:

Maximum Extended VGA Resolutions

50Hz = $352\times288$ (standard 256 resolution)

60Hz = $352\times240$ (standard 256 resolution)

\begin{table}[h]\centering
  \caption{Ideal Extended Resolutions for Both VGA and HDMI}
  \csvautotabular{copper/idealres.csv}
\end{table}

Table 5.8 provides COPPER horizontal position and vertical
line compare parameters for ideal extended resolutions:

\begin{table}[h]\centering\small
  \caption{Ideal Extended Resolution Display Parameters}
  \csvautotabular{copper/idealparam.csv}

  \raggedright TOP: Initial line of the extended top border area - see notes below*

  BOT: Last line of the extended bottom border area - see notes below*

  LEFT: First pixel of the extended left border area - see notes below**

  RIGHT: Last pixel of the extended right border area - see notes below**

  * Line compare value for MOVE (bits 8..0).

  ** The integer part is the horizontal value for MOVE (bits 14..9).

  ** The fractional part is specified in dot clocks (NOOP instructions).
\end{table}

\section{Instructions}

This section describes the behaviour of the COPPER instructions as
well as the bit definitions and execution time.

The three 16 bit COPPER instructions are comprised of the following
bit definitions:

\begin{table}[h]\centering
  \caption{Instruction Bit Definition}
  \csvautotabular{copper/instrbit.csv}

  \raggedright H   6 bit horizontal dot clock compare

  V   9 bit vertical line compare

  R   7 bit Next register 0x00..0x7F

  D   8 bit data
\end{table}
\paragraph{NOOP}

NOOP (no-operation) executes in one dot clock. It is useful for fine
tuning timing, initialising COPPER RAM and 'NOP' out COPPER program
instructions.

It can be used to align colour and display changes to half pixel
positions in standard $256\times192$ resolution. Its duration is equal
to one Timex HIRES pixel.

This guide uses the name 'NOOP' to avoid confusion with the Z80 opcode
NOP.

\paragraph{MOVE}

MOVE executes in two dot clocks. It moves 8 bits of data into any of
the Next hardware registers in the range \$00 (0) .. \$7F (127).

The WORD value \$0000 is reserved for the NOOP instruction so no
register access is carried out for that special case. Register \$00 is
read-only so not affected by the restriction of not being able to
write zero to it.

This instruction can perform 7 million register writes per second for
VGA and 6.75 million register writes per second for HDMI.

\paragraph{WAIT}

WAIT executes in one dot clock. It performs a compare with the current
vertical line number and the current horizontal dot clock.

WAIT will hold until the current raster line matches the 9 bit value
stored in bits 8..0. When the line compare matches, WAIT will still
hold if the current horizontal dot clock is less than the value in
bits 14..9.

This compare logic means that out of order vertical line compares will
cause the COPPER to wait until the next video frame as the test is for
an exact match of the line number. The COPPER will continue to the
next instruction after an out of order horizontal pixel position
compare as the test checks for the current dot clock being greater
than or equal to the compare value.

WAIT will stop the COPPER when a compare is made against an out of
range vertical line or horizontal dot clock position as they will
never occur

A standard way to terminate a COPPER program is to wait for line 511
and horizontal position 63. This encodes into the instruction WORD
\$FFFF.

The horizontal dot clock position compare includes an adjustment
meaning that the compare completes three dot clocks early in standard
$256\times192$ resolution and two dot clocks early in Timex HIRES
$512\times192$ resolution. In practice, a pixel position can be
specified with clocks to spare to write a register value before the
pixel is displayed. This saves software having to auto-adjust
positions to arrive early. It also means that a wait for 0,0 can
affect the first pixel of the frame buffer before it is displayed and
set the scroll registers without visual artefacts.

\paragraph{Example}

The following example provides a simple COPPER program to move data to
a hardware register at two specific pixel positions. The BYTES for the
program are listed in the left column:

\begin{verbatim}
         PAL8 equ   0x41           ; 8 bit palette hardware register

$80,$00       WAIT  0,0            ; wait for pixel position 0,0 (H,V)
$00,$00       NOOP                 ; fine tune timing by one dot clock
$41,$E0       MOVE  PAL8,11100000b ; write RED to palette register

$C0,$BF       WAIT  32,191         ; wait for pixel position 256,191
$00,$00       NOOP                 ; fine tune timing by one dot clock
$41,$00       MOVE  PAL8,00000000b ; write BLACK to palette register

$FF,$FF       WAIT  63,511         ; wait for an out of range position
\end{verbatim}

\section{Control}

The COPPER is controlled by the following three write-only registers:

\begin{itemize}
\item[] \$60 (96)   Copper data
\item[] \$61 (97)   Copper control LO BYTE
\item[] \$62 (98)   Copper control HI BYTE
\end{itemize}

The COPPER instructions are written one BYTE at a time to the program
RAM using register \$60 (Copper data).

An index system is used to select the destination write address within
the 2K program RAM. Eleven bits are needed to represent the
index. Registers \$61 and \$62 hold this 11 bit index.

The index increments each time one BYTE is written to register
\$60. The index wraps to zero when the last BYTE of program RAM is
written.

The instruction data is normally written in big endian format although
there is no rule stating that partial instruction BYTES cannot be
written. It is safe to write to the COPPER program RAM while the
COPPER is executing as long the instruction data written does not
create a mall formed instruction which comprises of one half of the
current executing instruction and one half the new instruction - this
could result in unexpected behaviour.

The Z80 and DMA can be used to write the instruction data.

Writing to program RAM while the COPPER is running has no impact on
system performance as the RAM is contention free. COPPER timing is not
affected by the Z80 or DMA writing to the program RAM. Program RAM is
write-only.

The contents of the 2k program RAM are preserved during a hard/soft reset.

Register \$61 holds the lower 8 bits of the index. Register \$62 holds
the upper 3 bits of the index as well as two control bits which set
the COPPER operating mode.

\begin{table}[h]\centering
  \caption{Register Bit Definitions}
  \csvautotabular{copper/regbit.csv}

  \raggedright D    8 bit data

  I   11 bit index 
  
  C   2 bit control
\end{table}

The COPPER has an internal 10 bit program counter (PC). Each
instruction advances the program counter by one after completion. The
program counter wraps to zero after the last instruction at location
1023. This causes the copper list to loop.

The program counter defaults to zero during a hard/soft reset.

The control bits require a change to update the operating mode. This
feature preserves COPPER operation when setting the program RAM index
address.

The program counter is preserved when stopping the COPPER. Two of the
four control settings reset the internal PC to zero.

Table 5.11 describes the control bits:

\begin{table}[h]\centering
  \caption{Control Mode Definitions}
  \csvautotabular{copper/controlmodes.csv}

  \raggedright * The control mode names used in this guide differ from
  the official names.
\end{table}

Here is a detailed description of the control bits:

\paragraph{STOP}

This is the default operating mode set during a hard/soft reset. The
COPPER is idle in this state and will STOP if currently executing when
entering this mode. It is safe to write to any location within the 2K
program RAM when the COPPER is stopped.

Entering STOP mode preserves the internal program counter so that the
COPPER may continue when restarted.

\paragraph{RESET}

The program counter is RESET to zero when entering this mode. The
COPPER is started if idle otherwise entering this mode acts as a jump
to location zero when the COPPER is running.

\paragraph{START}

Entering this mode causes an idle COPPER to start executing
instructions from the current program counter. Entering this mode
while the COPPER is running has no effect other than to disable FRAME
mode if active.

\paragraph{FRAME}

The program counter is RESET to zero when entering this mode. The
COPPER is started if idle otherwise entering this mode acts as a jump
to location zero when the COPPER is running.

Entering this state enables FRAME mode. The program counter will be
reset to zero each frame at 0,0.

\section{Configuration}

Hardware registers provide timing and configuration data allowing
software to build and configure COPPER programs that function
correctly across the various video modes and machine types. It is not
essential to detect the machine type but it should be noted that
software should not assume that it is running on a specific machine as
the COPPER hardware is available across all four machine types.

Three registers can be read to determine the machine configuration for
Ts per line, dot clocks, refresh rate, line count and maximum
horizontal dot clock/pixel position compare.

\paragraph{Refresh Rate}

The refresh rate must be taken into account and can change real-time
so should be monitored and auto-configured when the COPPER is active
as the line count will change with the refresh rate. This could lead
to the COPPER waiting for lines that never occur.

\register{R/W}{05}{Peripheral 1 Settings}
\begin{itemize}
\item bits 7-6 = joystick 1 mode (MSB)
\item bits 5-4 = joystick 2 mode (MSB)
\item bit 3 = joystick 1 mode (LSB)
\item bit 2 = 50/60 Hz mode (0 = 50Hz, 1 = 60Hz)
\item bit 1 = joystick 2 mode (LSB)
\item bit 0 = Enable Scandoubler
\end{itemize}
Joystick modes
\begin{itemize}
\item 000 = Sinclair 2 (67890)
\item 001 = Kempston 2 (port \$37)
\item 010 = Kempston 1 (port \$1F)
\item 011 = Megadrive 1 (port \$1F)
\item 100 = Cursor
\item 101 = Megadrive 2 (port \$37)
\item 110 = Sinclair 1 (12345)
\item 111 = I/O Mode (3.01.04)
Both joysticks are places in I/O Mode if either is set to I/O
Mode. The underlying joystick type is not changed and reads of this
register will continue to return the last joystick type. Ehether the
joystick is in io mode or not is invisible but this state can be
cleared either through reset or by re-writing the gegister with
joystick type not equal to 111. Recovery time for a normal joystick
read after leaving I/O Mode is at most 64 scan lines.
\end{itemize}



\paragraph{Video Modes}

The video mode can only be changed during the boot process so one
initial read is required of this register during software start up
phase.

The machine timing is identical for the seven VGA modes although the
physical refresh rate of the video output speeds up for each mode in
turn by roughly 1Hz. The internal timing of the machine remains
constant and as close to the original hardware as possible. VGA is a
perfect Amstrad ZX Spectrum 128k +3 for example as far as timing is
concerned across the seven VGA modes.

The effect of this speed up means that mode 0 will execute in one
second of time whereas mode 6 will execute in a shorter time
period. Mode 0 is as close to 50/60 Hz as possible where mode 6 is
closer to 60/70 Hz. That would mean that one second of machine time
for mode 6 will execute in 0.83 seconds of human time when running 50
frames per second at 60Hz.

The eighth mode (mode 7) is used for HDMI timing. Machine
configuration is forced for this mode. Line counts, Ts and various
other settings are set to meet the rigid HDMI timing
specification. For mode 7, 50/60 Hz are rock solid but the original
hardware timing loses Ts across all machines to meet HDMI display
requirements.

Software that was previously written for specific hardware with
hard-coded software timing loops may fail. This is one of the risks of
coding timing loops counting Ts. We saw evidence of this with the
release of the 1985 Sinclair ZX Spectrum 128k+ and the later Amstrad
models as previous software written for the ZX Spectrum 48k/48k+ would
fail when trying to display colour attribute and border effects as the
number of Ts per line was changed from 224Ts (1982 original 48k) to
228Ts (128k models). The ZX Spectrum Next runs slower in HDMI
mode. Demos may fail to display correctly and games may slow down
although setting the Z80 to 7Mhz can solve the game slow down, demos
should be run in VGA mode for maximum compatibility.

Video timing also affects audio output as the sample rate can vary
depending on the output timing method.

The following register allows software to read the video timing mode:

\register{R/W}{11}{Video Timing (writable in config mode only)}
\begin{itemize}
\item bits 7-3 = Reserved, must be 0
\item bits 2-0 = Mode (VGA = 0..6, HDMI = 7)
  \begin{itemize}
  \item 000 = Base VGA timing, clk28 = 28000000
  \item 001 = VGA setting 1, clk28 = 28571429
  \item 010 = VGA setting 2, clk28 = 29464286
  \item 011 = VGA setting 3, clk28 = 30000000
  \item 100 = VGA setting 4, clk28 = 31000000
  \item 101 = VGA setting 5, clk28 = 32000000
  \item 110 = VGA setting 6, clk28 = 33000000
  \item 111 = HDMI, clk28 = 27000000
  \end{itemize}
\item[] 50/60Hz selection depends on bit 2 of register \$05
\item[] Only writable in config mode
\end{itemize}



\paragraph{Machine Type}

The machine type register can be used to provide the number of Ts per
line, line count, dot clock and maximum horizontal COPPER wait.

The dot clock (DC) is the number of Ts per line * 4.

The maximum horizontal COPPER wait (H) is in multiples of 16 clocks.

Video mode 7 (HMDI) overrides the timing.

The following list shows the various parameters that can be gained
from reading the machine register combined with the refresh register
and video mode bits:

\register{R/W}{03}{Machine Type}\\
\begin{itemize}
\item bit 7 (R) = nextreg \$44 second byte indicator
\item bit 7 (W) = allow changes to bits 6-4 (0 on hard reset)
\item bits 6-4 = Display Timing
  \begin{itemize}
  \item 000 = Internal use
  \item 001 = ZX 48k
  \item 010 = ZX 128k/2+
  \item 011 = ZX +2A/+2B/+3
  \item 100 = Pentagon
  \end{itemize}
\item bit 3 = Display Timing user lock control (0 on hard reset)
\item bits 2-0 = Machine type (write on config mode only) determines ROMs loaded
  \begin{itemize}
  \item 000 = Configuration mode
  \item 001 = ZX 48k
  \item 010 = ZX 128k/+2 (Grey)
  \item 011 = ZX +2A/+2BB/+3/Next Native
  \item 100 = Pentagon
  \end{itemize}
\end{itemize}


      
\paragraph{Summary}

Table 5.13 provides a full list of video timing configuration
data:

\begin{table}[h]\centering
  \caption{Summary of Video Modes}
  \csvautotabular{copper/videotiming.csv}
\end{table}




