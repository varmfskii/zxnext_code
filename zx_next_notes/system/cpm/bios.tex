\subsection{BIOS}
\subsubsection{System Initialization Functions}
This section defines the BIOS system initialization routines BOOT,
WBOOT, DEVTBL, DEVINI, and DRVTBL.

\index{BOOT}
BIOS Function 0: BOOT

Get Control from Cold Start Loader and Initialize System\\
Entry Parameters: None\\
Returned Values: None

The BOOT entry point gets control from the Cold Start Loader in Bank 0
and is responsible for basic system initialization. Any remaining
hardware initialization that is not done by the boot ROMS, the Cold
Boot Loader, or the LDRBIOS should be performed by the BOOT routine.

\index{WBOOT}
BIOS Function 1: WBOOT

Get Control When a Warm Start Occurs\\
Entry Parameters: None\\
Returned Values: None

The WBOOT entry point is entered when a warm start occurs. A warm
start is performed whenever a user program branches to location 0000H
or attempts to return to the CCP.

\index{DEVTBL}
BIOS Function 20: DEVTBL

Return Address of Character I/O Table\\
Entry Parameters: None\\
Returned Values: HL=address of Chrtbl

The DEVTBL and DEVINI entry points allow you to support device
assignment with a flexible, yet completely optional system. It
replaces the IOBYTE facility of CP/M 2.2.

\index{DEVINI}
BIOS Function 21: DEVINI

Initialize Character I/O Device\\
Entry Parameters: C=device number, 0-15\\
Returned Values: None

The DEVINI routine initializes the physical character device specified
in register C to the baud rate contained in the appropriate entry of
the CHRTBL.

\index{DRVTBL}
BIOS Function 22: DRVTBL

Return Address of Disk Drive Table\\
Entry Parameters: None\\
Returned Values:
\begin{itemize}
\item[] HL=Address of Drive Table of Disk Parameter Headers (DPH);
  Hashing can utilized if specified by the DPHs Referenced by this
  DRVTBL.
\item[] HL=\$ffff if no Drive Table; GENCPM does not set up buffers.
  Hashing is supported.
\item[] HL=\$fffe if no Drive Table; GENCPM does not set up buffers.
  Hashing is not supported.
\end{itemize}

The first instruction of this subroutine must be an LXI H,<address>
where <address> is one of the above returned values. The GENCPM
utility accesses the address in this instruction to locate the drive
table and the disk parameter data structures to determine which system
configuration to use.

\subsubsection{Character I/O Functions}
This section defines the CP/M 3 character I/O routines CONST, CONIN,
CONOUT, LIST, AUXOUT, AUXIN, LISTST, CONOST, AUXIST, and AUXOST.  CP/M
3 assumes all simple character I/O operations are performed in
eight-bit ASCII, upper and lowercase, with no parity. An ASCII CTRL-Z
(\$1a) denotes an end-of-file condition for an input device.

In CP/M 3, you can direct each of the five logical character devices
to any combination of up to twelve physical devices. Each of the five
logical devices has a 16-bit vector in the System Control Block (SCB)
. Each bit of the vector represents a physical device where bit 15
corresponds to device zero, and bit 4 is device eleven. Bits 0 through
3 are reserved for future system use.

\index{CONST}
BIOS Function 2: CONST

Sample the Status of the Console Input Device\\
Entry Parameters: None\\
Returned value:
\begin{itemize}
\item[] A=\$ff if a console character is ready to read
\item[] A=\$00 if no console character is ready to read
\end{itemize}
 
Read the status of the currently assigned console device and return
\$ff in register A if a character is ready to read, and \$ff in
register A if no console characters are ready.

\index{CONIN}
BIOS Function 3: CONIN

Read a Character from the Console\\
Entry Parameters: None\\
Returned Values: A=Console Character

Read the next console character into register A with no parity. If no
console character is ready, wait until a character is available before
returning.

\index{CONOUT}
BIOS Function 4: CONOUT

Output Character to Console\\
Entry Parameters: C=Console Character\\
Returned Values: None

Send the character in register C to the console output device. The
character is in ASCII with no parity.

\index{LIST}
BIOS Function 5: LIST

Output Character to List Device\\
Entry Parameters: C=Character\\
Returned Values: None

Send the character from register C to the listing device. The
character is in ASCII with no parity.

\index{AUXOUT}
BIOS Function 6: AUXOUT

Output a Character to the Auxiliary Output Device\\
Entry Parameters: C=Character\\
Returned Values: None

Send the character from register C to the currently assigned AUXOUT
device. The character is in ASCII with no parity.

\index{AUXIN}
BIOS Function 7: AUXIN

Read a Character from the Auxiliary Input Device\\
Entry Parameters: None\\
Returned Values: A=Character

Read the next character from the currently assigned AUXIN device into
register A with no parity. A returned ASCII CTRL-Z (\$1a) reports an
end-of-file.

\index{LISTST}
BIOS Function 15: LISTST

Return the Ready Status of the List Device\\
Entry Parameters: None\\
Returned Values:
\begin{itemize}
\item[] A=\$00 if list device is not ready to accept a character
\item[] A=\$ff if list device is ready to accept a character
\end{itemize}

\index{CONOST}
BIOS Function 17: CONOST

Return Output Status of Console\\
Entry Parameters: None\\
Returned Values:
\begin{itemize}
\item[] A=\$ff if ready
\item[] A=\$00 if not ready
\end{itemize}

The CONOST routine checks the status of the console. CONOST returns an
\$ff if the console is ready to display another character. This entry
point allows for full polled handshaking communications support.

\index{AUXIST}
BIOS Function 18: AUXIST

Return Input Status of Auxiliary Port\\
Entry Parameters: None\\
Returned Values:
\begin{itemize}
\item[] A=\$ff if ready
\item[] A=\$00 if not ready
\end{itemize}

The AUXIST routine checks the input status of the auxiliary port. This
entry point allows full polled handshaking for communications support
using an auxiliary port.

\index{AUXOST}
BIOS Function 19: AUXOST

Return Output Status of Auxiliary Port\\
Entry Parameters: None\\
Returned Values:
\begin{itemize}
\item[] A=\$ff if ready
\item[] A=\$00 if not ready
\end{itemize}

The AUXOST routine checks the output status of the auxiliary
port. This routine allows full polled handshaking for communications
support using an auxiliary port.

\subsubsection{Disk I/O Functions}
This section defines the CP/M 3 BIOS disk I/O routines HOME, SELDSK,
SETTRK, SETSEC, SETDMA, READ, WRITE, SECTRN, MULTIO, and FLUSH.

\index{HOME}
BIOS Function 8: HOME

Select Track 00 of the Specified Drive\\
Entry Parameters: None\\
Returned Values: None

Return the disk head of the currently selected disk to the track 00
position. Usually, you can translate the HOME call into a call on
SETTRK with a parameter of 0.

\index{SELDSK}
BIOS Function 9: SELDSK

Select the Specified Disk Drive\\
Entry Parameters:
\begin{itemize}
\item[] C=Disk Drive (0-15)
\item[] E=Initial Select Flag
\end{itemize}
Returned Values:
\begin{itemize}
\item[] HL=Address of Disk Parameter Header (DPH) if drive exists
\item[] HL=0000H if drive does not exist
\end{itemize}
Select the disk drive specified in register C for further operations,
where register C contains 0 for drive A, 1 for drive B, and so on to
15 for drive P. On each disk select, SELDSK must return in HL the base
address of a 25-byte area called the Disk Parameter Header. If there
is an attempt to select a nonexistent drive, SELDSK returns HL=\$0000
as an error indicator.  On entry to SELDSK, you can determine if it is
the first time the specified disk is selected. Bit 0, the least
significant bit in register E, is set to 0 if the drive has not been
previously selected. This information is of interest in systems that
read configuration information from the disk to set up a dynamic disk
definition table.

\index{SETTRK}
BIOS Function 10: SETTRK

Set Specified Track Number\\
Entry Parameters: BC=Track Number\\
Returned Values: None

Register BC contains the track number for a subsequent disk access on
the currently selected drive. Normally, the track number is saved
until the next READ or WRITE occurs.

\index{SETSEC}
BIOS Function 11: SETSEC

Set Specified Sector Number\\
Entry Parameters: BC=Sector Number\\
Returned Values: None

Register BC contains the sector number for the subsequent disk access
on the currently selected drive. This number is the value returned by
SECTRN. Usually, you delay actual sector selection until a READ or
WRITE operation occurs.

\index{SETDMA}
BIOS Function 12: SETDMA

Set Address for Subsequent Disk I/O\\
Entry Parameters: BC=Direct Memory Access Address\\
Returned Values: None

Register BC contains the DMA (Direct Memory Access) address for the
subsequent READ or WRITE operation. For example, if B = \$00 and C =
\$80 when the BDOS calls SETDMA, then the subsequent read operation
reads its data starting at \$80, or the subsequent write operation
gets its data from 80H, until the next call to SETDMA occurs.

\index{READ}
BIOS Function 13: READ

Read a Sector from the Specified Drive\\
Entry Parameters: None\\
Returned Values:
\begin{itemize}
\item[] A=\$00 if no errors occurred
\item[] A=\$01 if nonrecoverable error condition occurred
\item[] A=\$ff if media has changed
\end{itemize}
Assume the BDOS has selected the drive, set the track, set the sector,
and specified the DMA address. The READ subroutine attempts to read
one sector based upon these parameters, then returns one of the error
codes in register A as described above.

If the value in register A is \$00, then CP/M 3 assumes that the disk
operation completed properly. If an error occurs, the BIOS should
attempt several retries to see if the error is recoverable before
returning the error code.

If an error occurs in a system that supports automatic density
selection, the system should verify the density of the drive. If the
density has changed, return a \$ff in the accumulator. This causes the
BDOS to terminate the current operation and relog in the disk.

\index{WRITE}
BIOS Function 14: WRITE

Write a Sector to the Specified Disk\\
Entry Parameters: C=Deblocking Codes\\
Returned Values:
\begin{itemize}
\item[] A=\$00 if no error occurred
\item[] A=\$01 if physical error occurred
\item[] A=\$02 if disk is Read-Only
\item[] A=\$ff if media has changed
\end{itemize}

Write the data from the currently selected DMA address to the
currently selected drive, track, and sector. Upon each call to WRITE,
the BDOS provides the following information in register C:
\begin{itemize}
\item[] 0 = deferred write
\item[] 1 = nondeferred write
\item[] 2 = deferred write to the first sector of a new data block
\end{itemize}

This information is provided for those BIOS implementations that do
blocking/deblocking in the BIOS instead of the BDOS.

\index{SECTRN}
BIOS Function 16: SECTRN

Translate Sector Number Given Translate Table
Entry Parameters:
\begin{itemize}
\item[] BC=Logical Sector Number
\item[] DE=Translate Table Address
\end{itemize}
Returned Values: HL=Physical Sector Number

SECTRN performs logical sequential sector address to physical sector
translation to improve the overall response of CP/M 3.

\index{MULTIO}
BIOS Function 23: MULTIO

Set Count of Consecutive Sectors for READ or WRITE\\
Entry Parameters: C=Multisector Count\\
Returned Values: None

To transfer logically consecutive disk sectors to or from contiguous
memory locations, the BDOS issues a MULTIO call, followed by a series
of READ or WRITE calls. This allows the BIOS to transfer multiple
sectors in a single disk operation. The maximum value of the sector
count is dependent on the physical sector size, ranging from 128 with
128-byte sectors, to 4 with 4096-byte sectors. Thus, the BIOS can
transfer up to 16K directly to or from the TPA with a single
operation.

\index{FLUSH}
BIOS Function 24: FLUSH

Force Physical Buffer Flushing for User-supported Deblocking\\
Entry Parameters: None\\
Returned Values:
\begin{itemize}
\item[] A=\$00 if no error occurred
\item[] A=\$001 if physical error occurred
\item[] A=\$002 if disk is Read-Only
\end{itemize}

The flush buffers entry point allows the system to force physical
sector buffer flushing when your BIOS is performing its own record
blocking and deblocking.  The BDOS calls the FLUSH routine to ensure
that no dirty buffers remain in memory.

\subsection{Memory Select and Move Functions}
This section defines the memory management functions MOVE, XMOVE,
SELMEM, and SETBNK.

\index{MOVE}
BIOS Function 25: MOVE

Memory-to-Memory Block Move\\
Entry Parameters:
\begin{itemize}
\item[] HL=Destination address
\item[] DE=Source address
\item[] BC=Count
\end{itemize}
Returned Values: HL and DE must point to next bytes following move
operation

The BDOS calls the MOVE routine to perform memory to memory block
moves to allow use of the Z80 LDIR instruction or special DMA
hardware, if available. Note that the arguments in HL and DE are
reversed from the Z8O machine instruction, necessitating the use of
XCHG instructions on either side of the LDIR. The BDOS uses this
routine for all large memory copy operations. On return, the HL and DE
registers are expected to point to the next bytes following the move.

Usually, the BDOS expects MOVE to transfer data within the currently
selected bank or common memory. However, if the BDOS calls the XMOVE
entry point before calling MOVE, the MOVE routine must perform an
interbank transfer.

\index{SELMEM}
BIOS Function 27: SELMEM

Select Memory Bank\\
Entry Parameters: A=Memory Bank\\
Returned Values; None

The SELMEM entry point is only present in banked systems. The banked
version of the CP/M 3 BDOS calls SELMEM to select the current memory
bank for further instruction execution or buffer references. You must
preserve or restore all registers other than the accumulator, A, upon
exit.

\index{SETBNK}
BIOS Function 28: SETBNK

Specify Bank for DMA Operation\\
Entry Parameters: A=Memory Bank\\
Returned Values: None

SETBNK only occurs in the banked version of CP/M 3. SETBNK specifies
the bank that the subsequent disk READ or WRITE routine must use for
memory transfers. The BDOS always makes a call to SETBNK to identify
the DMA bank before performing a READ or WRITE call. Note that the
BDOS does not reference banks other than 0 or 1 unless another bank is
specified by the BANK field of a Data Buffer Control Block (BCB).

\index{XMOVE}
BIOS Function 29: XMOVE

Set Banks for Following MOVE\\
Entry Parameters:
\begin{itemize}
\item[] B=destination bank
\item[] C=source bank
\end{itemize}
Returned Values: None

XMOVE is provided for banked systems that support memory-to- memory
DMA transfers over the entire extended address range. Systems with
this feature can have their data buffers located in an alternate bank
instead of in common memory, as is usually required. An XMOVE call
affects only the following MOVE call. All subsequent MOVE calls apply
to the memory selected by the latest call to SELMEM. After a call to
the XMOVE function, the following call to the MOVE function is not
more than 128 bytes of data.

\subsubsection{Clock Support Function}
This section defines the clock support function TIME.

\index{TIME}
BIOS Function 26: TIME

Get and Set Time\\
Entry Parameters: C=Time Get/Set Flag\\
Returned values: None

The BDOS calls the TIME function to indicate to the BIOS whether it
has just set the Time and Date fields in the SCB, or whether the BDOS
is about to get the Time and Date from the SCB. On entry to the TIME
function, a zero in register C indicates that the BIOS should update
the Time and Date fields in the SCB. A \$ff in register C indicates
that the BDOS has just set the Time and Date in the SCB and the BIOS
should update its clock. Upon exit, you must restore register pairs HL
and DE to their entry values.
