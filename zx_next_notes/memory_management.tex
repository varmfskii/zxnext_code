\chapter{Memory Management}
\section{ZX Spectrum 128}
128-style memory management can only alter the bank addressed at
\$c000 (16k-slot 4, or 8k-slot 7-8). The active 16k-bank at \$c000 is
selected by writing the 3 LSBs of the 16k-bank number to the bottom 3
bits of Memory Paging Control (\$7FFD), and the 3 MSBs to the bottom 3
bits of Next Memory Bank Select (\$DFFD). (The reason for the division
is that the original Spectrum 128, having only 128k of memory, only
needed 3 bits.)

On an unexpanded Next, this allows any 16k-bank to be paged in at
\$c000. On an expanded next, there are not enough bits available to
access the banks at the bottom of the expanded memory, so Next memory
management must be used to access these.

If you are using the standard interrupt handler or OS routines, then
any time you write to Memory Paging Control (\$7FFD) you should also
store the value at \$5B5C. Any time you write to Plus 3 Memory Paging
Control (\$1FFD) you should also store the value at \$5B67. There is
no corresponding system variable for the Next-only Next Memory Bank
Select (\$DFFD) and standard OS routines may not support the extended
banks properly.

\paragraph{128 Special Paging Mode}

"Special paging mode" (also called "AllRam mode" or "CP/M mode") is
enabled by writing a value with the LSB set to Plus 3 Memory Paging
Control (\$1FFD). Depending on the 3 low bits of this value a memory
configuration is selected as follows:

\begin{table}[h]\centering
  \caption{Special Paging Modes}
  \csvautotabular{zx128mm.csv}
\end{table}

\section{ZX Spectrum Next}
The 8k-bank accessed in an 8k-slot is selected by writing the 8k-bank
number to the bottom 7 bits of the 8 Next registers from (\$50)
upwards. \$50 addresses 8k-slot 0, \$51 addresses 8k-slot 1, and so
on.

In addition, in 8k-slots 1 and 2 only, the ROM can be paged in by
selecting the otherwise non-existent 8k-page \$FF. Whether the high or
the low 8k of the ROM is mapped is determined by which 8k-slot is
used.

\paragraph{Interactions between paging methods}
Changes made in 128 style and Next style memory management are
synchronized. The most recent change always has priority. This means
that

using 128-style memory management to select a new 16k-bank in 16k-slot
4 will update the MMU registers for the two 8k-slots with the
corresponding 8k-bank numbers.  enabling AllRam mode will update all
of the 8k-bank values with the appropriate 8k-slot numbers. These may
then be overwritten using Next memory management without needing to
alter the value at port \$1FFD.  Since the 128-style memory management
ports are not readable, there is no synchronization applicable in the
other direction.

\paragraph{ROM paging and selection}
\$0000-\$3fff is usually mapped to ROM. This area can only be fully
remapped using Next memory management. ROM is not considered one of
the numbered banks; it is mapped to the two 8k-banks by default, or by
setting their 8k-bank numbers to 255.

The 128k Spectrum has 2 ROM pages. Which of these is mapped is
selected by altering Bit 4 of Memory Paging Control (\$7FFD). The
+2a/+3 has 4 ROM pages; the extra bit needed to select between these
is bit 2 of Plus 3 Memory Paging Control (\$1FFD). This maintains
compatibility with the original machines' ROM paging as long as the
ROM is not paged out.

\paragraph{Paging out ROM}
ROM can be paged out by enabling AllRam mode, or by using Next memory
management. Beware that some programs may assume that they can find
ROM service routines at fixed addresses between \$0000-\$3fff. More
importantly, if the default interrupt mode (IM 1) is set, the Z80 will
jump the program counter to \$0038 every frame expecting to find an
interrupt handler there. If it does not, pain and suffering will
likely result. DI is your friend. On the plus side, this does allow
you to write your own interrupt handler without the nuisance of using
IM 2.

\paragraph{Layer 2 Switching}
Layer 2 switching can allow any 16k-bank to be written to (but not
read) in 16k-slot 1, by writing the 16k-bank number to Layer 2 RAM
Page Register (\$12) and then enabling Layer 2 paging by writing a
value with the LSB set to Layer 2 Access Port (\$123B).

Writing to this area will then write the appropriate area of memory,
whereas reading from it will give the area mapped by other memory
management.

\paragraph{Screen}
16k-Bank 5 is the bank read by the ULA to determine what to show on
screen. The ULA connects directly to the larger memory space ignoring
mapping; the screen is always 16k-Bank 5, no matter where in memory it
is (or if it is switched in at all). Setting bit 3 of Memory Paging
Control (\$7FFD) will have the ULA read from 16k-bank 7 (the "shadow
screen") instead, which can be used as an alternate screen. Beware
that this does not map 16k-bank 7 into RAM; to alter 16k-bank 7 it
must be mapped by other means.
