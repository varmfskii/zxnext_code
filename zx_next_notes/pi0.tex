\chapter{Raspberry Pi0 Acceleration}
The Spectrum Next has a header (with male pins) which can be attached
to a Raspberry Pi Zero. There is a modified version of DietPi called
NextPi which is the standard distro for the Raspberry Pi0
accelerator. Software for the general public should be written
assuming that it will be interfacing with a Pi0 running this distro.

If you are more adventurous, you may choose to use another distro, or
even another accelerator that uses the Raspberry Pi style (40 pin)
expansion bus.  Chief concers when doing this is that you have a
console presented on the UART that defaults to 115,200 bps, you don't
need to login, the machine is configured with a driver to treat the
$I^2S$ interface as a sound card, and the presence of the nextpi
scripts.

The Raspberry Pi 0 has a Broadcom BCM2835 SoC with an ARMv6 core, a
Videocore 4 GPU, and its own 512 MB memory and HDMI output. It has its
own SD card from which it boots. For this application the Pi 0 ships
with a 1GB microSD card containing NextPi a customized version of
DietPi.

The Pi Zero, if installed, is a smart peripheral for the ZX Spectrum
Next. Available interfaces are: low level access to the GPIO pins,
higher level access to standardized I/O interfaces, and use of the Pi
Zero as a sound card.

When using the low level GPIO interface Pi Zero GPIO pins 2-27 can be
configured as either inputs or outputs using nextregs \$90-\$93. If
they are outputs, the output state can be set by writing to nextregs
\$98-\$9b. The current status of the GPIO pins can be read from
nextregs \$98-\$9b whether it is the state driven by the ZX Spectrum
Next or the state drive by some other peripherial attached to the bus
(normally the Raspberry Pi Zero).

Standardized I/O access with the Pi Zero can use the $I^2C$, SPI, or
UART interfaces and is configured using nextreg \$a0. Any enabled port
will disable low level (write) access to the corresponding GPIO
pins.

The $I^2C$ interface is controlled using ports \$103b (SCL) and \$113b
(SDA). This is the same $I^2C$ interface that is used for the optional
Real Time Clock. Interfacing with the Pi Zero over $I^2C$ is
complicated by the fact that it is a master/slave interface, but both
the ZX Spectrum Next and Pi Zero are configured to be bus masters.

The SPI interface is controlled using ports \$e7 (/CS) and \$eb
(/DATA). The SPI interface is shared between the SD card(s), the flash
memory, and the Pi Zero. Interfacing with the Pi Zero over SPI is
complicated by the fact it is a master/slave interface and both the ZX
Spectrum Next and Pi Zero are configured to be bus masters.

The UART interface is controlled by ports \$133b (TX), \$143b (RX),
and \$153b (control). This is the default means of bidirectional
communication between the ZX Spectrum Next and Pi Zero. To use the
UART you must first enable the UART on the GPIO and connecting it to
the Pi Zero (not hats) by setting nextreg \$a0 bits 5 and 4 to 1 and
selecting Pi for the UART by setting port \$153b bit 6 to 1. Assuming
that the serial parameters match (by default both ends are set to
115,200 bps with 8N1 and no flow control) this will give you access to
the serial console on the Pi Zero over the UART.

\begin{verbatim}
;; enable UART connection with Pi Zero
   ld c,$3b
   ld b,$15 ; UART control
;; select Pi on UART control
   in a,(c)
   or $40
   out (c),a
   ld b,$24 ; Next Register Select
   ld a,$a0
   out (c),a
   inc b ; Next Register Data
;; Enable UART on GPIO and select Pi
   in a,(c)
   or $30
   out (c),a
\end{verbatim}

The $I^2S$ sound interface between the ZX Spectrum Next and the Pi Zero
is controlled by nextregs \$a2 and \$a3. Normally, one would control
the Pi through some other channel such as the UART recieve audio from
the Pi to either use as a fulloy programmable sound card or to allow
loading of tape files on the ZX Spectrum Next.

(R/W) \$90 (144) $\Rightarrow$ PI GPIO Output Enable 0
\begin{itemize}
\item bit 7 = Pi GPIO 7
\item bit 6 = Pi GPIO 6
\item bit 5 = Pi GPIO 5
\item bit 4 = Pi GPIO 4
\item bit 3 = Pi GPIO 3
\item bit 2 = Pi GPIO 2
\item bit 1 = Pi GPIO 1 (cannot be enabled)
\item bit 0 = Pi GPIO 0 (cannot be enabled)
\item[] \$00 on soft reset
\end{itemize}

(R/W) \$91 (145) $\Rightarrow$ PI GPIO Output Enable 1
\begin{itemize}
\item bit 7 = Pi GPIO 15
\item bit 6 = Pi GPIO 14
\item bit 5 = Pi GPIO 13
\item bit 4 = Pi GPIO 12
\item bit 3 = Pi GPIO 11
\item bit 2 = Pi GPIO 10
\item bit 1 = Pi GPIO 9
\item bit 0 = Pi GPIO 8
\item[] \$00 on soft reset
\end{itemize}

(R/W) \$92 (146) $\Rightarrow$ PI GPIO Output Enable 2
\begin{itemize}
\item bit 7 = Pi GPIO 23
\item bit 6 = Pi GPIO 22
\item bit 5 = Pi GPIO 21
\item bit 4 = Pi GPIO 20
\item bit 3 = Pi GPIO 19
\item bit 2 = Pi GPIO 18
\item bit 1 = Pi GPIO 17
\item bit 0 = Pi GPIO 16
\item[] \$00 on soft reset
\end{itemize}

(R/W) \$93 (147) $\Rightarrow$ PI GPIO Output Enable 3
\begin{itemize}
\item bits 7-4 = Reserved, must be 0
\item bit 3 = Pi GPIO 27
\item bit 2 = Pi GPIO 26
\item bit 1 = Pi GPIO 25
\item bit 0 = Pi GPIO 24
\item[] \$00 on soft reset
\end{itemize}

(R/W) \$98 (152) $\Rightarrow$ PI GPIO 0
\begin{itemize}
\item bit 7 = Pi GPIO 7
\item bit 6 = Pi GPIO 6
\item bit 5 = Pi GPIO 5
\item bit 4 = Pi GPIO 4
\item bit 3 = Pi GPIO 3
\item bit 2 = Pi GPIO 2
\item bit 1 = Pi GPIO 1 (read only)
\item bit 0 = Pi GPIO 0 (read only)
\item[] \$ff on soft reset
\end{itemize}

(R/W) \$99 (153) $\Rightarrow$ PI GPIO 1
\begin{itemize}
\item bit 7 = Pi GPIO 15
\item bit 6 = Pi GPIO 14
\item bit 5 = Pi GPIO 13
\item bit 4 = Pi GPIO 12
\item bit 3 = Pi GPIO 11
\item bit 2 = Pi GPIO 10
\item bit 1 = Pi GPIO 9
\item bit 0 = Pi GPIO 8
\item[] \$01 on soft reset
\end{itemize}

(R/W) \$9A (154) $\Rightarrow$ PI GPIO 2
\begin{itemize}
\item bit 7 = Pi GPIO 23
\item bit 6 = Pi GPIO 22
\item bit 5 = Pi GPIO 21
\item bit 4 = Pi GPIO 20
\item bit 3 = Pi GPIO 19
\item bit 2 = Pi GPIO 18
\item bit 1 = Pi GPIO 17
\item bit 0 = Pi GPIO 16
\item[] \$00 on soft reset
\end{itemize}

(R/W) \$9B (155) $\Rightarrow$ PI GPIO 3
\begin{itemize}
\item bits 7-4 = Reserved, must be 0
\item bit 3 = Pi GPIO 27
\item bit 2 = Pi GPIO 26
\item bit 1 = Pi GPIO 25
\item bit 0 = Pi GPIO 24
\item[] \$00 on soft reset
\end{itemize}

(R/W) \$A0 (160) $\Rightarrow$ PI Peripheral Enable
\begin{itemize}
\item bits 7-6 = Reserved, must be 0
\item bit 5 = Enable UART on GPIO 14,15 (overrides gpio) (soft reset
  = 0)
\item bit 4 = 0 to connect Rx to GPIO 15, Tx to GPIO 14 (for comm
  with pi hats) (soft reset = 0) = 1 to connect Rx to GPIO 14, Tx to
  GPIO 15 (for comm with pi)
\item bit 3 = Enable $I^2C$ on GPIO 2,3 (override gpio) (soft reset =
  0)
\item bits 2-1 = Reserved, must be 0
\item bit 0 = Enable SPI on GPIO 7,8,9,10,11 (overrides gpio) (soft
  reset = 0)
\end{itemize}

(R/W) \$A2 (162) $\Rightarrow$ PI $I^2S$ Audio Control
\begin{itemize}
\item bits 7-6 = $I^2S$ enable (soft reset = 00)
  \begin{itemize}
  \item 00 = $I^2S$ off
  \item 01 = $I^2S$ is mono source right
  \item 10 = $I^2S$ is mono source left
  \item 11 = $I^2S$ is stereo
  \end{itemize}
\item bit 5 = Reserved, must be 0
\item bit 4 = 0 PCM\_DOUT to pi, PCM\_DIN from pi (hats) (soft reset
  = 0) = 1 PCM\_DOUT from pi, PCM\_DIN to pi (pi)
\item bit 3 = Mute left side (soft reset = 0)
\item bit 2 = Mute right side (soft reset = 0)
\item bit 1 = Slave mode (PCM\_CLK, PCM\_FS supplied externally)
  (soft reset = 0)
\item bit 0 = Direct $I^2S$ audio to EAR on port \$FE (soft reset = 0)
\end{itemize}

(R/W) \$A3 (163) $\Rightarrow$ PI $I^2S$ Clock Divide (Master Mode)
\begin{itemize}
\item bits 7-0 = Clock divide sets sample rate when in master mode
  (soft reset = 11)
\item[] clock divider = 538461 / SampleRateHz - 1
\end{itemize}
