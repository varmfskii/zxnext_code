\chapter{Raspberry Pi0 Acceleration}
The Spectrum Next has a header (with male pins) which can be attached
to a Raspberry Pi Zero. There is a modified version of DietPi called
NextPi which is the standard distro for the Raspberry Pi0
accelerator. Software for the general public should be written
assuming that it will be interfacing with a Pi0 running this distro.

If you are more adventurous, you may choose to use another distro, or
even another accelerator that uses the Raspberry Pi style (40 pin)
expansion bus.  Chief concers when doing this is that you have a
console presented on the UART that defaults to 115,200 bps, you don't
need to login, the machine is configured with a driver to treat the
\iis interface as a sound card, and the presence of the nextpi
scripts.

The Raspberry Pi 0 has a Broadcom BCM2835 SoC with an ARMv6 core, a
Videocore 4 GPU, and its own 512 MB memory and HDMI output. It has its
own SD card from which it boots. For this application the Pi 0 ships
with a 1GB microSD card containing NextPi a customized version of
DietPi.

The Pi Zero, if installed, is a smart peripheral for the ZX Spectrum
Next. Available interfaces are: low level access to the GPIO pins,
higher level access to standardized I/O interfaces, and use of the Pi
Zero as a sound card.

When using the low level GPIO interface Pi Zero GPIO pins 2-27 can be
configured as either inputs or outputs using nextregs \$90-\$93. If
they are outputs, the output state can be set by writing to nextregs
\$98-\$9b. The current status of the GPIO pins can be read from
nextregs \$98-\$9b whether it is the state driven by the ZX Spectrum
Next or the state drive by some other peripherial attached to the bus
(normally the Raspberry Pi Zero).

\input{registers/tbblue/90.tex}
\input{registers/tbblue/91.tex}
\input{registers/tbblue/92.tex}
\input{registers/tbblue/93.tex}
\input{registers/tbblue/98.tex}
\input{registers/tbblue/99.tex}
\input{registers/tbblue/9A.tex}
\input{registers/tbblue/9B.tex}

Standardized I/O access with the Pi Zero can use the \iic, SPI, or
UART interfaces and is configured using nextreg \$a0. Any enabled port
will disable low level (write) access to the corresponding GPIO
pins.

\input{registers/tbblue/A0.tex}

The \iic interface is controlled using ports \$103b (SCL) and \$113b
(SDA). This is the same \iic interface that is used for the optional
Real Time Clock. Interfacing with the Pi Zero over \iic is
complicated by the fact that it is a master/slave interface, but both
the ZX Spectrum Next and Pi Zero are configured to be bus masters.

\input{ports/103B.tex}
\input{ports/113B.tex}

The SPI interface is controlled using ports \$e7 (/CS) and \$eb
(/DATA). The SPI interface is shared between the SD card(s), the flash
memory, and the Pi Zero. Interfacing with the Pi Zero over SPI is
complicated by the fact it is a master/slave interface and both the ZX
Spectrum Next and Pi Zero are configured to be bus masters.

\input{ports/E7.tex}
\input{ports/EB.tex}

The default means of communication between the ZX Next and the Pi is
through the UART interface (see serial communications chapter). In
order to communicate withe the Pi the Pi UART must be connected to the
Pi by setting nextreg \$a0 bits 5 and 4 to 1, selecting the Pi UART by
setting port \$153b bit 6 to 1 and ensuring that both ends are using
matching communication protocols (by default 115,200 bps, 8N1 and no
flow control). On the Pi end the UART is connected to the serial
console.

\lstinputlisting{pi_serial.asm}

The \iis sound interface between the ZX Spectrum Next and the Pi Zero
is controlled by nextregs \$a2 and \$a3. Normally, one would control
the Pi through some other channel such as the UART recieve audio from
the Pi to either use as a fulloy programmable sound card or to allow
loading of tape files on the ZX Spectrum Next.

\input{registers/tbblue/A2.tex}
\input{registers/tbblue/A3.tex}
