\chapter{zxnDMA}
\begin{multicols}{2}
 February 25, 2019  Phoebus Dokos  Off  Hardware, Resources,

The ZX Spectrum Next DMA (zxnDMA)

\section{Overview}

The ZX Spectrum Next DMA (zxnDMA) is a single channel dma device that
implements a subset of the Z80 DMA functionality. The subset is large
enough to be compatible with common uses of the similar Datagear
interface available for standard ZX Spectrum computers and
compatibles. It also adds a burst mode capability that can deliver
audio at programmable sample rates to the DAC device.

\section{Accessing the zxnDMA}
The zxnDMA is mapped to two Read/Write IO Ports 0x0B and 0x6B which
are the same ones used by the Datagear but unlike the Datagear, the
port at 0x6B runs in zxnDMA mode while the second, port 0x0B, is
similar to the MB-02 interface and runs in compatibility mode.

\register{R/W}{0B}{Envelope period fine}


\port{6B}{DMA Control (Next Mode, 3.01.02)}




\section{Description}
The normal Z80 DMA (Z8410) chip is a pipelined device and because of
that it has numerous off-by-one idiosyncrasies and requirements on the
order that certain commands should be carried out. These issues are
not duplicated in the zxnDMA. You can continue to program the zxnDMA
as if it is were a Z8410 DMA device but it can also be programmed in a
simpler manner.

The single channel of the zxnDMA chip consists of two ports named A
and B. Transfers can occur in either direction between ports A and B,
each port can describe a target in memory or IO, and each can be
configured to autoincrement, autodecrement or stay fixed after a byte
is transferred.

A special feature of the zxnDMA can force each byte transfer to take a
fixed amount of time so that the zxnDMA can be used to deliver sampled
audio.

\section{Modes of Operation}
The zxnDMA can operate in a z80-dma compatibility mode.

The z80-dma compatibility mode is selected by setting bit 6 of nextreg
\$06. In this mode, all transfers involve length+1 bytes which is the
same behaviour as the z80-dma chip. In zxn-dma mode, the transfer
length is exactly the number of bytes programmed. This mode is mainly
present to accommodate existing spectrum software that uses the
z80-dma and for cp/m programs that may have a z80-dma option.

The zxnDMA can also operate in either burst or continuous modes.

Continuous mode means the DMA chip runs to completion without allowing
the CPU to run. When the CPU starts the DMA, the DMA operation will
complete before the CPU executes its next instruction.

Burst mode nominally means the DMA lets the CPU run if either port is
not ready. This condition can't happen in the zxnDMA chip except when
operated in the special fixed time transfer mode. In this mode, the
zxnDMA will let the CPU run while it waits for the fixed time to
expire between bytes transferred.

Note that there is no byte transfer mode as in the Z80 DMA.

\section{Programming the zxnDMA}
Like the Z80 DMA chip, the zxnDMA has seven write registers named
WR0-WR6 that control the device. Each register WR0-WR6 can have zero
or more parameters associated with it.

In a first write to the zxnDMA port, the write value is compared
against a bitmask to determine which of the WR0-WR6 is the
target. Remaining bits in the written value can contain data as well
as a list of associated parameter bits. The parameter bits determine
if further writes are expected to deliver parameter values. If there
are multiple parameter bits set, the expected order of parameter
values written is determined by parameter bit position from right to
left (bit 0 through bit 7). Once all parameters are written, the
zxnDMA again expects a regular register write selecting WR0-WR6.

The table X.Y describes the registers and the bitmask required to
select them on the zxnDMA.

\sinset
\begin{table}[h]\centering
  \caption{zxnDMA Registers}
  \csvautotabular{zxndma/registers.csv}
\end{table}
\einset

\section{zxnDMA Registers}
\sinset
These are described below following the same convention used by Zilog
for its DMA chip:

\paragraph{WR0 – Write Register Group 0}
\begin{verbatim}
D7  D6  D5  D4  D3  D2  D1  D0  BASE REGISTER BYTE
 0   |   |   |   |   |   |   |
     |   |   |   |   |   0   0  Do not use
     |   |   |   |   |   0   1  Transfer (Prefer this for Z80 DMA compatibility)
     |   |   |   |   |   1   0  Do not use (Behaves like Transfer, Search on Z80
     |   |   |   |   |                       DMA)
     |   |   |   |   |   1   1  Do not use (Behaves like Transfer, Search/Trans-
     |   |   |   |   |                      fer on Z80 DMA)
     |   |   |   |   0 = Port B -> Port A (Byte transfer direction)
     |   |   |   |   1 = Port A -> Port B
     |   |   |   V
D7  D6  D5  D4  D3  D2  D1  D0  PORT A STARTING ADDRESS (LOW BYTE)
     |   |   V
D7  D6  D5  D4  D3  D2  D1  D0  PORT A STARTING ADDRESS (HIGH BYTE)
     |   V
D7  D6  D5  D4  D3  D2  D1  D0  BLOCK LENGTH (LOW BYTE)
     V
D7  D6  D5  D4  D3  D2  D1  D0  BLOCK LENGTH (HIGH BYTE)
\end{verbatim}

Several registers are accessible from WR0. The first write to WR0 is
to the base register byte. Bits D6:D3 are optionally set to indicate
that associated registers in this group will be written next. The
order the writes come in are from D3 to D6 (right to left). For
example, if bits D6 and D3 are set, the next two writes will be
directed to PORT A STARTING ADDRESS LOW followed by BLOCK LENGTH HIGH.

\paragraph{WR1 – Write Register Group 1}
\begin{verbatim}
D7  D6  D5  D4  D3  D2  D1  D0  BASE REGISTER BYTE
 0   |   |   |   |   1   0   0
     |   |   |   |
     |   |   |   0 = Port A is memory
     |   |   |   1 = Port A is IO
     |   |   |
     |   0   0 = Port A address decrements
     |   0   1 = Port A address increments
     |   1   0 = Port A address is fixed
     |   1   1 = Port A address is fixed
     |
     V
D7  D6  D5  D4  D3  D2  D1  D0  PORT A VARIABLE TIMING BYTE
 0   0   0   0   0   0   |   |
                         0   0 = Cycle Length = 4
                         0   1 = Cycle Length = 3
                         1   0 = Cycle Length = 2
                         1   1 = Do not use
\end{verbatim}

The cycle length is the number of cycles used in a read or write
 operation. The first cycle asserts signals and the last cycle
 releases them. There is no half cycle timing for the control signals.

\paragraph{WR2 – Write Register Group 2}
\begin{verbatim}
D7  D6  D5  D4  D3  D2  D1  D0  BASE REGISTER BYTE
 0   |   |   |   |   0   0   0
     |   |   |   |
     |   |   |   0 = Port B is memory
     |   |   |   1 = Port B is IO
     |   |   |
     |   0   0 = Port B address decrements
     |   0   1 = Port B address increments
     |   1   0 = Port B address is fixed
     |   1   1 = Port B address is fixed
     |
     V
D7  D6  D5  D4  D3  D2  D1  D0  PORT B VARIABLE TIMING BYTE
 0   0   |   0   0   0   |   |
         |               0   0 = Cycle Length = 4
         |               0   1 = Cycle Length = 3
         |               1   0 = Cycle Length = 2
         |               1   1 = Do not use
         |
         V
D7  D6  D5  D4  D3  D2  D1  D0  ZXN PRESCALAR (FIXED TIME TRANSFER)
\end{verbatim}

The ZXN PRESCALAR is a feature of the zxnDMA implementation. If
non-zero, a delay will be inserted after each byte is transferred such
that the total time needed for each transfer is determined by the
prescalar. This works in both the continuous mode and the burst
mode. If the DMA is operated in burst mode, the DMA will give up any
waiting time to the CPU so that the CPU can run while the DMA is idle.

The rate of transfer is given by the formula ``Frate = 875kHz /
prescalar'' or, rearranged, ``prescalar = 875kHz / Frate''. The formula
is framed in terms of a sample rate (Frate) but Frate can be inverted
to set a transfer time for each byte instead. The 875kHz constant is a
nominal value assuming a 28MHz system clock; the system clock actually
varies from this depending on the video timing selected by the user
(HDMI, VGA0-6) so for complete accuracy the constant should be
prorated according to documentation for nextreg \$11.

In a DMA audio setting, selecting a sample rate of 16kHz would mean
setting the prescalar value to 55. This sample period is constant
across changes in CPU speed.

\paragraph{WR3 – Write Register Group 3}
\begin{verbatim}
D7  D6  D5  D4  D3  D2  D1  D0  BASE REGISTER BYTE
 1   |   0   0   0   0   0   0
     |
     1 = DMA Enable
\end{verbatim}

The Z80 DMA defines more fields but they are ignored by the zxnDMA.

The two other registers defined by the Z80 DMA in this group on D4 and
D3 are implemented by the zxnDMA but they do nothing.

It is preferred to start the DMA by writing an Enable DMA command to
WR6.

\paragraph{WR4 – Write Register Group 4}
\begin{verbatim}
D7  D6  D5  D4  D3  D2  D1  D0  BASE REGISTER BYTE
 1   |   |   0   |   |   0   1
     |   |       |   |
     0   0 = Do not use (Behaves like Continuous mode, Byte mode on Z80 DMA)
     0   1 = Continuous mode
     1   0 = Burst mode
     1   1 = Do not use
                 |   |
                 |   V
D7  D6  D5  D4  D3  D2  D1  D0  PORT B STARTING ADDRESS (LOW BYTE)
                 |
                 V
D7  D6  D5  D4  D3  D2  D1  D0  PORT B STARTING ADDRESS (HIGH BYTE)
\end{verbatim}

The Z80 DMA defines three more registers in this group through D4 that
define interrupt behaviour. Interrups and pulse generation are not
implemented in the zxnDMA nor are these registers available for
writing.

\paragraph{WR5 – Write Register Group 5}
\begin{verbatim}
D7  D6  D5  D4  D3  D2  D1  D0  BASE REGISTER BYTE
 1   0   |   |   0   0   1   0
         |   |
         |   0 = /ce only
         |   1 = /ce & /wait multiplexed
         |
         0 = Stop on end of block
         1 = Auto restart on end of block
\end{verbatim}
The /ce \& /wait mode is implemented in the zxnDMA but is not currently
used. This mode has an external device using the DMA's /ce pin to
insert wait states during the DMA's transfer.

The auto restart feature causes the DMA to automatically reload its
source and destination addresses and reset its byte counter to zero to
repeat the last transfer when a previous one is finished.

\paragraph{WR6 – Command Register}
\begin{verbatim}
D7  D6  D5  D4  D3  D2  D1  D0  BASE REGISTER BYTE
 1   ?   ?   ?   ?   ?   1   1
     |   |   |   |   |
     1   0   0   0   0 = \$C3 = Reset
     1   0   0   0   1 = \$C7 = Reset Port A Timing
     1   0   0   1   0 = \$CB = Reset Port B Timing
     0   1   1   1   1 = \$BF = Read Status Byte
     0   0   0   1   0 = \$8B = Reinitialize Status Byte
     0   1   0   0   1 = \$A7 = Initialize Read Sequence
     1   0   0   1   1 = \$CF = Load
     1   0   1   0   0 = \$D3 = Continue
     0   0   0   0   1 = \$87 = Enable DMA
     0   0   0   0   0 = \$83 = Disable DMA
 +-- 0   1   1   1   0 = \$BB = Read Mask Follows
 |
D7  D6  D5  D4  D3  D2  D1  D0  READ MASK
 0   |   |   |   |   |   |   |
     |   |   |   |   |   |   V
D7  D6  D5  D4  D3  D2  D1  D0  Status Byte
     |   |   |   |   |   |
     |   |   |   |   |   V
D7  D6  D5  D4  D3  D2  D1  D0  Byte Counter Low
     |   |   |   |   |
     |   |   |   |   V
D7  D6  D5  D4  D3  D2  D1  D0  Byte Counter High
     |   |   |   |
     |   |   |   V
D7  D6  D5  D4  D3  D2  D1  D0  Port A Address Low
     |   |   |
     |   |   V
D7  D6  D5  D4  D3  D2  D1  D0  Port A Address High
     |   |
     |   V
D7  D6  D5  D4  D3  D2  D1  D0  Port B Address Low
     |
     V
D7  D6  D5  D4  D3  D2  D1  D0  Port B Address High
\end{verbatim}
Unimplemented Z80 DMA commands are ignored.

Prior to starting the DMA, a LOAD command must be issued to copy the
Port A and Port B addresses into the DMA's internal pointers. Then an
ìEnable DMAî command is issued to start the DMA.

The ìContinueî command resets the DMA’s byte counter so that a
following ìEnable DMAî allows the DMA to repeat the last transfer but
using the current internal address pointers. I.e. it continues from
where the last copy operation left off.

Registers can be read via an IO read from the DMA port after setting
the read mask. (At power up the read mask is set to \$7f). Register
values are the current internal dma counter values. So ìPort Address A
Lowî is the lower 8-bits of Port A’s next transfer address. Once the
end of the read mask is reached, further reads loop around to the
first one.

The format of the DMA status byte is as follows:

00E1101T

E is set to 0 if the total block length has been transferred
at least once.

T is set to 1 if at least one byte has been transferred.
\einset

\paragraph{Operating speed}

The zxnDMA operates at the same speed as the CPU, that is 3.5MHz, 7MHz
or 14MHz. This is a contended clock that is modified by the ULA and
the auto-slowdown by Layer2.

Auto-slowdown occurs without user intervention if speed exceeds 7Mhz
and the active Layer2 display is being generated (higher speed
operation resumes when the active Layer2 display is not
generated). Programmers do NOT need to account for speed differences
regarding DMA transfers as this happens automatically.

Because of this, the cycle lengths for Ports A and B can be set to
their minimum values without ill effects. The cycle lengths specified
for Ports A and B are intended to selectively slow down read or write
cycles for hardware that cannot operate at the DMA's full speed.

\paragraph{The DMA and Interrupts}

The zxnDMA cannot currently generate interrupts.

The other side of this is that while the DMA controls the bus, the Z80
cannot respond to interrupts. On the Z80, the nmi interrupt is edge
triggered so if an nmi occurs the fact that it occurred is stored
internally in the Z80 so that it will respond when it is woken up. On
the other hand, maskable interrupts are level triggered. That is, the
Z80 must be active to regularly sample the /INT line to determine if a
maskable interrupt is occurring. On the Spectrum and the ZX Next, the
ULA (and line interrupt) are only asserted for a fixed amount of time
~30 cycles at 3.5MHz. If the DMA is executing a transfer while the
interrupt is asserted, the CPU will not be able to see this and it
will most likely miss the interrupt. In burst mode, the CPU will never
miss these interrupts, although this may change if multiple channels
are implemented.

\section{Programming examples}

A simple way to program the DMA is to walk down the list of registers
WR0-WR5, sending desired settings to each. Then start the DMA by
sending a LOAD command followed by an ENABLE\_DMA command to WR6. Once
more familiar with the DMA, you will discover that the amount of
information sent can be reduced to what changes between transfers.

\sinset
\begin{enumerate}
\item Assembly

  Short example program to DMA memory to the screen then DMA a sprite
  image from memory to sprite RAM, and then showing said sprite scroll
  across the screen.

\begin{verbatim}
;------------------------------------------------------------------------------
device zxspectrum48
;------------------------------------------------------------------------------
; DEFINE testing
;------------------------------------------------------------------------------
; DMA (Register 6)
;
;------------------------------------------------------------------------------
;zxnDMA programming example
;------------------------------------------------------------------------------
;(c) Jim Bagley
;------------------------------------------------------------------------------
DMA_RESET equ $c3
DMA_RESET_PORT_A_TIMING equ $c7
DMA_RESET_PORT_B_TIMING equ $cb
DMA_LOAD equ $cf ; %11001111
DMA_CONTINUE equ $d3
DMA_DISABLE_INTERUPTS equ $af
DMA_ENABLE_INTERUPTS equ $ab
DMA_RESET_DISABLE_INTERUPTS equ $a3
DMA_ENABLE_AFTER_RETI equ $b7
DMA_READ_STATUS_BYTE equ $bf
DMA_REINIT_STATUS_BYTE equ $8b
DMA_START_READ_SEQUENCE equ $a7
DMA_FORCE_READY equ $b3
DMA_DISABLE equ $83
DMA_ENABLE equ $87
DMA_WRITE_REGISTER_COMMAND equ $bb
DMA_BURST equ %11001101
DMA_CONTINUOUS equ %10101101
ZXN_DMA_PORT equ $6b
SPRITE_STATUS_SLOT_SELECT equ $303B
SPRITE_IMAGE_PORT equ $5b
SPRITE_INFO_PORT equ $57
;------------------------------------------------------------------------------

IFDEF testing
org $6000
ELSE
org $2000
ENDIF

start
ld hl,$0000
ld de,$4000
ld bc,$800
call TransferDMA ; copy some random data to the screen pointing
; to ROM for now, for the purpose of showing
; how to do a DMA copy.
ld a,0 ; sprite image number we want to update
ld bc,SPRITE_STATUS_SLOT_SELECT
out (c),a ; set the sprite image number
ld bc,1*256 ; number to transfer (1)
ld hl,testsprite ; from
call TransferDMASprite ; transfer to sprite ram

nextreg 21,1 ; turn sprite on. for more info on this check
; out https://www.specnext.com/tbblue-io-port-system/
ld de,0
ld (xpos),de ; set initial X position ( doesn't need it for
; this demo, but if you run the .loop again it
; will continue from where it was
ld a,$20
ld (ypos),a ; set initial Y position

.loop
ld a,0 ; sprite number we want to position
ld bc,SPRITE_STATUS_SLOT_SELECT
out (c),a
ld de,(xpos)
ld hl,(ypos) ; ignores H so doing this rather than
; ld a,(ypos):ld l,a
ld bc,(image) ; not flipped or palette shifted
call SetSprite

halt

ld de,(xpos)
inc de
ld (xpos),de
ld a,d
cp $01
jr nz,.loop ; if high byte of xpos is not 1 (right of
; screen )
ld a,e
cp $20+1
jr nz,.loop ; if low byte is not $21 just off the right of
; the screen, $20 is off screen but as the
; INC DE is just above and not updated sprite
; after it, it needs to be $21
xor a
ret ; return back to basic with OK

xpos dw 0 ; x position
ypos db 0 ; y position
; these next two BITS and IMAGE are swapped
; as bits needs to go into B register image
; db 0+$80 ; use image 0 (for the image we
; transfered)+$80 to set the sprite to active
bits db 0 ; not flipped or palette shifted

c1 = %11100000
c2 = %11000000
c3 = %10100000
c4 = %10000000
c5 = %01100000
c6 = %01000000
c7 = %00100000
c8 = %00000000

testsprite
db c1,c1,c1,c1,c1,c1,c1,c1,c1,c1,c1,c1,c1,c1,c1,c1
db c1,c2,c2,c2,c2,c2,c2,c2,c2,c2,c2,c2,c2,c2,c2,c1
db c1,c2,c3,c3,c3,c3,c3,c3,c3,c3,c3,c3,c3,c3,c2,c1
db c1,c2,c3,c4,c4,c4,c4,c4,c4,c4,c4,c4,c4,c3,c2,c1
db c1,c2,c3,c4,c5,c5,c5,c5,c5,c5,c5,c5,c4,c3,c2,c1
db c1,c2,c3,c4,c5,c6,c6,c6,c6,c6,c6,c5,c4,c3,c2,c1
db c1,c2,c3,c4,c5,c6,c7,c7,c7,c7,c6,c5,c4,c3,c2,c1
db c1,c2,c3,c4,c5,c6,c7,c8,c8,c7,c6,c5,c4,c3,c2,c1
db c1,c2,c3,c4,c5,c6,c7,c8,c8,c7,c6,c5,c4,c3,c2,c1
db c1,c2,c3,c4,c5,c6,c7,c7,c7,c7,c6,c5,c4,c3,c2,c1
db c1,c2,c3,c4,c5,c6,c6,c6,c6,c6,c6,c5,c4,c3,c2,c1
db c1,c2,c3,c4,c5,c5,c5,c5,c5,c5,c5,c5,c4,c3,c2,c1
db c1,c2,c3,c4,c4,c4,c4,c4,c4,c4,c4,c4,c4,c3,c2,c1
db c1,c2,c3,c3,c3,c3,c3,c3,c3,c3,c3,c3,c3,c3,c2,c1
db c1,c2,c2,c2,c2,c2,c2,c2,c2,c2,c2,c2,c2,c2,c2,c1
db c1,c1,c1,c1,c1,c1,c1,c1,c1,c1,c1,c1,c1,c1,c1,c1

;-------------------------------------------------
; de = X
; l = Y
; b = bits
; c = sprite image
SetSprite
push bc
ld bc,SPRITE_INFO_PORT
out (c),e ; Xpos
out (c),l ; Ypos
pop hl
ld a,d
and 1
or h
out (c),a
ld a,l:or $80
out (c),a ; image
ret

;--------------------------------
; hl = source
; de = destination
; bc = length
;--------------------------------
TransferDMA
di
ld (DMASource),hl
ld (DMADest),de
ld (DMALength),bc
ld hl,DMACode
ld b,DMACode_Len
ld c,ZXN_DMA_PORT
otir
ei
ret

DMACode db DMA_DISABLE
db %01111101 ; R0-Transfer mode, A -> B, write adress
; + block length
DMASource dw 0 ; R0-Port A, Start address
; (source address)
DMALength dw 0 ; R0-Block length (length in bytes)
db %01010100 ; R1-write A time byte, increment, to
; memory, bitmask
db %00000010 ; 2t
db %01010000 ; R2-write B time byte, increment, to
; memory, bitmask
db %00000010 ; R2-Cycle length port B
db DMA_CONTINUOUS ; R4-Continuous mode (use this for block
; transfer), write dest adress
DMADest dw 0 ; R4-Dest address (destination address)
db %10000010 ; R5-Restart on end of block, RDY active
; LOW
db DMA_LOAD ; R6-Load
db DMA_ENABLE ; R6-Enable DMA

DMACode_Len equ $-DMACode

;------------------------------------------------------------------------------
; hl = source
; bc = length
; set port to write to with TBBLUE_REGISTER_SELECT
; prior to call
;------------------------------------------------------------------------------
TransferDMAPort
di
ld (DMASourceP),hl
ld (DMALengthP),bc
ld hl,DMACodeP
ld b,DMACode_LenP
ld c,ZXN_DMA_PORT
otir
ei
ret

DMACodeP db DMA_DISABLE
db %01111101 ; R0-Transfer mode, A -> B, write adress
; + block length
DMASourceP dw 0 ; R0-Port A, Start address (source address)
DMALengthP dw 0 ; R0-Block length (length in bytes)
db %01010100 ; R1-read A time byte, increment, to
; memory, bitmask
db %00000010 ; R1-Cycle length port A
db %01101000 ; R2-write B time byte, increment, to
; memory, bitmask
db %00000010 ; R2-Cycle length port B
db %10101101 ; R4-Continuous mode (use this for block
; transfer), write dest adress
dw $253b ; R4-Dest address (destination address)
db %10000010 ; R5-Restart on end of block, RDY active
; LOW
db DMA_LOAD ; R6-Load
db DMA_ENABLE ; R6-Enable DMA

DMACode_LenP equ $-DMACodeP
;------------------------------------------------------------------------------
; hl = source
; bc = length
;------------------------------------------------------------------------------
TransferDMASprite
di
ld (DMASourceS),hl
ld (DMALengthS),bc
ld hl,DMACodeS
ld b,DMACode_LenS
ld c,ZXN_DMA_PORT
otir
ei
ret

DMACodeS db DMA_DISABLE
db %01111101 ; R0-Transfer mode, A -> B, write adress
; + block length
DMASourceS dw 0 ; R0-Port A, Start address (source address)
DMALengthS dw 0 ; R0-Block length (length in bytes)
db %01010100 ; R1-read A time byte, increment, to
; memory, bitmask
db %00000010 ; R1-Cycle length port A
db %01101000 ; R2-write B time byte, increment, to
; memory, bitmask
db %00000010 ; R2-Cycle length port B
db %10101101 ; R4-Continuous mode (use this for block
; transfer), write dest adress
dw SPRITE_IMAGE_PORT ; R4-Dest address (destination address)
db %10000010 ; R5-Restart on end of block, RDY active
; LOW
db DMA_LOAD ; R6-Load
db DMA_ENABLE ; R6-Enable DMA
DMACode_LenS equ $-DMACodeS
;------------------------------------------------------------------------------
; de = dest, a = fill value, bc = lenth
;------------------------------------------------------------------------------
DMAFill
di
ld (FillValue),a
ld (DMACDest),de
ld (DMACLength),bc
ld hl,DMACCode
ld b,DMACCode_Len
ld c,ZXN_DMA_PORT
otir
ei
ret

FillValue db 22
DMACCode db DMA_DISABLE
db %01111101
DMACSource dw FillValue
DMACLength dw 0
db %00100100,%00010000,%10101101
DMACDest dw 0
db DMA_LOAD,DMA_ENABLE
DMACCode_Len equ $-DMACCode

;------------------------------------------------------------------------------
; End of file
;------------------------------------------------------------------------------

IFDEF testing
savesna "dmatest.sna",start
ELSE
fin
savebin "DMATEST",start,fin-start
ENDIF
\end{verbatim}
\end{enumerate}
\einset
\end{multicols}
