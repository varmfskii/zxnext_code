\section {Memory Management}

There are a number of different systems for controling memory papping
into the 64k memory space of the Z80N CPU in the ZX Next: ZX Next
native memory paging, ZX Spectrum 128, ZX Spectrum +3, divMMC, and
Multiface.

\subsection{Default Layout}
The default mapping of memory banks is the same as on 128k Spectrum
models with a ROM0 (128k editor and menu system) mapped in at
\$0000-\$3FFF, bank 5 at \$4000-\$7FFF, bank 2 at \$8000-\$BFFF, and
bank 0 at \$C000-\$FFFF.

\subsection{RAM}

\paragraph{ZX Spectrum Next Native}
Registers \$50 to \$57 control the which SRAM pages are in each of the
eight memory slots.  Registers \$50 and \$51 support the special value
\$FF which indicates that the currently selected ROM is to be mapped
into slots 0 and/or 1 (\$0000-\$3FFF).

\input{registers/tbblue/50.tex}
\input{registers/tbblue/51.tex}
\input{registers/tbblue/52.tex}
\input{registers/tbblue/53.tex}
\input{registers/tbblue/54.tex}
\input{registers/tbblue/55.tex}
\input{registers/tbblue/56.tex}
\input{registers/tbblue/57.tex}

In addition the ZX Next has special controls which allow the data area
for Layer 2 to be overlaied on memory in a fashion that permits
selective read or write access. For details see the section on Layer 2
video.

\paragraph{ZX Spectrum 128}
In addition to the native memory management, the ZX Next supports a
memory management system that is an expanded, and backward compatible,
version of the the system used on earlier ZX Spectrum models. This
system uses registers \$1FFD, \$7FFD, and \$DFFD.

\input{ports/1FFD.tex}
\port{7FFD}{Memory Paging Control}
\begin{itemize}
\item[] bits 7-6 = Extra two bits for 16k RAM bank if in Pentagon 512k/1024k
        mapping mode (nextreg \$8f)
\item[] bit 5 = Lock memory paging
\item[] bit 4 = low bit of ROM Select (high bit is in Port \$1FFD) 
  \begin{itemize} \item[] 00 = ROM0 = 128k editor and menu
  system \item[] 01 = ROM1 = 128k syntax checker \item[] 10 = ROM2 =
  +3DOS \item[] 11 = ROM3 = 48k BASIC \end{itemize}
\item[] bit 3 = Shadow screen toggle
\item[] bits 2-0 = LSB of Bank number for slot 4 (MSB is in Port \$DFFD)
\end{itemize}
Disable with bit 5 port \$7FFD


\input{ports/DFFD.tex}

\subparagraph{Spectrum 128 Standard Paging}

128-style memory management can only alter the bank addressed at
\$c000 (16k-slot 4, or 8k-slots 7-8). The active 16k-bank at \$c000 is
selected by writing the 3 LSBs of the 16k-bank number to the bottom 3
bits of Memory Paging Control (\$7FFD), and the 4 MSBs to the bottom 4
bits of Next Memory Bank Select (\$DFFD). (The reason for the division
is that the original Spectrum 128, having only 128k of memory, only
needed 3 bits.)

If you are using the standard interrupt handler or OS routines, then
any time you write to Memory Paging Control (\$7FFD) you should also
store the value at \$5B5C. Any time you write to Plus 3 Memory Paging
Control (\$1FFD) you should also store the value at \$5B67. There is
no corresponding system variable for the Next-only Next Memory Bank
Select (\$DFFD) and standard OS routines may not support the extended
banks properly.

\subparagraph{Paging out ROM}

ROM can be paged out by enabling AllRam mode, or by using Next memory
management. Beware that some programs may assume that they can find
ROM service routines at fixed addresses between \$0000-\$3fff. More
importantly, if the default interrupt mode (IM 1) is set, the Z80 will
jump the program counter to \$0038 every frame expecting to find an
interrupt handler there. If it does not, pain and suffering will
likely result. DI is your friend. On the plus side, this does allow
you to write your own interrupt handler without the nuisance of using
IM 2.

\subparagraph{Spectrum 128 Special Paging}

``Special paging mode'' (also called ``AllRam mode'' or ``CP/M mode'')
is enabled by writing a value with the LSB set to Plus 3 Memory Paging
Control (\$1FFD). Depending on the 3 low bits of this value a memory
configuration is selected as follows:

\begin{table}[ht]\centering
  \caption{Special Paging Modes}
  \csvautotabular{memory/zx128mm.csv}
\end{table}

\subsection{ROM}

The ZX Spectrum Next had several ROMS: ROM0 (16k) - 128k editor and
menu system, ROM1 (16k) - 128k syntax checker, ROM2 (16k) - +3DOS,
ROM3 (16k) - 48k BASIC, divMMC/esxDOS ROM (8k), divMMC RAM (128k),
Multiface ROM (8k) and Alternate ROM (16k).

\paragraph{ZX Next native}

Slots 0 and 1 select use by ROM by selecting page \$FF. Which what ROM
is mapped in is determined by the other memory management system. If
the rest of the system selected the 48k ROM, Nextreg \$8C determines
whether the actual 48k ROM, or the ZX Next Alternate ROM is in use. In
addition, it is possible to enable writing to the Alternate ROM.

\input{registers/tbblue/8C.tex}

\paragraph{ZX Spectrum 128k}

\subparagraph{ROM paging and selection}

\$0000-\$3fff is usually mapped to ROM. This area can only be fully
remapped using Next memory management. ROM is not considered one of
the numbered banks; it is mapped to the two 8k-banks by default, or by
setting their 8k-bank numbers to 255.

The 128k Spectrum has 2 ROM pages. Which of these is mapped is
selected by altering Bit 4 of Memory Paging Control (\$7FFD). The
+2a/+3 has 4 ROM pages; the extra bit needed to select between these
is bit 2 of Plus 3 Memory Paging Control (\$1FFD). This maintains
compatibility with the original machines' ROM paging as long as the
ROM is not paged out.

\paragraph{divMMC}

The divMMC ROM mapping can take priority when it is enabled by port
\$E3 or, when automapping has been enabled by nextreg \$06, when it
has been automapped due to reading one of the appropriate
addresses. Port \$E3 also controls whether the divMMC maps the esxDOS
ROM or divMMC RAM page 3 into slot 0 and which divMMC RAM page is
mapped into slot 1.

\port{E3}{divMMC Control}\\     
Disable with bit 2 of Nextreg \$09
\begin{itemize}
\item[] bit 7 = conmem, enable divMMC memory
\item[] bit 6 = mapram, enable divMMC allRAM mode
\item[] bits 5-4 = reserved
\item[] bits 3-0 = bank, selected divMMC ram bank for \$2000-\$3FFF region
\item conmem can be used to manually control divMMC mapping. When enabled\\
\$0000-\$1FFF contains esxDOS ROM or esxDOS page 3\\
\$2000-\$3FFF contains esxDOS RAM page selected by bits 3-0
\item divMMC automatically maps itself in when instruction fetches hit
specific addresses in the ROM. When this happens, the esxDOS ROM (or divMMC
bank 3 if mapram is set) appears in \$0000-\$1FFF and the selected divMMC
bank appears as RAM in \$2000-\$3FFF
\item bit 6 can only be set, once set only a power cycle can reset it.
nextreg \$09 bit 3 can be set to reset this bit.
\end{itemize}
divMMC automapping is normally disabled by NextZXOS see nextreg \$06 bit 4.\\

\register{R/W}{06}{Peripheral 2 Settings}
\begin{itemize}
\item bit 7 = Enable F8 cpu speed hotkey (soft reset = 1)
\item bit 6 = Divert BEEP only to internal speaker (hard reset = 0)
\item bit 5 = Enable F3 50/60 Hz hotkey (soft reset = 1)
\item bit 4 = Enable divmmc nmi by DRIVE button (hard reset = 0)
\item bit 3 = Enable multiface nmi by M1 button (hard reset = 0)
\item bit 2 = PS/2 mode (config mode only)
\begin{itemize}
\item 0 = keyboard primary
\item 1 = mouse primary
\end{itemize}
\item bits 1-0 = Audio chip mode
\begin{itemize}
\item 00 = YM
\item 01 = AY
\item 10 = ZXN-8950 (3.02.00)
\item 11 = Hold all AY in reset
\end{itemize}
\end{itemize}



\paragraph{Multiface}

Need to find useful docs on the Multiface memory.

9f 1-In, 128-In2\\
1f 1-Out\\
bf 128-In, 3-Out\\
3f 128-Out, 3-In, 3-button\\
7f3f 3-7ffd\\
1f3f 3-1ffd

\section{Interactions between paging methods}

Changes made in 128 style and Next style memory management are
synchronized. The most recent change always has priority. This means
that

using 128-style memory management to select a new 16k-bank in 16k-slot
4 will update the MMU registers for the two 8k-slots with the
corresponding 8k-bank numbers.  enabling AllRam mode will update all
of the 8k-bank values with the appropriate 8k-slot numbers. These may
then be overwritten using Next memory management without needing to
alter the value at port \$1FFD.  Since the 128-style memory management
ports are not readable, there is no synchronization applicable in the
other direction.
