\chapter{Basic Input/Output}

The basic I/O (human interface) system of the ZX Spectrum Next
supports keyboard, mouse and game controllers. These all extend the
functionality found on common ZX Spectrum peripherals. game
controllers and PS/2 keyboards can be customized to simulate responses
on the stock keyboard.

\section{Keyboard}

The ZX Spectrum Next can use a classic ZX Spectrum keyboard with a
$8\times 5$ matrix, an extended ZX Spectrum Next keyboard with a
$8\times 7$ matrix (the one used in a cased Next or the custom cases
for the N-Go), or a PS/2 keyboard. In all of these cases the system
translates the physical signals to look like a classic Spectrum
keyboard with access to the additional lines of the ZX Spectrum Next
keyboard as well.

The classic Spectrum part of the matrix is accessed using port \$xxFE
where there is a single unset bit in ``xx'' to select the row being
read. Next extensions are read using Nextregs \$B0 and \$B1 to read
columns 5 and 6.

\begin{center}
  \begin{tabular}{ | c || c | c | c | c | c | c | c || c | }
    \hline
    Bit & 0 & 1 & 2 & 3 & 4 & \$B0 & \$B1 & b \\
    \hline
    \$FEFE & Cap Sh & Z & X & C & V & $\Rightarrow$ & Extend & 0 \\
    \$FDFE & A & S & D & F & G & $\Leftarrow$ & Cap Lk & 1 \\
    \$FBFE & Q & W & E & R & T & $\Downarrow$ & Graph & 2 \\
    \$F7FE & 1 & 2 & 3 & 4 & 5 & $\Uparrow$ & True Vid & 3 \\
    \$FBFE & 0 & 9 & 8 & 7 & 6 & . & Inv Vid & 4  \\
    \$EFFE & P & O & I & U & Y & , & Break & 5  \\
    \$DFFE & Enter & L & K & J & H & \textquotedbl & Edit & 6 \\
    \$BFFE & Space & Sym Sh & M & N & B & ; & Del & 7 \\
    \hline
  \end{tabular}
\end{center}

\port{7FFE}{Keyboard 8 (read only)}
\begin{itemize}
\item[] bit 0: 'B'
\item[] bit 1: 'N'
\item[] bit 2: 'M'
\item[] bit 3: Symbol Shift
\item[] bit 4: Space
\end{itemize}


\port{BFFE}{Keyboard 7 (read only)}
\begin{itemize}
\item[] bit 0 = 'H'
\item[] bit 1 = 'J'
\item[] bit 2 = 'K'
\item[] bit 3 = 'L'
\item[] bit 4 = Enter
\end{itemize}


\port{DFFE}{Keyboard 6 (read only)}
\begin{itemize}
\item[] bit 0 = 'Y'
\item[] bit 1 = 'U'
\item[] bit 2 = 'I'
\item[] bit 3 = 'O'
\item[] bit 4 = 'P'
\end{itemize}


\port{EFFE}{Keyboard 5 (read only)}
\begin{itemize}
\item[] bit 0 = ‘6’
\item[] bit 1 = ‘7’
\item[] bit 2 = ‘8’
\item[] bit 3 = ‘9’
\item[] bit 4 = ‘0’
\end{itemize}


\port{F7FE}{Keyboard 4 (read only)}
\begin{itemize}
\item[] bit 0 = ‘5’
\item[] bit 1 = ‘4’
\item[] bit 2 = ‘3’
\item[] bit 3 = ‘2’
\item[] bit 4 = ‘1’
\end{itemize}


\port{FBFE}{Keyboard 3 (read only)}
\begin{itemize}
\item[] bit 0 = ‘T’
\item[] bit 1 = ‘R’
\item[] bit 2 = ‘E’
\item[] bit 3 = ‘W’
\item[] bit 4 = ‘Q’
\end{itemize}


\port{FDFE}{Keyboard 2 (read only)}
\begin{itemize}
\item[] bit 0 = ‘G’
\item[] bit 1 = ‘F’
\item[] bit 2 = ‘D’
\item[] bit 3 = ‘S’
\item[] bit 4 = ‘A’
\end{itemize}


\port{FEFE}{Keyboard 1 (read only)}
\begin{itemize}
\item[] bit 0 = ‘V’
\item[] bit 1 = ‘C’
\item[] bit 2 = ‘X’
\item[] bit 3 = ‘Z’
\item[] bit 4 = Caps Shift
\end{itemize}


\register{R}{B0}{Extended Keys 0 (3.01.05)}
\begin{itemize}
\item bit 7 = 1 if ; pressed
\item bit 6 = 1 if \" pressed
\item bit 5 = 1 if , pressed
\item bit 4 = 1 if . pressed
\item bit 3 = 1 if UP pressed
\item bit 2 = 1 if DOWN pressed
\item bit 1 = 1 if LEFT pressed
\item bit 0 = 1 if RIGHT pressed
\end{itemize}

\register{R}{B1}{Extended Keys 1 (3.01.04)}
\begin{itemize}
\item bit 7 = 1 if DELETE pressed
\item bit 6 = 1 if EDIT pressed
\item bit 5 = 1 if BREAK pressed
\item bit 4 = 1 if INV VIDEO pressed
\item bit 3 = 1 if TRUE VIDEO pressed
\item bit 2 = 1 if GRAPH pressed
\item bit 1 = 1 if CAPS LOCK pressed
\item bit 0 = 1 if EXTEND pressed
\end{itemize}


\section{Game Controllers}

The ZX Spectrum used a number of joystick standards and the ZX
Spectrum Next can make Atari controllers or Mega Drive controllers
look like many of these standards. Atari joystick and driving (not the
similar looking paddle) controllers are supported by the
interface. The interface supports Mega Drive controllers with up to 11
buttons (start, a, b, c, x, y, z, up, down, left, right). It is
possible to map unused buttons to keys and simulate unavailable
buttons with keys.

Nextreg \$05 selects the interface mode for each of the two
joysticks. If in Kempston or Megadrive mode the joystick 1 can be read
using port \$1F and joystick 2 using port \$37. When using these ports
in Megadrive mode, some buttons will be paired A/X, B/Y, C/Z with B/Y
also corresponding to the single fire button on Atari joysticks. In
order to disambiguate presses of A/X, B/Y, and C/Z it is necessary to
also read nextreg \$B2.

\port{1F}{Kempston/Mega Drive Joystick 1/DAC A}\\
Read
\begin{itemize}
\item[] bit 7 = ''start'' button
\item[] bit 6 = A/X button
\item[] bit 5 = C/Z button
\item[] bit 4 = Fire/C/Y button
\item[] bit 3 = Up
\item[] bit 2 = Down
\item[] bit 1 = Left
\item[] bit 0 = Right
\end{itemize}
Disable with Nextreg \$05\\
Write
\begin{itemize}
\item[] bits 7-0 = DAC Value
\end{itemize}
Disable with bit 3 of Nextreg \$08

\register{W}{37}{Sprite Attribute 2}
\begin{itemize}
\item bits 7-4 = 4-bit Palette offset
\item bit 3 = Enable horizontal mirror (reverse)
\item bit 2 = Enable vertical mirror (reverse)
\item bit 1 = Enable 90$^O$  Clockwise Rotation
\end{itemize}
Normal Sprites
\begin{itemize}
\item bit 0 = X coordinate MSB
\end{itemize}
Relative Sprites
\begin{itemize}
\item bit 0 = Palette offset is relative to anchor sprite
\end{itemize}
Rotation is applied before mirroring


\register{R/W}{05}{Peripheral 1 Settings}
\begin{itemize}
\item bits 7-6 = joystick 1 mode (MSB)
\item bits 5-4 = joystick 2 mode (MSB)
\item bit 3 = joystick 1 mode (LSB)
\item bit 2 = 50/60 Hz mode (0 = 50Hz, 1 = 60Hz)
\item bit 1 = joystick 2 mode (LSB)
\item bit 0 = Enable Scandoubler
\end{itemize}
Joystick modes
\begin{itemize}
\item 000 = Sinclair 2 (67890)
\item 001 = Kempston 2 (port \$37)
\item 010 = Kempston 1 (port \$1F)
\item 011 = Megadrive 1 (port \$1F)
\item 100 = Cursor
\item 101 = Megadrive 2 (port \$37)
\item 110 = Sinclair 1 (12345)
\item 111 = I/O Mode (3.01.04)
Both joysticks are places in I/O Mode if either is set to I/O
Mode. The underlying joystick type is not changed and reads of this
register will continue to return the last joystick type. Ehether the
joystick is in io mode or not is invisible but this state can be
cleared either through reset or by re-writing the gegister with
joystick type not equal to 111. Recovery time for a normal joystick
read after leaving I/O Mode is at most 64 scan lines.
\end{itemize}


\register{R}{B2}{Extended MD Pad Buttons (3.01.10)}
\begin{itemize}
\item bit 7 = 1 if Right Pad X Pressed
\item bit 6 = 1 if Right Pad Z Pressed
\item bit 5 = 1 if Right Pad Y Pressed
\item bit 4 = 1 if Right Pad START Pressed
\item bit 3 = 1 if Left Pad X Pressed
\item bit 2 = 1 if Left Pad Z Pressed
\item bit 1 = 1 if Left Pad Y Pressed
\item bit 0 = 1 if Left Pad START Pressed
\end{itemize}


\section{Mouse}

A mouse attached to the PS/2 port looks like a kempston mouse to the
system. Port \$FADF is used to read the state of the mouse wheel and
buttons while ports \$FBDF and \$FBDF are used to read the X and Y
positions.

\port{FADF}{Kempston Mouse Buttons}
\begin{itemize}
\item[] bits 7-4 = Wheel delta since last read
\item[] bit 3 = fourth button
\item[] bit 2 = middle button
\item[] bit 1 = left button
\item[] bit 0 = right button
\end{itemize}


\port{FBDF}{Kempston Mouse X}
\begin{itemize}
\item[] bits 7-0 = X coordinate of mouse
\end{itemize}


\port{FFDF}{Kempston Mouse Y}
\begin{itemize}
\item[] bits 7-0 = Y coordinate of mouse (0-192)
\end{itemize}



\section{Keymapping}

Both a PS/2 keyboard and the contoller buttons can be given custom
keyboard mappings. These mappings are accomplished using nextregs \$28
(keymap address MSB), \$29 (keymap address LSB), \$2A (keymap data
MSB) and \$2B (keymap data LSB).

\register{R/W}{28}{Stored Palette Value and PS/2 Keymap Address MSB}\\
Read
\begin{itemize}
\item bits 7-0 = Stored palette value (see NextREG \$44)
\end{itemize}
Write  
\begin{itemize}
\item bits 7-1 = Reserved, must be 0
\item bit 0 = PS/2 Keymap Address MSB
\end{itemize}


\register{W}{29}{PS/2 Keymap Address LSB}
\begin{itemize}
\item bits 7-0 = PS/2 Keymap Address LSB
\end{itemize}


\register{W}{2A}{PS/2 Keymap Data MSB}
\begin{itemize}
\item bits 7-1 = Reserved, must be 0
\item bit 0 = PS/2 Keymap Data MSB
\end{itemize}


\register{W}{2B}{PS/2 Keymap Data LSB}
\begin{itemize}
\item bits 7-0 = PS/2 Keymap Data LSB
\end{itemize}
(writing this register auto-increments the address)



\subsection{Keyboard}

To remap PS/2 keyboard keys, the starting address for the keymap data
is written to nextregs \$28 and \$29 keyboard data address followed by
writing the data to nextregs \$2A and \$2B with the write to \$2B
autoincrementing the address for the data.

\subsection{Joystick}

To map joystick buttons an \$80 is written to nextreg followed by a
\$00 (left) or \$10 (right) to nextreg \$29 to select which joystick
is being mapped to the keyboard 11 writes to nextgreg \$2B to indicate
the key(s) for each button. Each byte to addresses the row and column
of the ZX Next keyboard with bits 5-3 being the row and bits 2-0 being
the column (column 7, 111 is no action).
