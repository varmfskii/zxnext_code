\chapter{Basic Input/Output}

The basic I/O (human interface) system of the ZX Spectrum Next
supports keyboard, mouse and game controllers. These all extend the
functionality found on common ZX Spectrum peripherals. game
controllers and PS/2 keyboards can be customized to simulate responses
on the stock keyboard.

\section{Keyboard}

The ZX Spectrum Next can use a classic ZX Spectrum keyboard with a
$8\times 5$ matrix, an extended ZX Spectrum Next keyboard with a
$8\times 7$ matrix (the one used in a cased Next or the custom cases
for the N-Go), or a PS/2 keyboard. In all of these cases the system
translates the physical signals to look like a classic Spectrum
keyboard with access to the additional lines of the ZX Spectrum Next
keyboard as well.

The classic Spectrum part of the matrix is accessed using port \$xxFE
where there is a single unset bit in ``xx'' to select the row being
read. Next extensions are read using Nextregs \$B0 and \$B1 to read
columns 5 and 6.

\begin{center}
  \begin{tabular}{ | c || c | c | c | c | c | c | c || c | }
    \hline
    Bit & 0 & 1 & 2 & 3 & 4 & \$B0 & \$B1 & b \\
    \hline
    \$FEFE & Cap Sh & Z & X & C & V & $\Rightarrow$ & Extend & 0 \\
    \$FDFE & A & S & D & F & G & $\Leftarrow$ & Cap Lk & 1 \\
    \$FBFE & Q & W & E & R & T & $\Downarrow$ & Graph & 2 \\
    \$F7FE & 1 & 2 & 3 & 4 & 5 & $\Uparrow$ & True Vid & 3 \\
    \$FBFE & 0 & 9 & 8 & 7 & 6 & . & Inv Vid & 4  \\
    \$EFFE & P & O & I & U & Y & , & Break & 5  \\
    \$DFFE & Enter & L & K & J & H & \textquotedbl & Edit & 6 \\
    \$BFFE & Space & Sym Sh & M & N & B & ; & Del & 7 \\
    \hline
  \end{tabular}
\end{center}

\input{ports/7FFE.tex}
\input{ports/BFFE.tex}
\input{ports/DFFE.tex}
\input{ports/EFFE.tex}
\input{ports/F7FE.tex}
\input{ports/FBFE.tex}
\input{ports/FDFE.tex}
\input{ports/FEFE.tex}
\input{registers/tbblue/B0.tex}
\input{registers/tbblue/B1.tex}

\section{Game Controllers}

The ZX Spectrum used a number of joystick standards and the ZX
Spectrum Next can make Atari controllers or Mega Drive controllers
look like many of these standards. Atari joystick and driving (not the
similar looking paddle) controllers are supported by the
interface. The interface supports Mega Drive controllers with up to 11
buttons (start, a, b, c, x, y, z, up, down, left, right). It is
possible to map unused buttons to keys and simulate unavailable
buttons with keys.

Nextreg \$05 selects the interface mode for each of the two
joysticks. If in Kempston or Megadrive mode the joystick 1 can be read
using port \$1F and joystick 2 using port \$37. When using these ports
in Megadrive mode, some buttons will be paired A/X, B/Y, C/Z with B/Y
also corresponding to the single fire button on Atari joysticks. In
order to disambiguate presses of A/X, B/Y, and C/Z it is necessary to
also read nextreg \$B2.

\port{1F}{Kempston/Mega Drive Joystick 1/DAC A}\\
Read
\begin{itemize}
\item[] bit 7 = ''start'' button
\item[] bit 6 = A button
\item[] bit 5 = Fire 2/C button
\item[] bit 4 = Fire 1/B button
\item[] bit 3 = Up
\item[] bit 2 = Down
\item[] bit 1 = Left
\item[] bit 0 = Right
\end{itemize}
Disable with Nextreg \$05\\
Write
\begin{itemize}
\item[] bits 7-0 = DAC Value
\end{itemize}
The XYZ buttons on md pads can be read through nextreg 0xB2.\\
The joysticks can also be placed in i/o mode see nextreg 0x0B.\\
All eleven md pad buttons can be assigned to the keyboard see nextreg 0x05.

\port{37}{Kempston/Mega Drive Joystick 2}\\
Read
\begin{itemize}
\item[] bit 7 = ''start'' button
\item[] bit 6 = A button
\item[] bit 5 = Fire 2/C button
\item[] bit 4 = Fire 1/B button
\item[] bit 3 = Up
\item[] bit 2 = Down
\item[] bit 1 = Left
\item[] bit 0 = Right
\end{itemize}
The XYZ buttons on md pads can be read through nextreg \$B2.\\
The joysticks can also be placed in i/o mode see nextreg \$0B.\\
All eleven md pad buttons can be assigned to the keyboard see nextreg \$05.



\register{R/W}{05}{Peripheral 1 Settings}
\begin{itemize}
\item bits 7-6 = joystick 1 mode (MSB)
\item bits 5-4 = joystick 2 mode (MSB)
\item bit 3 = joystick 1 mode (LSB)
\item bit 2 = 50/60 Hz mode (0 = 50Hz, 1 = 60Hz)
\item bit 1 = joystick 2 mode (LSB)
\item bit 0 = Enable Scandoubler
\end{itemize}
Joystick modes
\begin{itemize}
\item 000 = Sinclair 2 (67890)
\item 001 = Kempston 2 (port \$37)
\item 010 = Kempston 1 (port \$1F)
\item 011 = Megadrive 1 (port \$1F)
\item 100 = Cursor
\item 101 = Megadrive 2 (port \$37)
\item 110 = Sinclair 1 (12345)
\item 111 = User Defined Keys Joystick
\end{itemize}
\begin{itemize}
\item[*] Joysticks can be placed in i/o mode via nextreg 0x0B.
\item[*] Programming the user defined keys joystick is done through the ps2
keymap interface on nextreg 0x28, 0x29 and 0x2B:
\end{itemize}
\begin{enumerate}
\item Write 128 to nextreg 0x28
\item Write 0 (left joystick) or 16 (right joystick) to nextreg 0x29
\item Write eleven bytes to nextreg 0x2B. The bytes correspond to the eleven
buttons on an md pad (X=11 Z Y START A C B U D L R=1)
\item Each byte written identifies a key in the 8x7 membrane; bits 5:3 select
the row and bits 2:0 select the column with 111 meaning no action.\\ *
In all joystick modes, excess buttons on an md pad not read via ports
will generate key input if so programmed.
\end{enumerate}


\input{registers/tbblue/B2.tex}

\section{Mouse}

A mouse attached to the PS/2 port looks like a kempston mouse to the
system. Port \$FADF is used to read the state of the mouse wheel and
buttons while ports \$FBDF and \$FBDF are used to read the X and Y
positions.

\input{ports/FADF.tex}
\input{ports/FBDF.tex}
\input{ports/FFDF.tex}

\section{Keymapping}

Both a PS/2 keyboard and the contoller buttons can be given custom
keyboard mappings. These mappings are accomplished using nextregs \$28
(keymap address MSB), \$29 (keymap address LSB), \$2A (keymap data
MSB) and \$2B (keymap data LSB).

\input{registers/tbblue/28.tex}
\input{registers/tbblue/29.tex}
\input{registers/tbblue/2A.tex}
\input{registers/tbblue/2B.tex}

\subsection{Keyboard}

To remap PS/2 keyboard keys, the starting address for the keymap data
is written to nextregs \$28 and \$29 keyboard data address followed by
writing the data to nextregs \$2A and \$2B with the write to \$2B
autoincrementing the address for the data.

\subsection{Joystick}

To map joystick buttons an \$80 is written to nextreg followed by a
\$00 (left) or \$10 (right) to nextreg \$29 to select which joystick
is being mapped to the keyboard 11 writes to nextgreg \$2B to indicate
the key(s) for each button. Each byte to addresses the row and column
of the ZX Next keyboard with bits 5-3 being the row and bits 2-0 being
the column (column 7, 111 is no action).
