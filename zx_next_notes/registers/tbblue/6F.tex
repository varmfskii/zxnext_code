\register{R/W}{6F}{Layer 3 Tile Definitions Base Address}
\begin{itemize}
\item bits 7-6 = Read back as zero, write values ignored
\item bits 5-0 = MSB of address of the tilemap in Bank 5 (\$0C on
  reset)
\end{itemize}
Soft Reset default \$0C - This is because the address is \$4C00 so the
MSB is \$4C. But the stored value is only the lower 6 bits so it's an
offset into the 16k Bank 5. To calculate therefore subtract \$40
leaving you with \$0C.

The value written is an offset into the 16k Bank 5 allowing the
tilemap to be placed at any multiple of 256 bytes.  Writing a physical
MSB address in \$40 -- \$7F or \$C0 -- \$FF range is permitted.

The value read back should be treated as having a fully significant
8-bit value.

