\section{ZX Spectrum Next Registers}
February 25, 2019  Phoebus Dokos

TBBlue stores configuration state in a field of registers. These
registers are accessible via two I/O ports or via the special nextreg
instructions.

Port \$243B (9275) is used to set the register number, listed below.

Port \$253B (9531) is used to access the register value.

Some registers are accessible only during the initialization process.

\register{R}{00}{Machine ID}
\begin{itemize}
\item 00000001 = DE1A
\item 00000010 = DE2A
\item 00000101 = FBLABS
\item 00000110 = VTRUCCO
\item 00000111 = WXEDA
\item 00001000 = EMULATORS*
\item 00001010 = ZX Spectrum Next*
\item 00001011 = Multicore
\item 11111010 = ZX Spectrum Next Anti-brick*
\end{itemize}
* Relevant for ZX Next machines \& software

\register{R}{01}{Core Version}
\begin{itemize}
\item bits 7-4 = Major version number
\item bits 3-0 = Minor version number
\item[] See register \$0E for sub minor version number
\end{itemize}

\register{R/W}{02}{Reset}\\
Read
\begin{itemize}
\item bit 7 = Expansion bus \textoverline{RESET} Asserted
\item bits 6-2 = Reserved
\item bit 1 = Last reset was Hard reset
\item bit 0 = Last reset was Soft reset
\end{itemize}
Write
\begin{itemize}
\item bit 7 = Generate/Release Expansion bus \textoverline{RESET}
\item bits 6-2 = Reserved, must be 0
\item bit 1 = generate Hard reset
\item bit 0 = generate Soft reset
\end{itemize}

\register{R/W}{03}{Machine Type}\\
A write to this register disables the boot rom in config mode\\
bits 2-0 select machine type when in config mode
\begin{itemize}
\item bit 7 = (W) Display Timing change enable (allow changes to
  bits 6-4) (0 on hard reset)
\item bits 6-4 = Display Timing
\item bit 3 = Display Timing user lock control
  \item[] Read
  \begin{itemize}
  \item 0 = No user lock on display timing
  \item 1 = User lock on display timing
  \end{itemize}
  \item[] Write
  \begin{itemize}
  \item 1 = Apply user lock on display timing (0 on hard reset)
  \end{itemize}
\item bits 2-1 = Machine Type
\item[] Machine Types/Display Timings
  \begin{itemize}
  \item 000 or 001 = ZX 48K
  \item 010 = ZX 128K/+2 (Grey)
  \item 011 = ZX +2A-B/+3e/Next Native
  \item 100 = Pentagon 128K
  \end{itemize}
\end{itemize}

\register{W}{04}{Configuration Mapping}
\begin{itemize}
\item bits 7-5 = Reserved, must be 0
\item bits 4-0 = 16k SRAM bank mapping* (\$00 on hard reset)
\item[] *Maps a 16k SRAM bank over the bottom 16k. Applies only in
  config mode when the bootrom is disabled
\end{itemize}

\register{R/W}{05}{Peripheral 1 Settings}
\begin{itemize}
\item bits 7-6 = joystick 1 mode (MSB)
\item bits 5-4 = joystick 2 mode (MSB)
\item bit 3 = joystick 1 mode (LSB)
\item bit 2 = 50/60 Hz mode (0 = 50Hz, 1 = 60Hz)
\item bit 1 = joystick 2 mode (LSB)
\item bit 0 = Enable Scandoubler
\end{itemize}
Joystick modes
\begin{itemize}
\item 000 = Sinclair 2 (67890)
\item 001 = Kempston 2 (port \$37)
\item 010 = Kempston 1 (port \$1F)
\item 011 = Megadrive 1 (port \$1F)
\item 100 = Cursor
\item 101 = Megadrive 2 (port \$37)
\item 110 = Simclair 1 (12345)
\end{itemize}

\register{R/W}{06}{Peripheral 2 Settings}
\begin{itemize}
\item bit 7 = F8 CPU Speed Hotkey Enable (1 on reset)
\item bit 6 = DMA mode (0 = zxnDMA, 1 = Z80DMA) (0 on hard reset)
\item bit 5 = F3 50Hz/60Hz Hotkey Enable (1 on reset)
\item bit 4 = divMMC Automap/NMI Enable (0 on hard reset)
\item bit 3 = NMI Button Enable (0 on hard reset)
\item bit 2 = PS/2 Mode (0 = keyboard, 1 = mouse)
\item bits 1-0 = PSG Mode (00 = YM, 01 = AY, 11 = hold all PSGs in Reset)
\end{itemize}

\register{R/W}{07}{Turbo mode}\\
Read
\begin{itemize}
\item bits 7-6 = Reserved
\item bits 5-4 = Current Actual CPU Speed
\item bits 3-2 = Reserved
\item bits 1-0 = Current Selected CPU Speed (00 on reset)
\end{itemize}
Write
\begin{itemize}
\item bits 7-2 = Reserved, must be 0
\item bits 1-0 = Select CPU Speed
\end{itemize}
CPU Speeds
\begin{itemize}
\item 00 = 3.5MHz
\item 01 = 7MHz
\item 10 = 14MHz
\item 11 = 28MHz
\end{itemize}

\register{R/W}{08}{Peripheral 3 Settings}
\begin{itemize}
\item bit 7 = 128K Banking Unlock (inverse of port \$7FFD, bit 5) (0
  on reset)
\item bit 6 = Disable RAM and Port Contention (0 on reset)
\item bit 5 = PSG Stereo Mode Control (0 = ABC, 1 = ACB) (0 on hard
  reset)
\item bit 4 = Enable internal speaker (1 on hard reset)
\item bit 3 = Enable DACs (0 on hard reset)
\item bit 2 = Enable read of port \$FF (Timex) (0 on hard reset)
\item bit 1 = Enable Multiple PSGs (0 on hard reset)
\item bit 0 = Enable Issue 2 Keyboard
\end{itemize}

\register{R/W}{09}{Peripheral 4 setting:}
\begin{itemize}
\item bit 7 = PSG 2 Mono Enable (0 on hard reset)
\item bit 6 = PSG 1 Mono Enable (0 on hard reset)
\item bit 5 = PSG 0 Mono Enable (0 on hard reset)
\item bit 4 = Sprite ID lockstep enable (1 = Nextreg \$34 and IO Port
  \$303B are in lockstep, 0 on reset)
\item bit 3 = divMMC mapRAM bit Control (reset bit 7 of port \$E3)
\item bit 2 = HDMI audio mute (0 on hard reset)
\item bits 1-0 = scanlines
  \begin{itemize}
  \item 00 = scanlines off
  \item 01 = scanlines 75\%
  \item 10 = scanlines 50\%
  \item 11 = scanlines 25\%
  \end{itemize}
\item[] In Sprite lockstep, NextREG \$34 and Port \$303B are in
  lockstep
\end{itemize}

\register{R}{0E}{Core Version (sub minor number)}
\begin{itemize}
\item bits 7-0 = Core sub minor version number
\item[] (see register \$01 for the major and minor version number)
\end{itemize}

\register{R/W}{10}{Core Boot}\\
Read
\begin{itemize}
\item bits 7-2 = Reserved
\item bit 1 = Drive button pressed
\item bit 0 = NMI button pressed
\end{itemize}
Write
\begin{itemize}
\item bit 7 = Reboot FPGA using selected core (0 on reset)
\item bits 6-5 = Reserved, must be 0
\item bits 4-0 = Core ID
\item[] Core ID with bits 4-0 can only be set in configuration mode
\end{itemize}

\register{R/W}{11}{Video Timing (writable in config mode only)}
\begin{itemize}
\item bits 7-3 = Reserved, must be 0
\item bits 2-0 = Mode (VGA = 0..6, HDMI = 7)
  \begin{itemize}
  \item 000 = Base VGA timing, clk28 = 28000000
  \item 001 = VGA setting 1, clk28 = 28571429
  \item 010 = VGA setting 2, clk28 = 29464286
  \item 011 = VGA setting 3, clk28 = 30000000
  \item 100 = VGA setting 4, clk28 = 31000000
  \item 101 = VGA setting 5, clk28 = 32000000
  \item 110 = VGA setting 6, clk28 = 33000000
  \item 111 = HDMI, clk28 = 27000000
  \end{itemize}
\item[] 50/60Hz selection depends on bit 2 of register \$05
\item[] Only writable in config mode
\end{itemize}

\register{R/W}{12}{Layer 2 Active RAM bank}
\begin{itemize}
\item bits 7-6 = Reserved, must be 0
\item bits 5-0 = RAM bank (point to bank 8 after a Reset, NextZXOS
  modifies to 9)
\end{itemize}

\register{R/W}{13}{Layer 2 Shadow RAM bank}
\begin{itemize}
\item bits 7-6 = Reserved, must be 0
\item bits 5-0 = RAM bank (point to bank 11 after a Reset, NextZXOS
  modifies to 12)
\end{itemize}

\register{R/W}{14}{Global transparency color}
\begin{itemize}
\item bits 7-0 = Transparency color value (\$E3 after a reset)
\end{itemize}
(Note: this value is 8-bit, so the transparency is compared against
only by the MSB bits of the final 9-bit colour)\\
(Note2: this only affects Layer 2, ULA and LoRes. Sprites use register
\$4B for transparency and tilemap uses nextreg \$4C)

\register{R/W}{15}{Sprite and Layer System Setup}
\begin{itemize}
\item bit 7 = LoRes mode (0 on reset)
\item bit 6 = Sprite priority (1 = sprite 0 on top, 0 = sprite 127 on
  top) (0 on reset)
\item bit 5 = Enable sprite clipping in over border mode (0 on reset)
\item bits 4-2 = set layers priorities (000 on reset)
  \begin{itemize}
  \item 000 - S L U
  \item 001 - L S U
  \item 010 - S U L
  \item 011 - L U S
  \item 100 - U S L
  \item 101 - U L S
  \item 110 - S(U+L) ULA and Layer 2 combined, colours clamped to 7
  \item 111 - S(U+L-5) ULA and Layer 2 combined, colours clamped to [0,7]
  \end{itemize}
\item bit 1 = Enable Sprites Over border (0 on reset)
\item bit 0 = Enable Sprites (0 on reset)
\end{itemize}

\register{R/W}{16}{Layer 2 Horizontal Scroll Control}
\begin{itemize}
\item bits 7-0 = X Offset (0-255)(0 on reset)
\end{itemize}

\register{R/W}{17}{Layer 2 Vertical Scroll Control}
\begin{itemize}
\item bits 7-0 = Y Offset (0-191)(0 on reset)
\end{itemize}

\register{R/W}{18}{Layer 2 Clip Window Definition}
\begin{itemize}
\item bits 7-0 = Coords of the clip window
  \begin{itemize}
  \item[] 1st write - X1 position
  \item[] 2nd write - X2 position
  \item[] 3rd write - Y1 position
  \item[] 4rd write - Y2 position
  \end{itemize}
\end{itemize}
Reads do not advance the clip position\\
The values are 0,255,0,191 after a Reset

\register{R/W}{19}{Sprite Clip Window Definition}
\begin{itemize}
\item bits 7-0 = Cood. of the clip window
  \begin{itemize}
  \item[] 1st write - X1 position
  \item[] 2nd write - X2 position
  \item[] 3rd write - Y1 position
  \item[] 4rd write - Y2 position
  \end{itemize}
\end{itemize}
The values are 0,255,0,191 after a Reset\\
Reads do not advance the clip position

When the clip window is enabled for sprites in "over border" mode, the
X coords are internally doubled and the clip window origin is moved to
the sprite origin inside the border.

\register{R/W}{1A}{Layer 0 (ULA/LoRes) Clip Window Definition}
\begin{itemize}
\item bits 7-0 = Coord. of the clip window
  \begin{itemize}
  \item[] 1st write = X1 position
  \item[] 2nd write = X2 position
  \item[] 3rd write = Y1 position
  \item[] 4rd write = Y2 position
  \end{itemize}
\end{itemize}
The values are 0,255,0,191 after a Reset\\
Reads do not advance the clip position

\register{R/W}{1B}{Layer 3 (Tilemap) Clip Window Definition}
\begin{itemize}
\item bits 7-0 = Coord. of the clip window
  \begin{itemize}
  \item[] 1st write = X1 position
  \item[] 2nd write = X2 position
  \item[] 3rd write = Y1 position
  \item[] 4rd write = Y2 position
  \end{itemize}
\end{itemize}
The values are 0,159,0,255 after a Reset\\
Reads do not advance the clip position\\
The X coords are internally doubled.

\register{R/W}{1C}{Clip Window Control}\\
Read
\begin{itemize}
\item bits 7-6 = Layer 3 Clip Index
\item bits 5-4 = Layer 0/1 Clip Index
\item bits 3-2 = Sprite clip index
\item bits 1-0 = Layer 2 Clip Index
\end{itemize}
Write
\begin{itemize}
\item bits 7-4 = Reserved, must be 0
\item bit 3 - reset Layer 3 clip index
\item bit 2 - reset Layer 0/1 clip index
\item bit 1 - reset sprite clip index.
\item bit 0 - reset Layer 2 clip index.
\end{itemize}

\register{R}{1E}{Active video line (MSB)}
\begin{itemize}
\item bits 7-1 = Reserved
\item bit 0 = Active line MSB
\end{itemize}

\register{R}{1F}{Active video line (LSB)}
\begin{itemize}
\item bits 7-0 = Active line LSB (0-255)
\end{itemize}

\register{R/W}{22}{Line Interrupt control}
\begin{itemize}
\item bit 7 = (R) ULA asserting interrupt
\item bit 7 = (W) Reserved, must be 0
\item bits 6-3 = Reserved, must be 0
\item bit 2 = Disable ULA Interrupt (0 on reset)
\item bit 1 = Enable Line Interrupt (0 on reset)
\item bit 0 = MSB of Line Interrupt line value (0 on reset)
\end{itemize}

\register{R/W}{23}{Line Interrupt value LSB}
\begin{itemize}
\item bits 7-0 = Line Interrupt line value LSB (0-255)(0 on reset)
\end{itemize}

\register{R/W}{26}{ULA Horizontal Scroll Control}
\begin{itemize}
\item bits 7-0 = ULA X Offset (0-255) (0 on reset)
\end{itemize}

\register{R/W}{27}{ULA Vertical Scroll Control}
\begin{itemize}
\item bits 7-0 = ULA Y Offset (0-191) (0 on reset)
\end{itemize}

\register{R/W}{28}{Stored Palette Value and PS/2 Keymap Address MSB}\\
Read
\begin{itemize}
\item bits 7-0 = Stored palette value (see NextREG \$44)
\end{itemize}
Write  
\begin{itemize}
\item bits 7-1 = Reserved, must be 0
\item bit 0 = PS/2 Keymap Address MSB
\end{itemize}

\register{W}{29}{PS/2 Keymap Address LSB}
\begin{itemize}
\item bits 7-0 = PS/2 Keymap Address LSB
\end{itemize}

\register{W}{2A}{PS/2 Keymap Data MSB}
\begin{itemize}
\item bits 7-1 = Reserved, must be 0
\item bit 0 = PS/2 Keymap Data MSB
\end{itemize}

\register{W}{2B}{PS/2 Keymap Data LSB}
\begin{itemize}
\item bits 7-0 = PS/2 Keymap Data LSB
\end{itemize}
(writing this register auto-increments the address)

\register{R/W}{2C}{DAC B Mirror (Left)/\iis Left Sample MSB}\\
Read
\begin{itemize}
\item bits 7-0 = \iis Left Sample MSB
\end{itemize}
Write
\begin{itemize}
\item bits 7-0 = 8-bit sample left DAC (\$80 on reset)
\end{itemize}

\register{R/W}{2D}{DAC A+D Mirror (mono/\iis Sample LSB}\\
Read
\begin{itemize}
\item bits 7-0 = \iis Last Sample LSB
\end{itemize}
Write
\begin{itemize}
\item bits 7-0 = 8-bit sample DACs A + D (\$80 on reset)
\end{itemize}

\register{R/W}{2E}{DAC C Mirror (Right/\iis Risht Sample MSB}\\
Read
\begin{itemize}
\item bits 7-0 = \iis Right Sameple MSB
\end{itemize}
Write
\begin{itemize}
\item bits 7-0 = 8-bit sample Right DACs C (\$80 on reset)
\end{itemize}

\register{R/W}{2F}{Layer 3 (Tilemap) Horizontal Scroll Control MSB}
\begin{itemize}
\item bits 7-2 = Reserved, must be 0
\item bits 1-0 = X Offset MSB (\$00 on reset)
\end{itemize}
Meaningful Range is 0-319 in 40 char mode, 0-639 in 80 char mode

\register{R/W}{30}{Layer 3 (Tilemap) Horizontal Scroll Control LSB}
\begin{itemize}
\item bits 7-0 = X Offset LSB (\$00 on reset)
\end{itemize}
Meaningful range is 0-319 in 40 char mode, 0-639 in 80 char mode

\register{R/W}{31}{Layer 3 (Tilemap) Vertical Scroll Control}
\begin{itemize}
\item bits 7-0 = Y Offset (0-255) )\$00 on reset)
\end{itemize}

\register{R/W}{32}{Layer 1,0 (LoRes) Horizontal Scroll Control)}
\begin{itemize}
\item bits 7-0 = X Offset (0-255) (\$00 on reset)
\end{itemize}
Layer 1,0 (LoRes) scrolls in "half-pixels" at the same resolution and
smoothness as Layer 2.

\register{R/W}{33}{Layer 1,0 (LoRes) Vertical Scroll Control)}
\begin{itemize}
\item bits 7-0 = Y Offset (0-191) (\$00 on reset)
\end{itemize}
Layer 1,0 (LoRes) scrolls in "half-pixels" at the same resolution and
smoothness as Layer 2.

\register{R/W}{34}{Sprite Number}\\
Lockstep (NextReg \$09 bit 4 set)
\begin{itemize}
\item bit 7 = Pattern address offset (Add 128 to pattern address)
\item bits 6-0 = Sprite number 0-127, Pattern number 0-63
\item[] effectively performs an out to port \$303B
\end{itemize}
No Lockstep (NextReg \$09 bit 4 clear)
\begin{itemize}
\item bit 7 = Reserved, must be 0
\item bits 6-0 = Sprite number 0-127
\end{itemize}
This register selects which sprite has its attributes connected to the
sprite attribute registers

\register{W}{35}{Sprite Attribute 0}
\begin{itemize}
\item bits 7-0 = Sprite X coordinate LSB (MSB in NextReg \$37)
\end{itemize}

\register{W}{36}{Sprite Attribute 1}
\begin{itemize}
\item bits 7-0 = Sprite Y coordinate LSB (MSB in NextReg \$39)
\end{itemize}

\register{W}{37}{Sprite Attribute 2}
\begin{itemize}
\item bits 7-4 = 4-bit Palette offset
\item bit 3 = Enable horizontal mirror (reverse)
\item bit 2 = Enable vertical mirror (reverse)
\item bit 1 = Enable 90$^O$  Clockwise Rotation
\end{itemize}
Normal Sprites
\begin{itemize}
\item bit 0 = X coordinate MSB
\end{itemize}
Relative Sprites
\begin{itemize}
\item bit 0 = Palette offset is relative to anchor sprite
\end{itemize}
Rotation is applied before mirroring

\register{W}{38}{Sprite Attribute 3}
\begin{itemize}
\item bit 7 = Enable Visiblity
\item bit 6 = Enable Attribute 4 (0 = Attribute 4 effectively \$00)
\item bits 5-0 = Sprite Pattern Number
\end{itemize}

\register{W}{39}{Sprite Attribute 4}\\
Normal Sprites
\begin{itemize}
\item bit 7 = 4-bit pattern switch (0 = 8-bit sprite, 1 = 4-bit sprite)
\item bit 6 = Pattern number bit-7 for 4-bit, 0 for 8-bit
\item bit 5 = Type of attached relative sprites (0 = Composite, 1 =
  Unified)
\item bits 4-3 = X scaling (00 = 1x, 01 = 2x, 10 = 4x, 11 = 8x)
\item bits 2-1 = Y scaling (00 = 1x, 01 = 2x, 10 = 4x, 11 = 8x)
\item bit 0 = MSB of Y coordinate
\end{itemize}
Relative, Composite Sprites
\begin{itemize}
\item bit 7-6 = 01
\item bit 5 = Pattern number bit-7 for 4-bit, 0 for 8-bit
\item bits 4-3 = X scaling (00 = 1x, 01 = 2x, 10 = 4x, 11 = 8x)
\item bits 2-1 = Y scaling (00 = 1x, 01 = 2x, 10 = 4x, 11 = 8x)
\item bit 0 = Pattern number is relative to anchor
\end{itemize}
Relative, Unified Sprites
\begin{itemize}
\item bit 7-6 = 01
\item bit 5 = Pattern number bit-7 for 4-bit, 0 for 8-bit
\item bits 4-1 = 0000
\item bit 0 = Pattern number is relative to anchor
\end{itemize}

\register{R/W}{40}{Palette Index Select}
\begin{itemize}
\item bits 7-0 = Palette Index Number
\end{itemize}
Selects the palette index to change the associated colour

For ULA only, INKs are mapped to indices 0 through 7, BRIGHT INKs to
indices 8 through 15, PAPERs to indices 16 through 23 and BRIGHT
PAPERs to indices 24 through 31.  In EnhancedULA mode, INKs come from
a subset of indices from 0 through 127 and PAPERs from a subset of
indices from 128 through 255.

The number of active indices depends on the number of attribute bits
assigned to INK and PAPER out of the attribute byte.

In ULAplus mode, the last 64 entries (indices 192 to 255) hold the
ULAplus palette.  The ULA always takes border colour from PAPER for
standard ULA and Enhanced ULA

\register{R/W}{41}{8-bit Palette Data}
\begin{itemize}
\item bits 7-0 = Colour Entry in RRRGGGBB format
\end{itemize}
The lower blue bit of the 9-bit internal colour will be the logical or
of bits 0 and 1 of the 8-bit entry. After each write, the palette
index auto-increments if aut-increment has been enabled (NextReg \$43
bit 7), Reads do not auto-increment.

\register{R/W}{42}{ULANext Attribute Byte Format}
\begin{itemize}
\item bits 7-0 = Attribute byte's INK representation mask (7 on reset)
\end{itemize}
The mask can only indicate a solid sequence of bits on the right side
of the attribute byte (1, 3, 7, 15, 31, 63, 127 or 255).

INKs are mapped to base index 0 in the palette and PAPERs and border
are mapped to base index 128 in the palette.

The 255 value enables the full ink colour mode making all the palette
entries INK. PAPER and border both take on the fallback colour
(nextreg \$4A) in this mode.

\register{R/W}{43}{Palette Control}
\begin{itemize}
\item bit 7 = Disable palette write auto-increment.
\item bits 6-4 = Select palette for reading or writing:
  \begin{itemize}
  \item 000 = ULA first palette
  \item 100 = ULA second palette
  \item 001 = Layer 2 first palette
  \item 101 = Layer 2 second palette
  \item 010 = Sprite first palette
  \item 110 = Sprite second palette
  \item 011 = Layer 3 first palette
  \item 111 = Layer 3 second palette
  \end{itemize}
\item bit 3 = Select Sprite palette (0 = first palette, 1 = second
  palette)
\item bit 2 = Select Layer 2 palette (0 = first palette, 1 = second
  palette)
\item bit 1 = Select ULA palette (0 = first palette, 1 = second
  palette)
\item bit 0 = Enable EnhancedULA mode if 1. (0 after a reset)
\end{itemize}

\register{R/W}{44}{9-bit Palette Data}\\
Non Level 2
\begin{itemize}
\item[] 1st write
\item bits 7-0 = MSB (RRRGGGBB)
\item[] 2nd write
\item bits 7-1 = Reserved, must be 0
\item bit 0 = LSB (B)
\end{itemize}
Level 2
\begin{itemize}
\item[] 1st write
\item bits 7-0 = MSB (RRRGGGBB)
\item[] 2nd write
\item bit 7 = Priority
\item bits 6-1 = Reserved, must be 0
\item bit 0 = LSB (B)
\end{itemize}
9-bit Palette Data is entered in two consecutive writes; the second
write autoincrements the palette index if auto-increment is enabled in
NextREG \$43 bit 7

If writing an L2 palette, the second write's D7 holds the L2 priority
bit which if set (1) brings the colour defined at that index on top of
all other layers. If you also need the same colour in regular priority
(for example: for enviromental masking) you will have to set it up
again, this time with no priority.

Reads return the second byte and do not autoincrement.

\register{R/W}{4A}{Fallback Colour Value}
\begin{itemize}
\item bits 7-0 = 8-bit colour if all layers are transparent (\$E3 on
  reset)
\end{itemize}
(black on reset = 0)

\register{R/W}{4B}{Sprite Transparency Index}
\begin{itemize}
\item bits 7-0 = Index value (\$E3 if reset)
\end{itemize}
For 4-bit sprites only the bottom 4-bits are relevant.

\register{R/W}{4C}{Level 3 Transparency Index}
\begin{itemize}
\item bits 7-4 = Reserved, must be 0
\item bits 3-0 = Index value (\$0F on reset)
\end{itemize}

\register{R/W}{50}{MMU Slot 0 Control}
\begin{itemize}
\item bits 7-0 = 8k RAM page at position \$0000 to \$1FFF (\$ff on
  reset)
\end{itemize}
Pages can be from 0 to 223 on a fully expanded Next.\\
A 255 value causes the ROM to become visible.

\register{R/W}{51}{MMU Slot 1 Control}
\begin{itemize}
\item bits 7-0 = 8k RAM page at position \$2000 to \$3FFF (\$ff on
  reset)
\end{itemize}
Pages can be from 0 to 223 on a fully expanded Next.\\
A 255 value causes the ROM to become visible.

\register{R/W}{52}{MMU Slot 2 Control}
\begin{itemize}
\item bits 7-0 = 8k RAM page at position \$4000 to \$5FFF (\$0A on
  reset)
\end{itemize}
Pages can be from 0 to 223 on a fully expanded Next.

\register{R/W}{53}{MMU Slot 3 Control}
\begin{itemize}
\item bits 7-0 = 8k RAM page at position \$6000 to \$7FFF (\$0B on
  reset)
\end{itemize}
Pages can be from 0 to 223 on a fully expanded Next.

\register{R/W}{54}{MMU Slot 4 Control}
\begin{itemize}
\item bits 7-0 = 8k RAM page at position \$8000 to \$9FFF (\$04 on
  reset)
\end{itemize}
Pages can be from 0 to 223 on a fully expanded Next.

\register{R/W}{55}{MMU Slot 5 Control}
\begin{itemize}
\item bits 7-0 = 8k RAM page at position \$A000 to \$BFFF (\$05 on
  reset)
\end{itemize}
Pages can be from 0 to 223 on a fully expanded Next.

\register{R/W}{56}{MMU Slot 6 Control}
\begin{itemize}
\item bits 7-0 = 8k RAM page at position \$C000 to \$DFFF (\$00 on
  reset)
\end{itemize}
Pages can be from 0 to 223 on a fully expanded Next.

\register{R/W}{57}{MMU Slot 7 Control}
\begin{itemize}
\item bits 7-0 = 8k RAM page at position \$E000 to \$FFFF (\$01 on
  reset)
\end{itemize}
Pages can be from 0 to 223 on a fully expanded Next.

Writing to ports \$1FFD, \$7FFD and \$DFFD writes \$FF to MMU0 and
MMU1 and writes appropriate values to MMU6 and MMU7 to map in the
selected 16k bank.

+3 special modes override the MMUs if used.

\register{W}{60}{Copper Data 8-bit Write}
\begin{itemize}
\item bits 7-0 = Byte to write to copper instruction memory
\end{itemize}
Note that each copper instruction is two bytes long, after a write,
the coppen address is auto-incremented to the next memory position.

After a write, the index is auto-incremented to the next memory position.

\register{W}{61}{Copper Address LSB}
\begin{itemize}
\item bits 7-0 = Copper instruction memory address LSB (0 on reset)
\end{itemize}

\register{W}{62}{Copper Control}
\begin{itemize}
\item bits 7-6 = Start Control
  \begin{itemize}
  \item[] 00 = Copper fully stopped
  \item[] 01 = Copper start, execute the list from index 0, and loop
    to the start
  \item[] 10 = Copper start, execute the list from last point, and
    loop to the start
  \item[] 11 = Copper start, execute the list from index 0, and
    restart the list when the raster reaches position (0,0)
  \end{itemize}
\item bits 2-0 = Copper instruction memory address (MSB) (0 on reset)
\end{itemize}

\register{W}{63}{Copper Data 16-bit Write}
\begin{itemize}
\item bits 7-0 = Byte to write to copper instruction memory
\end{itemize}
The 16-bit value is written in pairs. The first 8-bits are the MSB and
are destined for an even copper instruction address. The sesond 8-bits
are the LSB and are destined for an odd copper instruction address.

After each write, the copper address is auto-incremented to the next
memory position.

After a write to an odd address, the all 16-bits are written to copper
memory at once.

\register{R/W}{68}{ULA Control}
\begin{itemize}
\item bit 7 = Disable ULA output (0 on reset)
\item bit 6 = Color blending control for layering modes 6 \& 7
  [S(L+U)] (0 on reset)
  \begin{itemize}
  \item 0 = Layer 0
  \item 1 = Layer 0+3
  \end{itemize}
\item bits 5-4 = Reserved must be 0
\item bit 3 = Enable ULAplus (0 on reset)
\item bit 2 = Enable ULA half pixel scroll (0 on reset)
\item[] may change
\item bit 0 = Enable stencil mode (0 on reset)
\item[] When ULA and Layer 3 are enabled, if either are transparent,
  the resulr is transparent, otherwise the result is the logical AND
  of both colours.
\end{itemize}

\register{R/W}{69}{Display Control 1}
\begin{itemize}
\item bit 7 = Layer 2 Enable (Port \$123B bit 1 alias)
\item bit 6 = ULA Shadow display enable (Port \$7FFD bit 3 alias)
\item bits 5-0 = Timex alias (Port \$FF alias)
\end{itemize}

\register{R/W}{6A}{Layer 1,0 (LoRes) Control}
\begin{itemize}
\item bits 7-6 = reserved, must be 0
\item bit 5 = Enable Radistan (16-colour) (0 on reset)
\item bit 4 = Radistan DFILE switch (xor with bit 0 of port \$ff) (0
  on reset)
\item bits 3-0 = Radistsan palette offset (0 on reset)
\item bits 1-0 = ULAplus palette offset (0 on reset)
\end{itemize}

\register{R/W}{6B}{Layer 3 (Tilemap) Control}
\begin{itemize}
\item bit 7 = Layer 3 Enable (0 on reset)
\item bit 6 = Layer 3 Size control (0 on reset)
  \begin{itemize}
  \item 0 = 40x32
  \item 1 = 80x32
  \end{itemize}
\item bit 5 = Disable Arrtibute Entry (0 on reset)
\item bit 4 = palette select (0 on reset)
\item bit 3 = Enable Text mode (1-bit tilemap) (0 on reset)
\item bit 2 = Reserved, must be 0
\item bit 1 = Activate 512 tile mode (0 on reset)
\item bit 0 = Enable Layer 3 on top of ULA (0 on reset)
\end{itemize}

\register{R/W}{6C}{Default Layer 3 Attribute}*
\begin{itemize}
\item bits 7-4 = Palette Offset (\$00 on reset)
\item bit 3 = X mirror (0 on reset)
\item bit 2 = Y mirror (0 on reset)
\item bit 1 = Rotate (0 on reset)
\item bit 0 = Bit 8 of the tile number (512 tile mode) (0 on reset)
\item bit 0 = ULA over tilemap (256 tile mode) (0 on reset)
\end{itemize}
*Active tile attribute if bit 5 of nextreg \$6B is set.

\register{R/W}{6E}{Layer 3 Tilemap Base Address}
\begin{itemize}
\item bits 7-6 = Read back as zero, write values ignored
\item bits 5-0 = MSB of address of the tilemap in Bank 5 (\$2C on
  reset)
\end{itemize}
Soft Reset default \$2C - This is because the address is \$6C00 so the
MSB is \$6C. But the stored value is only the lower 6 bits so it's an
offset into the 16k Bank 5. To calculate therefore subtract \$40
leaving you with \$2C.

The value written is an offset into the 16k Bank 5 allowing the
tilemap to be placed at any multiple of 256 bytes.  Writing a physical
MSB address in \$40 -- \$7F or \$C0 -- \$FF range is permitted.

The value read back should be treated as having a fully significant
8-bit value.

\register{R/W}{6F}{Layer 3 Tile Definitions Base Address}
\begin{itemize}
\item bits 7-6 = Read back as zero, write values ignored
\item bits 5-0 = MSB of address of the tilemap in Bank 5 (\$0C on
  reset)
\end{itemize}
Soft Reset default \$0C - This is because the address is \$4C00 so the
MSB is \$4C. But the stored value is only the lower 6 bits so it's an
offset into the 16k Bank 5. To calculate therefore subtract \$40
leaving you with \$0C.

The value written is an offset into the 16k Bank 5 allowing the
tilemap to be placed at any multiple of 256 bytes.  Writing a physical
MSB address in \$40 -- \$7F or \$C0 -- \$FF range is permitted.

The value read back should be treated as having a fully significant
8-bit value.

\register{R/W}{70}{Layer 2 Control}
\begin{itemize}
\item bits 7-6 = Reserved, must be 0
\item bits 5-4 = Resolution (00 on soft reset)
  \begin{itemize}
  \item 00 = $256\times192\times256$
  \item 01 = $320\times256\times256$
  \item 10 = $640\times256\times16$
  \item 11 = Do not use
  \end{itemize}
\item bits 3-0 = Palette offset (\$0 on soft reset)
\end{itemize}

\register{W}{75}{Sprite Attribute 0 (Auto-incrementing)}\\
See nextreg \$35

\register{W}{76}{Sprite Attribute 1 (Auto-incrementing)}\\
See nextreg \$36

\register{W}{77}{Sprite Attribute 2 (Auto-incrementing)}\\
See nextreg \$37

\register{W}{78}{Sprite Attribute 3 (Auto-incrementing)}\\
See nextreg \$38

\register{W}{79}{Sprite Attribute 4 (Auto-incrementing)}\\
See nextreg \$39

\register{R/W}{7F}{User Register 0}
\begin{itemize}
\item bits 7-0 = User Register (\$FF on reset)
\end{itemize}

Caution NextReg numbers above \$7F are inaccessible to the Copper

\register{R/W}{80}{Expansion Bus Enable}\\
Immediate
\begin{itemize}
\item bit 7 = Expansion Bus Enable (0 on hard reset)
\item bit 6 = reserved, must be 0
\item bit 5 = I/O cycle Disable/Ignore \textoverline{IORQULA} (0 on
  hard reset)
\item bit 4 = Memory cycle Disable/Ignore \textoverline{ROMCS} (0 on
  hard reset)
\end{itemize}
After Soft Reset (Copied into bits 7-4)
\begin{itemize}
\item bit 3 = Expansion Bus Enable (0 on hard reset)
\item bit 2 = reserved, must be 0
\item bit 1 = I/O cycle Disable/Ignore \textoverline{IORQULA} (0 on
  hard reset)
\item bit 0 = Memory cycle Disable/Ignore \textoverline{ROMCS} (0 on
  hard reset)
\end{itemize}

\register{R/W}{81}{Expansion Bus Control}
\begin{itemize}
\item bit 7 = (R) Expansion bus \textoverline{ROMCS} asserted
\item bits 6-5 = Reserved, must be 0
\item bit 4 = (W) Propagate max CPU clock at all times (0 on hard
  reset)
\item bits 3-2 = Reserved, must be 0
\item bits 1-0 = Max CPU Speed when Expansion Bus is enabled (\$00 on
  hard reset, currently fixed at \$00)
  \begin{itemize}
  \item 00 = 3.5 MHz
  \item 01 = 7 MHz
  \item 10 = 14 MHz
  \item 11 = 28 MHz
  \end{itemize}
\end{itemize}

\register{R/W}{82}{Internal Port decoding control 1/4}
\begin{itemize}
\item bit 7 = Enable Kempston Port 2 (Port \$37) (1 on reset)
\item bit 6 = Enable Kempston Port 1 (Port \$1F) (1 on reset)
\item bit 5 = Enable DMA (Port \$6B) (1 on reset)
\item bit 4 = Enable +3 Floating Bus (1 on reset)
\item bit 3 = Enable +3 Paging (Port \$1FFD) (1 on reset)
\item bit 2 = Enable Next Memory Paging (Port \$DFFD) (1 on reset)
\item bit 1 = Enable Paging (Port \$7FFD) (1 on reset)
\item bit 0 = Enable Timex (Port \$FF) (1 on reset)
\end{itemize}

\register{R/W}{83}{Internal Port decoding control 2/4}
\begin{itemize}
\item bit 7 = Enable Layer 2 (Port \$123B) (1 on reset)
\item bit 6 = Enable Sprites (Ports \$57, \$5B, \$303B) (1 on reset)
\item bit 5 = Enable Kempston Mouse (Ports \$FADF, \$FBDF, \$FFDF) (1
  on reset)
\item bit 4 = Enable UART (Ports \$133B, \$143B, \$153B) (1 on reset)
\item bit 3 = Enable SPI (Ports \$E7, \$EB) (1 on reset)
\item bit 2 = Enable \iic (Ports \$103B, \$113B) (1 on reset)
\item bit 1 = Enable Multiface (two variable ports) (1 on reset)
\item bit 0 = Enable divMMC (Port \$E3) (1 on reset)
\end{itemize}

\register{R/W}{84}{Internal Port decoding control 3/4}
\begin{itemize}
\item bit 7 = Enable SPECdrum Mono DAC (Port \$DF) (1 on reset)
\item bit 6 = Enable Covox/GS Mono DAC (Port \$B3) (1 on reset)
\item bit 5 = Enable Pentagon/ATM DAC (Port \$FB) (1 on reset)
\item bit 4 = Enable Covox Stereo DAC (Ports \$0F, \$4F) (1 on reset)
\item bit 3 = Enable Profi/Covox Stereo DAC (Ports \$3F, \$5F) (1 on
  reset)
\item bit 2 = Enable Soundrive DAC Mode 2 (Ports \$F1, \$F3, \$F9,
  \$FB) (1 on reset)
\item bit 1 = Enable Soundrive DAC Mode 1 (Ports \$0F, \$1F, \$4F,
  \$5F) (1 on reset)
\item bit 0 = Enable AY (Ports \$FFFD, \$BFFD) (1 on reset)
\end{itemize}

\register{R/W}{85}{Internal Port decoding control 4/4}
\begin{itemize}
\item bits 7-1 = Reserved
\item bit 0 = Enable ULAplus (Ports \$BF3B, \$FF3B) (1 on reset)
\end{itemize}

\register{R/W}{86}{Expansion Port decoding control 1/4}
\begin{itemize}
\item bit 7 = Enable Kempston Port 2 (Port \$37) (1 on reset)
\item bit 6 = Enable Kempston Port 1 (Port \$1F) (1 on reset)
\item bit 5 = Enable DMA (Port \$6B) (1 on reset)
\item bit 4 = Enable +3 Floating Bus (1 on reset)
\item bit 3 = Enable +3 Paging (Port \$1FFD) (1 on reset)
\item bit 2 = Enable Next Memory Paging (Port \$DFFD) (1 on reset)
\item bit 1 = Enable Paging (Port \$7FFD) (1 on reset)
\item bit 0 = Enable Timex (Port \$FF) (1 on reset)
\end{itemize}

\register{R/W}{87}{Expansion Port decoding control 2/4}
\begin{itemize}
\item bit 7 = Enable Layer 2 (Port \$123B) (1 on reset)
\item bit 6 = Enable Sprites (Ports \$57, \$5B, \$303B) (1 on reset)
\item bit 5 = Enable Kempston Mouse (Ports \$FADF, \$FBDF, \$FFDF) (1
  on reset)
\item bit 4 = Enable UART (Ports \$133B, \$143B, \$153B) (1 on reset)
\item bit 3 = Enable SPI (Ports \$E7, \$EB) (1 on reset)
\item bit 2 = Enable \iic (Ports \$103B, \$113B) (1 on reset)
\item bit 1 = Enable Multiface (two variable ports) (1 on reset)
\item bit 0 = Enable divMMC (Port \$E3) (1 on reset)
\end{itemize}

\register{R/W}{88}{Expansion Port decoding control 3/4}
\begin{itemize}
\item bit 7 = Enable SPECdrum Mono DAC (Port \$DF) (1 on reset)
\item bit 6 = Enable Covox/GS Mono DAC (Port \$B3) (1 on reset)
\item bit 5 = Enable Pentagon/ATM DAC (Port \$FB) (1 on reset)
\item bit 4 = Enable Covox Stereo DAC (Ports \$0F, \$4F) (1 on reset)
\item bit 3 = Enable Profi/Covox Stereo DAC (Ports \$3F, \$5F) (1 on
  reset)
\item bit 2 = Enable Soundrive DAC Mode 2 (Ports \$F1, \$F3, \$F9,
  \$FB) (1 on reset)
\item bit 1 = Enable Soundrive DAC Mode 1 (Ports \$0F, \$1F, \$4F,
  \$5F) (1 on reset)
\item bit 0 = Enable AY (Ports \$FFFD, \$BFFD) (1 on reset)
\end{itemize}

\register{R/W}{89}{Expansion Port decoding control 4/4}
\begin{itemize}
\item bits 7-1 = Reserved
\item bit 0 = Enable ULAplus (Ports \$BF3B, \$FF3B) (1 on reset)
\end{itemize}

The Internal Port Decoding Enables always apply.

When the Expansion Bus is enabled, the Expansion Bus Port Decoding
Enables are logically ANDed with the Internal Enables. A result of 0
for the corresponding bit indicates the internal device is
\emph{disabled}. If the Expansion Bus is enabled, this allows
I/O cycles for disabled ports to propagate to the Expansion Bus,
otherwise corresponding I/O cycles to the Expansion Bus are filtered.

\register{R/W}{8A}{Expansion Bus I/O Propagate Control}
\begin{itemize}
\item bits 7-4 = Reserved, must be 0
\item bit 3 = Propagate port \$1FFD I/O Cycles (0 on hard reset)
\item bit 2 = Propagate port \$DFFD I/O Cycles (0 on hard reset)
\item bit 1 = Propagate port \$7FFD I/O Cycles (0 on hard reset)
\item bit 0 = Propagate port \$FE I/O Cycles (1 on hard reset)
\end{itemize}

\register{R/W}{8C}{Alternate ROM}\\
Immediate
\begin{itemize}
\item bit 7 = Alt ROM Enable (0 on hard reset)
\item bit 6 = Alt ROM visible ONLY during writes (0 on hard reset)
\item bit 5 = Reserved, must be 0
\item bit 4 = 48k ROM Lock (0 on hard reset)
\end{itemize}
After Soft Reset (copied into bits 7-4)
\begin{itemize}
\item bit 3 = Alt ROM Enable (0 on hard reset)
\item bit 2 = Alt ROM visible ONLY during writes (0 on hard reset)
\item bit 1 = Reserved, must be 0
\item bit 0 = 48k ROM Lock (0 on hard reset)
\end{itemize}

\register{R/W}{90}{Pi GPIO output enable 1/4}
\begin{itemize}
\item bit 7 = Enable Pin 7 (0 on reset)
\item bit 6 = Enable Pin 6 (0 on reset)
\item bit 5 = Enable Pin 5 (0 on reset)
\item bit 4 = Enable Pin 4 (0 on reset)
\item bit 3 = Enable Pin 3 (0 on reset)
\item bit 2 = Enable Pin 2 (0 on reset)
\item bit 1 = Enable Pin 1 (cannot be enabled) (0 on reset)
\item bit 0 = Enable Pin 0 (cannot be enabled) (0 on reset)
\end{itemize}

\register{R/W}{91}{Pi GPIO output enable 2/4}
\begin{itemize}
\item bit 7 = Enable Pin 15 (0 on reset)
\item bit 6 = Enable Pin 14 (0 on reset)
\item bit 5 = Enable Pin 13 (0 on reset)
\item bit 4 = Enable Pin 12 (0 on reset)
\item bit 3 = Enable Pin 11 (0 on reset)
\item bit 2 = Enable Pin 10 (0 on reset)
\item bit 1 = Enable Pin 9 (0 on reset)
\item bit 0 = Enable Pin 8 (0 on reset)
\end{itemize}

\register{R/W}{92}{Pi GPIO output enable 3/4}
\begin{itemize}
\item bit 7 = Enable Pin 23 (0 on reset)
\item bit 6 = Enable Pin 22 (0 on reset)
\item bit 5 = Enable Pin 21 (0 on reset)
\item bit 4 = Enable Pin 20 (0 on reset)
\item bit 3 = Enable Pin 19 (0 on reset)
\item bit 2 = Enable Pin 18 (0 on reset)
\item bit 1 = Enable Pin 17 (0 on reset)
\item bit 0 = Enable Pin 16 (0 on reset)
\end{itemize}

\register{R/W}{93}{Pi GPIO output enable 4/4}
\begin{itemize}
\item bits 7-4 = Reserved
\item bit 3 = Enable Pin 27 (0 on reset)
\item bit 2 = Enable Pin 26 (0 on reset)
\item bit 1 = Enable Pin 25 (0 on reset)
\item bit 0 = Enable Pin 24 (0 on reset)
\end{itemize}

\register{R/W}{98}{Pi GPIO Pin State 1/4}
\begin{itemize}
\item bit 7 = Pin 7 Data (1 on reset)
\item bit 6 = Pin 6 Data (1 on reset)
\item bit 5 = Pin 5 Data (1 on reset)
\item bit 4 = Pin 4 Data (1 on reset)
\item bit 3 = Pin 3 Data (1 on reset)
\item bit 2 = Pin 2 Data (1 on reset)
\item bit 1 = Pin 1 Data (1 on reset)
\item bit 0 = Pin 0 Data (1 on reset)
\end{itemize}

\register{R/W}{99}{Pi GPIO Pin State 2/4}
\begin{itemize}
\item bit 7 = Pin 15 Data (1 on reset)
\item bit 6 = Pin 14 Data (1 on reset)
\item bit 5 = Pin 13 Data (1 on reset)
\item bit 4 = Pin 12 Data (1 on reset)
\item bit 3 = Pin 11 Data (1 on reset)
\item bit 2 = Pin 10 Data (1 on reset)
\item bit 1 = Pin 9 Data (1 on reset)
\item bit 0 = Pin 8 Data (1 on reset)
\end{itemize}

\register{R/W}{9A}{Pi GPIO Pin State 3/4}
\begin{itemize}
\item bit 7 = Pin 23 Data (1 on reset)
\item bit 6 = Pin 22 Data (1 on reset)
\item bit 5 = Pin 21 Data (1 on reset)
\item bit 4 = Pin 20 Data (1 on reset)
\item bit 3 = Pin 19 Data (1 on reset)
\item bit 2 = Pin 18 Data (1 on reset)
\item bit 1 = Pin 17 Data (1 on reset)
\item bit 0 = Pin 16 Data (1 on reset)
\end{itemize}

\register{R/W}{9B}{Pi GPIO Pin State 4/4}
\begin{itemize}
\item bits 7-4 = Reserved
\item bit 3 = Pin 27 Data (1 on reset)
\item bit 2 = Pin 26 Data (1 on reset)
\item bit 1 = Pin 25 Data (1 on reset)
\item bit 0 = Pin 24 Data (1 on reset)
\end{itemize}

\register{R/W}{A0}{Pi Peripheral Enable}
\begin{itemize}
\item bits 7-6 = Reserved, must be 0
\item bit 5 = Enable UART on GPIO 14, 15 (0 on reset)*
\item bit 4 = Communication Type (0 on reset)
  \begin{itemize}
  \item 0 = Rx to GPIO 15, Tx to GPIO 14 (Pi)
  \item 1 = Rx to GPIO 14, Tx to GPIO 15 (Pi Hats)
  \end{itemize}
\item bit 3 = Enable \iic on GPIO 2, 3 (0 on reset)*
\item bits 2-1 = Reserved, must be 0
\item bit 0 = Enable SPI on GPIO 7, 8, 9, 10, 11 (0 on reset)*
\end{itemize}
*Overrides GPIO Enables

\register{R/W}{A2}{Pi \iis Audio Control}
\begin{itemize}
\item bits 7-6 = \iis State (\$00 on reset)
  \begin{itemize}
  \item 00 = \iis Disabled
  \item 01 = \iis is mono, source R
  \item 10 = \iis is mono, source L
  \item 11 = \iis is stereo
  \end{itemize}
\item bit 5 = Reserved, must be 0
\item bit 4 = Audio Flow Direction (0 on reset)
  \begin{itemize}
  \item 0 = PCM\_DOUT to Pi, PCM\_DIN from Pi (Hats)
  \item 1 = PCM\_DOUT from Pi, PCM\_DIN to Pi (Pi)
  \end{itemize}
\item bit 3 = Mute left (0 on reset)
\item bit 2 = Mute right (0 on reset)
\item bit 1 = Slave (PCM\_CLK, PCM\_FS external) (0 on reset)
\item bit 0 = Direct \iis audio to EAR on port \$FE
\end{itemize}

\register{R/W}{A3}{Pi \iis Clock Divide (Master Mode)}
\begin{itemize}
\item bits 7-0 = Clock divide value (\$0B on reset)
\end{itemize}
$\hbox{Divider}=\frac{538461}{\hbox{Rate}}-1$ or $\hbox{Rate}=\frac{538461}{\hbox{Divider}+1}$

\register{W}{FF}{Debug LEDs (DE-1, DE-2 am Multicore only)}
