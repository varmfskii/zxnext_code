\register{R/W}{05}{Peripheral 1 Settings}
\begin{itemize}
\item bits 7-6 = joystick 1 mode (MSB)
\item bits 5-4 = joystick 2 mode (MSB)
\item bit 3 = joystick 1 mode (LSB)
\item bit 2 = 50/60 Hz mode (0 = 50Hz, 1 = 60Hz)
\item bit 1 = joystick 2 mode (LSB)
\item bit 0 = Enable Scandoubler
\begin{itemize}
  \item[] 0 = Disabled for CRT
  \item[] 1 = Enabled for VGA
\end{itemize}
\end{itemize}
Joystick modes
\begin{itemize}
\item 000 = Sinclair 2 (67890)
\item 001 = Kempston 2 (port \$37)
\item 010 = Kempston 1 (port \$1F)
\item 011 = Megadrive 1 (port \$1F)
\item 100 = Cursor
\item 101 = Megadrive 2 (port \$37)
\item 110 = Sinclair 1 (12345)
\item 111 = User Defined Keys Joystick
\end{itemize}
\begin{itemize}
\item[*] Joysticks can be placed in i/o mode via nextreg 0x0B.
\item[*] Programming the user defined keys joystick is done through the ps2
keymap interface on nextreg 0x28, 0x29 and 0x2B:
\end{itemize}
\begin{enumerate}
\item Write 128 to nextreg 0x28
\item Write 0 (left joystick) or 16 (right joystick) to nextreg 0x29
\item Write eleven bytes to nextreg 0x2B. The bytes correspond to the eleven
buttons on an md pad (X=11 Z Y START A C B U D L R=1)
\item Each byte written identifies a key in the 8x7 membrane; bits 5:3 select
the row and bits 2:0 select the column with 111 meaning no action.\\ *
In all joystick modes, excess buttons on an md pad not read via ports
will generate key input if so programmed.
\end{enumerate}

