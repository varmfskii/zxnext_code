 

  • To come in
  • check in
  • Password recovery

[                    ]

[                    ] To come in

[*] Remember me 
[                    ]

Get link to change password
[strings]

  • Topics
  • Blogs
  • People
  • Activity
  • The calendar
  • FAQ
  • still

  • To register
  • To come in

  • Everything
  • Collective
  • Personal

To find
[                    ] []
Development

20 readers
57 topics
RSS feed

news

  • Calendar for 2019

    Blastoff December 31, 2018, 16:00 2

  • 3aRulem №24

    Blastoff December 8, 2018 21:11 0

  • 3aRulem # 24

    Blastoff November 10, 2018 09:53 eleven

  • Retro DG Mag Contest

    mr_r0ckers October 26, 2018, 14:21 one

  • Demoscene - art in numbers

    nyuk October 23, 2018 12:50 one

All News · RSS

Live

  • Comments
  • Publications

  • Comments
      □ Comments
      □ Publications

  • [avatar_mal]

    aa-dav July 22, 2019 07:15

    8/16-bit dream computer (Simpleton processor) 3
  • [avatar_mal]

    Winny July 6, 2019 11:40

    Era is leaving 18
  • [avatar_24x]

    tae1980 July 2, 2019 08:40

    ZX-Art 06.2019: maps and geocoding eight
  • [avatar_24x]

    pixelrat June 21, 2019 09:45

    chipwiki: pixel art ten
  • [avatar_24x]

    moroz1999 June 20, 2019 12:18

    ZX-Art 05.2019: Pouet, Simple HTML, cache 9
  • [avatar_24x]

    sq June 15, 2019 21:08

    We write video from the emulator. How? 41
  • [avatar_24x]

    Shiru June 14, 2019 11:51

    EVALUA - VST for mathematicians 6
  • [avatar_24x]

    Vbi June 1, 2019 11:46

    Game Boy Architecture Overview (+ Color) four

Entire broadcast · RSS
% logo-img%
% title%

% date%

[events-ico] events

[INS::INS]

Hyperadio

                                 [hyperadio-]
                                 [hyperadio-]
                              reed - flamethrower
━━━━━━━━━━━━━━━━━━━━━━━━━━━━━━━━━━━━━━━━━━━━━━━━━━━━━━━━━━━━━━━━━━━━━━━━━━━━━━━
                             listeners: 1 (4 peak)
                                 M3U XSPF VCLT

Best month

  • moroz1999 June 30, 2019 10:10

    Blog them. moroz1999 → ZX-Art 06.2019: maps and geocoding eight
  • aa-dav July 20, 2019 09:48

    Blog them. aa-dav → 8/16-bit dream computer (Simpleton processor) 3

Tags

  • All tags

[                    ]
  • 14days
  • Parulem
  • 3bm
  • 8bit
  • artfield
  • ay
  • basic
  • beeper
  • c64
  • Captain Drexx
  • cc
  • chaos constructions
  • chiptune
  • coding
  • compo
  • csp
  • demo
  • demoparty
  • demoscene
  • design
  • development
  • dihalt
  • famicom
  • faq
  • game
  • Game Boy Advance
  • GBA
  • gfx
  • graphics
  • hype
  • making of
  • multimatograf
  • music
  • musicdisk
  • nes
  • NOT Soft
  • party
  • Podcast
  • realtime
  • review
  • scene
  • ts docs
  • TS-Config
  • verve
  • video
  • vortex tracker
  • Weekly game club
  • z80
  • zx spectrum
  • Zxtune
  • video chip
  • graphics
  • demo
  • demoscene
  • wild ad
  • Behind rulem
  • games
  • Coding
  • contests
  • music
  • music for man
  • overview
  • party
  • transfer
  • programming
  • Lesson
  • 2015
  • 2016
  • 2017
  • 2018

Blogs

  • Top

  • Development 16.72
  • News 13.65
  • Sound 7.52
  • VERVE 7.47
  • Events 6.45
  • Graphics 6.45
  • Pixel Talks 5.61
  • Demo 5.31

All blogs

TSconf: ints

Development  
[14_1]
TSconf has an extended system of interrupts that can be triggered if there are
such states as: beam arrival at a given screen position, beam arrival at the
beginning of a line, line display on the screen, completion of data transfer.


    Tao says: The system has three types of masked interrupts that can be
    called at an address that has a high byte — the address in register I, and
    the youngest — its type:
     1. #FF - frame
     2. #FD - lowercase (Line)
     3. #FB - DMA.
    The processing of these interrupts can be switched by the INTMask port (#
    2A af), changing the state of the bits:
    0 - Frame, 1 - Line, 2 - DMA, which leads to an on / off call to handlers.
    State of bits: 0 - disabled / 1 - enabled.
    In the case of the arrival of several events at the same time, the
    interrupt with the lower number will be processed first.



How can we manage these capabilities? Consider in order:

Frame A
frame interrupt in the standard Spectrum comes at the very beginning of the
screen drawing. For TSconf, this parameter can be set using three ports:

  • VSINT: the high and low bytes of these two ports (VSINTH - # 24af, VSINTL -
    # 23af) allow you to specify the vertical position of the beam, which
    allows us to specify the entire area of ​​the screen lines: 0 - 319;
  • HSINT: port indicating the horizontal position of the beam (position in the
    line): 0–223 in 3.5 MHz cycles,

If the position of the beam coincides, a frame interrupt handler will be
called. If you specify values ​​outside the specified range, the interrupt will
not be generated.

So let's give an example in which the frame interrupt handler #BE will be
installed:

im2_init
                xor a
                ld bc,HSINT
                out (c),a
                ld bc,VSINTL
                out (c),a
                ld bc,VSINTH
                out (c),a

                ld a,#be
                ld i,a
                ld hl,int
                ld (#beff),hl
                im 2
                ret


In this case, the positions for calling the frame interrupt handler were set to
0 (upper left corner of the screen, leftmost position).
When the ray of the screen rendering arrives at this position, the transition
will be made to the personnel interrupt handler code by the int tag.
In this case, this handler will be called with a frame pulse frequency of 50
Hz.

And if we need more?
Consider the situation when we need to call the personnel handler at the
beginning of the screen, and at the beginning of the line at number 128.
Why? So that the upper part of the screen was colored red!
To do this, we will need to specify when we need to call the handler in each
next time. In the handler, the position of the screen 0 (the first line of the
screen) will indicate in which place of the screen the following personnel
interruption will be triggered:

int
                push af
                push bc
                push hl
                ld a,128
                ld bc,VSINTL
                out (c),a
                xor a
                ld bc,VSINTH
                out (c),a
                ld hl,int128
                ld (#beff),hl
                ld a,#f2
                ld bc,BORDER
                out (c),a
                pop hl
                pop bc
                pop af
                ei
                ret


Here we indicate that the next personnel interrupt should be called in the
vertical position of the beam on line 128, and the handler will be the int128
subroutine. Set the red color from the standard zx spectrum palette (palette #
0f, color 2 - for the border color, simply specify the desired color from the
entire 256-cell palette).

When the beam hits the beginning of line 128, the subroutine int128 is called:

int128
                push af
                push bc
                push hl
                xor a
                ld bc,VSINTL
                out (c),a
                ld bc,VSINTH
                out (c),a
                ld hl,int
                ld (#beff),hl
                ld a,#f0
                ld bc,BORDER
                out (c),a
                pop hl
                pop bc
                pop af
                ei
                ret


in which we indicated that the next personnel interruption will be processed by
the int subroutine at the beam position 0, and set the border color to 0 of the
standard palette.

We note that the HSINT horizontal position port does not change now and is set
to 0, therefore the call to the handlers is always at the beginning of the
line. I suggest you play with it yourself :)

Great, we can already get to the right place on the screen. But what if you
need a certain subroutine to execute EVERY line?
To do this, we have a tool in the form of a string interrupt.

Line interrupts
This type of interrupt will call its handler when the beam is at position 0 of
each line of the screen (whether it is visible or not on the screen is not
important for the system).

To use it, you must do the following:

  • set the address of the line interrupt handler
  • allow INTMASK port processing (# 2Aaf)


So, turn on the bits of the INTMASK port, which allow processing of both
personnel and line interrupts, set the address of the handler:

                ld      hl,line_proc
                ld      (#befd),hl

                ld      bc,INTMASK
                ld      a,%00000011
                out     (c),a


We note to ourselves that both personnel and line interrupts are now allowed.
The first handler will be processed with a smaller number - i.e. first
personnel, then lower case.

Change the task. Let each line on the border be visible strip of a different
color.
Disable the frame handler, leaving only the lowercase one:

                ld      hl,line_proc
                ld      (#befd),hl

                ld      bc,INTMASK
                ld      a,%00000010
                out     (c),a



Add a line interrupt handler:

line_proc
                exx
                ex af, af'
lp1             ld a,#ff
                inc a
                ld (lp1+1),a
                or #f0
                ld bc,BORDER
                out (c),a
                ex af, af'
                exx
                ei
                ret


Since this handler will be called very often, it is worth minimizing the
processor clock used by switching to an alternate set of registers. Then we
increase the color number counter in the palette and send it to the system
(border port # 0Faf).

The peculiarity of these interrupt systems is that using the HALT “stop the
processor” command so familiar to us will only wait for the next personnel /
line interrupt, that is, for the first example, drawing a full screen takes 2
HALTs and 320 HALT for a lowercase :)

DMA
Handler to complete the transfer will be called after the transfer is
completed. Your KO ;)
Turn on its processing and specify the address of the handler. As usual:

                ld      hl,dma_proc
                ld      (#befb),hl

                ld      bc,INTMASK
                ld      a,%00000100
                out     (c),a


Note that in this case, we have completely disabled frame and line interrupts,
and their handlers will NOT be called.

[Question-Mark-Dude1-300x300]Questions and Answers
:? Cho music when lowercase spoils?
! Alas, the standard pt3 player uses the stack in operation, and when calling
the line processor, the data at the address in the SP is replaced with its call
address. And spoil the muse data.
Accordingly, we cause the player to turn off the lowercase ones (for example,
outside the screen, set up personnel), or use the psg player.
... or playing a non-stackable player.

? And how do you play with fullscreen effects?
! 1) Heh, lowercase interrupts are called every line. Accordingly, we obtain a
lower frequency frequency of 15625 hertz, and each line is thrown into the
audio coking port. Naman? And do not forget the previous question ^^^
2) The case, if only the 15625 sound, and the personnel interruption is long
enough, then allow (nested) interrupts in the personnel handler. Plus, you can
move the personnel int (for example) to the middle of the line, so that they
would not come at the same time as the lower case.
Literature:
FAQ

  • 
  • TS-Config
  • , INTs
  • , Ts the docs

Share ...       

  • avatar Vbi
  • May 8, 2015, 13:18
  • 
  • 
  • 
    +12
      □ 12
      □ 0
      □ 0

2 comments

avatar
Oh, it turns out there is a one-shot on the right line :)

  • wbcbz7
  • May 8, 2015, 18:43
  • 
    0
  •  
  • ↓

avatar
Добавлю два нюанса.

Порядок событий, связанных с приходом сигнала прерывания выглядит так:
0. включение триггера соответствующим источником, при этом на процессор
выставляется !INT (при наличие хотя бы одного активного триггера)
1. процессор выполняет последний машцикл инструкции, цепляет инт
2. процессор генерирует цикл подтверждения им2
3. логика контроллера определяет, какой вектор выставить на шину данных (в
зависимости от включенных триггеров и их приоритетов)
4. проц цепляет вектор, читает из памяти адрес, летит на ISR
5. контроллер выключает тот триггер, вектор которого обработан; если остались
другие триггеры либо успели прийти новые события, инт не убирается
6. процессор заканчивает обработку ISR командами EI:RET, при этом новый инт
(если сигнал активен) обработается на последнем машцикле RET, с п.1.

Нюанс первый.
Имеем следующие события:
— установился триггер источника с приоритетом 2,
— прошло 3 такта,
— установился триггер источника с приоритетом 0,
— прошло 2 такта,
— процессор распознал инт и запросил вектор.
Вопрос: какой вектор обработается?
Ответ: источника с приоритетом 0. Несмотря на то, что по тактам событие
источника 2 пришло раньше.
Глубокая мораль: приоритеты обрабатываются не по факту прихода, а по факту
наличия на момент подтверждения цикла им2.

Нюанс второй.
If during the active trigger to extinguish the corresponding bit in the
register of masks, then this source will not only not be processed, but the
trigger will be reset. This is necessary for two things: so that the “hanging”
raw ints do not remain after turning off the mask bits and so that you can
steer the resolutions of lower priorities from the highest ones.

  • tsl
  • May 12, 2015, 18:42
  • 
    +4
  •  
  • ↓

Only registered and authorized users can leave comments.

  • You can
  • To register
  • To come in

  • Sections
  • Topics
  • Blogs
  • People
  • Activity
  • mobile version

  • The calendar
  • 2015
  • 2016
  • 2017
  • 2018

© Powered by LiveStreet CMS
xeoart Design by xeoart
2012
 

Google Translate

Original text

Contribute a better translation
━━━━━━━━━━━━━━━━━━━━━━━━━━━━━━━━━━━━━━━━━━━━━━━━━━━━━━━━━━━━━━━━━━━━━━━━━━━━━━━
Share with friends
In contact withClassmatesTwitterFacebookMy worldLivejournalGoogle Plus Yandex
