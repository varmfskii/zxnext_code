% This file was converted from HTML to LaTeX with
% gnuhtml2latex program
% (c) Tomasz Wegrzanowski <maniek@beer.com> 1999
% (c) Gunnar Wolf <gwolf@gwolf.org> 2005-2010
% Version : 0.4.
\documentclass{article}
\begin{document}
\section*{TSconf: Ports}

Development

In this section, I would like to describe all the port groups involved
in the management of the TSconf system.

This section describes the general purpose of ports. The complete dock
for the ports lies here .

So, the system ports can be divided into the following groups:

\begin{itemize}
\item control ports for code execution by the processor. This includes
  such ports as: SysConfig, CacheConfig
\item memory management ports: MemConfig, Page0 - Page3, FMAddr
\item TSU graphics page ports: VPage, T0GPage, T1GPage, TMPage, SGPage
\item Graphics / Color Management ports: VConfig, PalSel, Border
\item display control ports: TSConfig, GXOffs / GYOffs, T0XOffs /
  T0YOffs, T1XOffs / T1YOffs
\item DMA management ports: DMASAddr, DMADAddr, DMALen, DMANum,
  DMACtrl
\item ports of management of arrival INT: INTMask, HSINT, VSINT
\item Virtual FDD Management Port: FDDVirt
\end{itemize}

\begin{quotation}
  \textbf{Tao says:} Configuration management occurs through the
  programming of ports. The system ports can also be programmed by
  writing data to FMaps from address \#0400.

  There are two types of ports according to the way data is specified:
  changing individual significant bits, or specifying 8-bit values.

  Data for some devices is collected from a pair of ports, in the
  names of which the letters H (high byne) \& L (low byte) appear.
\end{quotation}

So, consider in order:

\begin{quotation}
\textbf{Management ports for code execution by the
  processor. \emph{SysConfig, CacheConfig}}
\end{quotation}
The \textbf{SysConfig} port \emph{(\#20af) is} designed to control the
speed of the z80, allowing you to turn on the processor at speeds of
3.5, 7.0 and 14.0 MHz, and also allows you to enable the internal
cache of the system.

Cache usage can be controlled by the \textbf{CacheConfig} port
\emph{(\#2baf)} , which allows you to enable cache usage for all
memory windows: EN\_C000 EN\_8000 EN\_4000 EN\_0000

\begin{quotation}
  \textbf{System Cache Usage}
\end{quotation}
\textbf{The cache} saves processor access to RAM. With each reading,
the bytes read from the RAM and the adjacent bytes are copied into the
cache, and all the following references will be read from it. The
performance gain is achieved by the absence of processor braking,
which occurs when accessing physical. RAM, as well as saving RAM
cycles, which are available for other devices.

\emph{Example 1 (adjacent addresses):}

- 0x8000 address is read, RAM is accessed, data from 0x8000 and 0x8001

is cached - address is 0x8001, cache is accessed

\emph{Example 2 (non-adjacent addresses):}

- address is read 0x8001, RAM is accessed, cache data from addresses
0x8000 and 0x8001

- address 0x8002 is read, access to RAM, data from addresses 0x8002
and 0x8003 get to the cache

\textbf{The cache size is 512 bytes} , but the data memory is located
in such a way that it coincides with the lower nine bits of the
processor address. Therefore, successive accesses to data located at
addresses whose lower 9 bits match will cause a conflict and overrun
of previous data from the cache. Such a cache is called a direct
mapping cache and is most easily implemented for FPGA.

\emph{Example 1 (conflicting addresses):}

- 0x8000 address is read, RAM is accessed, data from 0x8000 and 0x8001

is cached - address is 0x8200, address is to RAM, data from 0x8200 and
0x8201 addresses are in the cache (data from 0x8000 and 0x8001
addresses are lost)

\emph{Example 2 (non-conflicting addresses):}

- read the address 0x8000, an appeal to the RAM, the cache get data
from 0x8000 addresses and 0h8001

- read address 0h8100, appeal to the RAM, the cache get data from
addresses 0h8100 and 0h8101 (data address 0x8000 and 0h8001 saved)

From this we can conclude that for more rational use of the cache, it
is advisable to place the code on "non-intersecting" addresses. For
example, if the use of a string inta is required. This is not very
convenient, but they resorted to other tricks on other platforms.

\emph{The cache takes into account page addressing} , so the data at
0xC000 on page 0 and the data at 0xC000 on page 1 will be different.

\emph{Example}

- page 0 is turned on, the address is 0xС000, the address to the RAM
is read, data from the addresses 0xC000 and 0xC001 are in the cache

- page 1 is turned on, the address is 0xС000, the address to the RAM
is read, data from the addresses 0xC000 and 0xC001

is recalculated - page 1 is turned on, the address is 0xС001, the
address to the cache is read

\begin{quotation}
  \textbf{Ports of memory management. \emph{MemConfig, Page0 - Page3,
      FMAddr}}
\end{quotation}
Port \emph{\#21af} - MemConfig is described here .

Ports \textbf{Page0 - Page3 }\emph{(\#10af - \#13af)} are designed to
display pages in the specified RAM address windows: \#0000, \#4000,
\#8000, \#c000, which covers the entire address space of the memory.

The \textbf{FMAddr} port \emph{(\#15af) is} designed to control the
reflection of the system's internal memory in the memory available to
the processor for recording.

We have the ability to enable this window at multiple \#1000
addresses.

The following data is stored in the internal memory of the system: a
palette is stored at relative addresses 0-511 , and sprites handles
are stored at addresses 512-1023, and from the address \#400 there is
a port mapping block. Writing to these addresses leads to programming
of system ports, like direct writing to a port.

\begin{quotation}
  \textbf{Ports of TSU graphics pages. \emph{VPage, T0GPage, T1GPage,
      TMPage, SGPage}}
\end{quotation}
These ports are designed to specify the pages used by the TSU
(tile-sprite engine that displays everything on the screen) to display
the video, as well as graphics of tile layers and sprites.

The \textbf{VPage} port \emph{(\#01af)} indicates to the system,
starting from which page a consecutive set of memory pages will be
used as video pages .

Ports \textbf{T0GPage } \emph{(\#17af)} , \textbf{T1GPage} (\#18af),
\textbf{SGPage} (\#19af) point to the pages that contain the prepared
tile graphics for output. \textbf{TMPage}

port \emph{(\#16af)} points to the page used as a tile map.\textbf{}
\emph{}

\begin{quotation}
  \textbf{Ports control graphics modes / color. \emph{VConfig, PalSel,
      Border}}
\end{quotation}
The \textbf{VConfig} port \emph{(\#00af) is} intended for the
following tasks: setting the resolution and video mode, as well as
controlling the display of the TSU and graphics layer.

The \textbf{PalSel} port \emph{(\#07af) is} used to select a palette
that will be used in the main graphic screen, and also allows you to
specify the upper two bits of the palette number for tile
layers. Combining the two high bits of PalSel for the tiles, and the
selected lower two bits of the palette in the tile descriptor, we get
the number of the specific tile palette.

Port \textbf{Border } \emph{(\#0faf)} indicates the color of the
border in the current system palette.

\begin{quotation}
  \textbf{Display control ports. \emph{TSConfig, GXOffs / GYOffs,
      T0XOffs / T0YOffs, T1XOffs / T1YOffs}}
\end{quotation}
The \textbf{TSConfig} port \emph{(\#06af) is} intended for TSU and
allows you to enable / disable the display of sprites, tile layers by
setting the S\_EN, T1\_EN, T0\_EN bits to 1. Bits T0Z\_EN and T1Z\_EN
are designed to enable the display of tiles number 0. The

set of ports \textbf{X\_Offs is} used to control the window display
the specified resolution for the corresponding port layer - a graphic
screen or tile layers.

Since the video memory field (and the display of tiles) is 512 pixels
wide and looped, we use two ports (major and minor) to control the
display window in order to specify 9 data bits, which allow to specify
all available positions 0-511.

Example: T0XOffsL - low 8 bits, T0XOffsH - high bit.

Ports of the position of the window are a pair for X and a pair for Y:

Graphics: \emph{\#02af} - GXOffsL, \emph{\#03af} - GXOffsH;
\emph{\#04af} - GYOffsL, \emph{\#05af} - GYOffsH.

Tile layer 0: \emph{\#40af} - T0XOffsL, \emph{\#41af} - T0XOffsH;
\emph{\#42af} - T0YOffsL, \emph{\#43af} - T0YOffsH.

Tile layer 1: \emph{\#44af} - T1XOffsL, \emph{\#45af} - T1XOffsH;
\emph{\#46af} - T1YOffsL, \emph{\#47af} - T1YOffsH.

\begin{quotation}
  \textbf{DMA management ports. \emph{DMASAddr, DMADAddr, DMALen,
      DMANum, DMACtrl}}
\end{quotation}
Ports indicate the address of the source, the receiver, their pages,
the length of the transmitted data and the number of such
transactions, as well as the transmission control port.

These ports are sensibly described in the relevant article of the DMA

DMASAddrL \emph{(\#1aaf)} , DMASAddrH \emph{(\#1baf)} , DMASAddrX
\emph{(\#1caf)}

DMADAddrL \emph{(\#1daf)} , DMADAddrH \emph{(\#1eaf)} , DMADAddrX
\emph{(\#1faf)}

DMALen \emph{(\#26af)} , DMANum \emph{(\#28af)}

DMACtrl / DMAStatus \emph{(\#27af)} - management port, intended both
for recording and for reading DMA status.

\begin{quotation}
  \textbf{Ports control ports INT. \emph{INTMask, HSINT, VSINT}}
\end{quotation}
The \textbf{INTMask} port \emph{(\#2aaf)} allows you to control
interrupt handlers.

The \textbf{HSINT} port \emph{(\#22af)} indicates the horizontal point
of arrival of the interrupt, and the VSINT port pair is the high
VSINTH \emph{(\#24af)} (the low bit is significant) and the low VSINTL
\emph{(\#23a4)} (8). bit) - vertical position (string).

More about intah - here

\begin{quotation}
  \textbf{Virtual drive management. \emph{Fddvirt}}
\end{quotation}
The \textbf{FDDVirt} port \emph{(\#29af)} uses 4 low-order bits, each
of which indicates all available A-D drives.

The drive operates under these conditions:

- the desired drive bit = 1.

- sootv. flop in FF port [1: 0]

- enabled trdos

- any betadisc port was accessed

This turns on pagi 255 at 0-3ffff

any call to any betadisc port causes pag 255 to be turned off

\begin{quotation}
  PS: When you start all this barrel organ, \textbf{you need to
    initialize ALL registers that are related to the subject} - this
  applies to all ports.

  Because it is not known what is set in the ports at the moment.
\end{quotation}

\textbf{Questions }

\emph{:? Where to find port numbers? }

! In the Directory. I deliberately in the article did not place the
addresses of the ports, use the \emph{intuitive} eksel.

\emph{? Is there a mapping in the emulsion? }

! And dumb for now. We are waiting, sir.

Literature: Handbook , FAQ , Doc from EARL: TS configuration
\end{document}
