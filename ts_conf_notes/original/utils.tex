% This file was converted from HTML to LaTeX with
% gnuhtml2latex program
% (c) Tomasz Wegrzanowski <maniek@beer.com> 1999
% (c) Gunnar Wolf <gwolf@gwolf.org> 2005-2010
% Version : 0.4.
\documentclass{article}
\begin{document}
\section*{TSconf: Utils}

Development

\textbf{TSconf Developer Tools}

\begin{itemize}
\item ASM includes
\item Graphic conversion, graphics editor
\item Spg builder
\end{itemize}

\begin{quotation}
  \textbf{ASM includes}
\end{quotation}
I usually write all the code in sublime text + z80 highlight by psb,
for standard sjasmplus. To export a binary, use the binary save:

\begin{verbatim}
	device zxspectrum128
	org #8000
start
;        ...
;        ...
end
	include "tsconfig.asm"
	savebin "_spg/lirus_main.bin",start,end-start
\end{verbatim}

As a result of compilation, the file lirus\_main.bin will be generated
in the \_spg folder, which will later go to the collector.

In my previous articles, I constantly used the names of the ports
instead of their numbers. Alas, most often the names carry more
semantic load than just numbers :)

To use the names of ports and significant bits of the system, you
should add them to your source with the include operator
"tsconfig.asm"

Content tsconfig.asm looks like this:

\begin{verbatim}
; -$\,$-$\,$-$\,$-$\,$-$\,$-$\,$- definitions
; -$\,$- TS-config port regs

VCONFIG         equ \$00AF
STATUS          equ \$00AF
VPAGE           equ \$01AF
GXOFFSL         equ \$02AF
GXOFFSH         equ \$03AF
GYOFFSL         equ \$04AF
GYOFFSH         equ \$05AF
TSCONFIG        equ \$06AF
PALSEL          equ \$07AF
BORDER          equ \$0FAF
PAGE0           equ \$10AF
PAGE1           equ \$11AF
PAGE2           equ \$12AF
PAGE3           equ \$13AF
FMADDR          equ \$15AF
TMPAGE          equ \$16AF
T0GPAGE         equ \$17AF
T1GPAGE         equ \$18AF
SGPAGE          equ \$19AF
DMASADDRL       equ \$1AAF
DMASADDRH       equ \$1BAF
DMASADDRX       equ \$1CAF
DMADADDRL       equ \$1DAF
DMADADDRH       equ \$1EAF
DMADADDRX       equ \$1FAF
SYSCONFIG       equ \$20AF
MEMCONFIG       equ \$21AF
HSINT           equ \$22AF
VSINTL          equ \$23AF
VSINTH          equ \$24AF
DMALEN          equ \$26AF
DMACTR          equ \$27AF
DMASTATUS       equ \$27AF
DMANUM          equ \$28AF
FDDVIRT         equ \$29AF
INTMASK         equ \$2AAF
T0XOFFSL        equ \$40AF
T0XOFFSH        equ \$41AF
T0YOFFSL        equ \$42AF
T0YOFFSH        equ \$43AF
T1XOFFSL        equ \$44AF
T1XOFFSH        equ \$45AF
T1YOFFSL        equ \$46AF
T1YOFFSH        equ \$47AF

; TS parameters
FM_EN           equ \$10

; VIDEO
VID_256X192     equ \$00
VID_320X200     equ \$40
VID_320X240     equ \$80
VID_360X288     equ \$C0
VID_RASTER_BS	equ 6

VID_ZX          equ \$00
VID_16C         equ \$01
VID_256C        equ \$02
VID_TEXT        equ \$03
VID_NOGFX       equ \$20
VID_MODE_BS		equ 0

; PALSEL
PAL_GPAL_MASK	equ \$0F
PAL_GPAL_BS		equ 0
PAL_T0PAL_MASK	equ \$30
PAL_T0PAL_BS	equ 4
PAL_T1PAL_MASK	equ \$C0
PAL_T1PAL_BS	equ 6

; TSU
TSU_T0ZEN		equ \$04
TSU_T1ZEN		equ \$08
TSU_T0EN		equ \$20
TSU_T1EN		equ \$40
TSU_SEN			equ \$80

; SYSTEM
SYS_ZCLK3_5		equ \$00
SYS_ZCLK7		equ \$01
SYS_ZCLK14		equ \$02
SYS_ZCLK_BS		equ 0

SYS_CACHEEN		equ \$04

; MEMORY
MEM_ROM128		equ \$01
MEM_W0WE		equ \$02
MEM_W0MAP_N		equ \$04
MEM_W0RAM		equ \$08

MEM_LCK512		equ \$00
MEM_LCK128		equ \$40
MEM_LCKAUTO		equ \$80
MEM_LCK1024		equ \$C0
MEM_LCK_BS		equ 6

; INT
INT_VEC_FRAME	equ \$FF
INT_VEC_LINE	equ \$FD
INT_VEC_DMA		equ \$FB

INT_MSK_FRAME	equ \$01
INT_MSK_LINE	equ \$02
INT_MSK_DMA		equ \$04

; DMA
DMA_WNR         equ \$80
DMA_SALGN       equ \$20
DMA_DALGN       equ \$10
DMA_ASZ         equ \$08

DMA_RAM         equ \$01
DMA_BLT         equ \$81
DMA_FILL	    equ \$04
DMA_SPI_RAM     equ \$02
DMA_RAM_SPI     equ \$82
DMA_IDE_RAM     equ \$03
DMA_RAM_IDE     equ \$83
DMA_RAM_CRAM    equ \$84
DMA_RAM_SFILE   equ \$85

; SPRITES
SP_XF			equ \$80
SP_YF			equ \$80
SP_LEAP			equ \$40
SP_ACT			equ \$20

SP_SIZE8		equ \$00
SP_SIZE16		equ \$02
SP_SIZE24		equ \$04
SP_SIZE32		equ \$06
SP_SIZE40		equ \$08
SP_SIZE48		equ \$0A
SP_SIZE56		equ \$0C
SP_SIZE64		equ \$0E
SP_SIZE_BS		equ 1

SP_PAL_MASK		equ \$F0

SP_XF_W			equ \$8000
SP_YF_W			equ \$8000
SP_LEAP_W		equ \$4000
SP_ACT_W		equ \$2000

SP_SIZE8_W		equ \$0000
SP_SIZE16_W		equ \$0200
SP_SIZE24_W		equ \$0400
SP_SIZE32_W		equ \$0600
SP_SIZE40_W		equ \$0800
SP_SIZE48_W		equ \$0A00
SP_SIZE56_W		equ \$0C00
SP_SIZE64_W		equ \$0E00
SP_SIZE_BS_W	equ 9

SP_X_MASK_W		equ \$01FF
SP_Y_MASK_W		equ \$01FF
SP_TNUM_MASK_W	equ \$0FFF
SP_PAL_MASK_W	equ \$F000

; TILES
TL_XF			equ \$40
TL_YF			equ \$80
TL_PAL_MASK		equ \$30
TL_PAL_BS		equ 4

TL_XF_W			equ \$4000
TL_YF_W			equ \$8000

TL_TNUM_MASK_W	equ \$0FFF
TL_PAL_MASK_W	equ \$3000
TL_PAL_BS_W		equ 12
\end{verbatim}

The use of port names and bits greatly improves the reading of the
source.

\begin{quotation}
  \textbf{Graphic conversion, graphics editor}
\end{quotation}
To use graphics in a demo, you need to draw / cook it in a graphic
editor.

I use Photoshop, in which I perform all the operations for preparing
graphics - resizing, clipping, etc.

At the last stage, conversion into indexed colors is performed. For
tiles / sprites you need to remember that one of the colors will be
used as the transparency color.

At the last stage, everything is unloaded in tga format (256 colors, 8
bits) - even if 16 or less colors are used.

To set the desired color as the first (0th, transparency) in the
palette, you can use a very pleasant and useful for our purposes
Usenti editor. Its biggest advantage is the means for working with the
palette - sorting, cell exchange, and so on. And draw them convenient,
try.

We receive the received tga in the wonderful converter with which help
separate files of pixel data of the image and its palette will be
prepared.

The tga2ts.exe converter itself is here , you need both files for it
to work - both tga2ts.exe and levels.map.

Example of use:
\begin{verbatim}
tga2ts.exe back9.tga
\end{verbatim}

we will get the following set of files:

\begin{verbatim}
back9.tga.pal
back9.tga.pix
back9.tga.pix4
\end{verbatim}

* .pal - palette, 512 bytes (256 * 2 bytes per color). When using 16
color images, I use pens to remove the zeros, leaving only 32 bytes of
the palette.

* .pix - image in 256 color format - byte per point (byte per color)

* .pix4 - image in 16 color format - byte on two points

All image data is linear.

So, the converted images are received. Given the possible large size -
these files must be placed in memory for use. Naturally, we can save
files to a floppy disk, write a bootloader, and load the file in the
old way, as usual. But what if you need to immediately load into
memory more than 600 KB? for example, megabyte so \emph{jwa} ?

For this, it makes sense to use the SPG collector.

\begin{quotation}
  \textbf{SPG collector}
\end{quotation}
This program is designed to generate * .spg files. This format is a
smart snapshot of the pentevo memory pages used in the program, and
when loading it places the data blocks in the necessary pages, after
which it starts at the address specified in the spg header.

* .spg allows you to run WildCommander

Collector can be found here , data on the location of files in memory
are described in spgbld.ini.

Information in spgbld.ini about the location in memory is set as follows:

\begin{verbatim}
Desc = lirus    ; name without quotes
Start = #8000    ; start address
Stack = #5fff    ; initial stack location
Resident = #5B00    ; location of the resident
Page3 = 0    ; home page
Clock = 2    ; processor speed. 0 - 3.5 MHz / 1 - 7 MHz / 2 - 14 MHz / 3 - 14 MHz overclocked
INT = 0    ; interrupts are off
Pager = 0    ; Page manager address, 0 - without manager

;Block = #e000,  5,violent.bin    ; specify the location address in memory, page, file. In this case, the file violent.bin will be located at #6000, page 5
Block = #c000,  2,lirus_main.bin    ; this file will be located at #8000
Block = #c000,  0,lirus_p0.bin    ; and this one is in pag 0, address #c000

Block = #c000, #10,player49152.bin
Block = #c800, #10,nq-first_warning-8.pt3
Block = #e000, #10,outro.pt3
Block = #c000, #11,tiles2.tga.pix4
Block = #d000, #12,geebeeyay-8x8.tga.pix4
\end{verbatim}

A feature of the collector is that:

\begin{itemize}
\item Location address must be a multiple of 512 (\#200)
\item The address is indicated as a position in the memory page,
  starting with the address \#c000.

  i.e. address \#6000 is page 5, address \#e000 (offset \#2000);
  \#b000 is page 2, address \#f000 (offset \#3000)

  Clarification - everything lies in our pag that can be connected to
  different windows (0-3).

  Accordingly, the specified address will always be given in the form
  \#0000 - \#3e00, and
  \textbf{the upper two bits of the address will be RESET ~ OSEN}
  as unnecessary.

  Accordingly, if we want to load the block into the 5th page from the
  address \#6000 - we indicate:
      
  Block = \#2000, 5, file.bin; this file will be located at \#2000 of
  the included window.

  If we include this pag in window 1 - this is 6000, if in window 2 -
  then the address is a000, if in window 3 - e000

\item The size of the data block may exceed the standard page size
  (16kb), while the data will occupy subsequent pages of memory
\end{itemize}

After doing
\begin{verbatim}
spgbld.exe -b spgbld.ini TEST.spg -c 0
\end{verbatim}
the TEST.spg file is assembled and created, which can be opened both
on PentEVO from WC and in the emulator.

The -c 0 parameter indicates that no compression is needed during
assembly. If you set this flag to 1, then the assembly will call the
wonderful mhmt.exe compressor from our dear friend lvd , which will
test for the best compression and the best block in size will fit into
the assembly.

Naturally, it takes time - both when building and running on real
hardware, consider the moment of waiting for decompression when
starting the file on PentEVO.

\textbf{Questions}

\emph{:? I can not download files}

! Contact

Literature: Doc on CG builder
\end{document}
