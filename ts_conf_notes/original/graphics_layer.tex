% This file was converted from HTML to LaTeX with
% gnuhtml2latex program
% (c) Tomasz Wegrzanowski <maniek@beer.com> 1999
% (c) Gunnar Wolf <gwolf@gwolf.org> 2005-2010
% Version : 0.4.
\documentclass{article}
\begin{document}
\section*{TSconf: Gfx layer}

Development

Sprites, tiles ... Perhaps this would be enough for us ... 

But under these layers is the \textbf{base graphics layer} .

\begin{quotation}
  Tao says: The graphics layer displays data that resides in memory
  pages. The first page to display (its address is necessarily a
  multiple of 8 for 16 colors, 16 for a 256 color mode, total length -
  8/16 pages) is indicated by the port VPage (\#01af). The way this
  memory is mapped is set by the VConfig port bits, which sets the
  resolution and color depth.

  The memory display window is a block with the dimensions of the
  specified resolution and is displayed by the positions X (0-511) and
  Y (0-511), which are indicated by the pairs of ports GXOffs and
  GYOffs. The window is looped around the edges in the display.

  The colors of the display are given by the palette, the number is
  selected by the first notebook of the PalSel register (\#06af).
\end{quotation}

So, we have a screen with our own internal scrolling.

The output to the screen is a data record in the video memory page at
the desired address.

Depending on the selected color mode, recording one byte allows you to
output one (256 colors) or two (16 color mode) dots at a time. In the
16 color mode, the highest tetrad is responsible for the color number
in the palette for the left point, the youngest - for the right one.

In 256 color mode - the byte of the dot indicates the color in the
palette.

\textbf{What is the choice of resolution?}

\begin{verbatim}
RRES[1:0]	Pixels
00	256x192
01	320x200
10	320x240
11	360x288

VM[2:0]	Mode
00	ZX
01	16c
10	256c
11	Text}
\end{verbatim}

All this is set by the VConfig port bits (\#00af):

\begin{verbatim}
VConfig	RRES[1:0]		NOGFX	NOTSU	-	-	VM[1:0]
RRES - 7,6 биты - разрешение
VM[1:0]	1, 0 биты - цветовой режим}
\end{verbatim}

In addition, this port offers us a bit to enable / disable graphics
display and the bits on / off of the sprite-tile engine.

In the first case - when turning off the graphics, we will see a clean
screen of the color specified by the border. In the second - sprite
and tile layers will not be displayed.

\emph{Disabling unused layers leads to the release of system resources
  and will allow, for example, to transfer more DMA data per line. }

\textbf{Let's display something on the screen. }

The easiest way, I think, is to organize the output of a picture using
DMA - let the iron work, the processor thinks at this time, and the
person is resting :)

So, we will display the image in 16 color mode, 256x256 pixels in
size.

To do this, you must prepare the ports, indicating the desired
resolution, color gamut, address of the video memory page:

\begin{verbatim}
		ld hl,init_ts
		jp set_ports

Vid_page	equ #40

init_ts		db high VCONFIG,VID_16C+VID_320X240
		db high VPAGE,Vid_page
		db high PALSEL,0

	include "tsconfig.asm"
\end{verbatim}

Note that the palette for graphics has the number 0. I recall that the
standard initial palette of the zx spectrum is located in the last,
15th palette.

\begin{verbatim}
		ld hl,pic_copy
		call set_ports
                jr \$
screen_page	equ #80 ; page number in which the image is stored for output
pic_copy
		db #1a,0
	        db #1b,0
		db #1c,screen_page
	        db #1d,0
	        db #1e,0
	        db #1f,Vid_page
	        db #26,256/4-1
	        db #28,256-1
		db #27,DMA_RAM + DMA_DALGN
		db #ff
\end{verbatim}

Everything, the image is on the screen, starting from the 0x0 coordinate. 

It’s very simple to move this image around the screen by setting
indents by programming ports of offsets for the base graphics layer -
GXOffs / GYOffs (\#02af-03af / \#04af-05af):

\begin{verbatim}
offs1		ld a,0
		inc a
		and #7f
		ld (offs1+1),a
		ld bc,GXOFFSL
		out (c),a
		ld bc,GYOFFSL
		out (c),a
		halt
		jr offs1
\end{verbatim}

In this case, the offsets are written for the lower 8 bits, and the
offset is in the range of 0-127.

\textbf{Questions and Answers }

\emph{:? Why on the screen after turning on the mode and selecting the
  video page a color NOISE? }

! The initial contents of the memory have a random nature, which is
displayed on the screen. Clean up!

\emph{? What is the address of the point y = 1? }

! Depending on the selected color mode:

With 16 colors: y = 0: \#c000; y = 1: \#c100

With 256 colors: y = 0: \#c000; y = 1: \#c200

\emph{? I do not understand what is displayed! }

! The screen start page should be a multiple of 8. Should the palette
be loaded and selected by the PalSel port

\emph{? Only a narrow bar is displayed!}

! Depending on the selected mode, up to 262,144 bytes of memory are
used, which does not fit into 16k at all.

Switch the video page and draw on!

Literature:

Graphic Modes TS-Config

F.A.Q
\end{document}
