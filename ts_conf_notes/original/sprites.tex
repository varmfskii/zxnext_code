% This file was converted from HTML to LaTeX with
% gnuhtml2latex program
% (c) Tomasz Wegrzanowski <maniek@beer.com> 1999
% (c) Gunnar Wolf <gwolf@gwolf.org> 2005-2010
% Version : 0.4.
\documentclass{article}
\begin{document}
\section*{TSconf: Sprites}

Development

But why not pile over the sprite graphics, eh?

They have a distort.

\begin{quotation}
  Tao says: Sprite is a graphic tile with a size from 8 to 64 points,
  which can be displayed in coordinates 0-511 in X and Y, with the
  possibility of reflecting the display vertically / horizontally, and
  having transparency.

  A total of 85 sprites are available for one sprite layer.
\end{quotation}

\emph{In general, working out sprites is as follows:}

\begin{itemize}
\item The sprite needs to be placed in memory in exactly the same way
  as for tiles (or on the screen in 16 color mode).
\item The sprite palette must also be loaded into the overall palette
  of the system.
\item After that - we load the description of the sprite into the
  system and turn on the display.
\end{itemize}
Sprite becomes visible.

Let's sort it out in order:

Well, firstly - the sprite itself.

Sprite we can have a size of a multiple of 8 points in both X and Y,
from 8 to 64;

\begin{quotation}
  XS X Size. (0 = 8, 1 = 16, ... 7 = 64 pixels)

  YS Y Size. (0 = 8, 1 = 16, ... 7 = 64 lines)

\end{quotation}
those. - 16x64 - please, 48x8 - also, but 6x33 - alas, no. The size
will be a multiple of 8 points.

This is due to the fact that the sprites are almost the same tiles .

The sprite does not fit even in the size of 64 points? No problem!
Solve, but a little later;)

You also need to consider that in TSconf for one sprite, one palette
is selected, which includes 15 colors of the sprite and a transparent
color. Transparency is the color number 0 in the palette chosen for
the sprite. The palette, of course, is loaded into the palette memory,
the number of this palette is noted.

\textbf{Sprite must be placed in memory. }To do this, select the page,
the number of which must be a multiple of 8 (0, 8, 192, 240, etc.).

Sprite data should not lie in a linear way, but as on the screen -
i.e. the next line of the sprite corresponds to the next line of the
screen.

To do this, you should use DMA and copy it to the page used to store
the graphics of the sprite, taking into account these requirements.

Well, we will assume that the sprite is correctly copied into memory.

In order for us to see the sprite, it is necessary to transfer
information about it to the system. For the operation of the sprites
in the system there is a separate memory block, which is responsible
for storing the "descriptors" of sprites - information describing the
sprite, its size, position, etc.

\textbf{A single sprite handle} takes 6 bytes:

\begin{verbatim}
SFILE	Reg.16	7	6	5	4	        3    2    1        	0
0	R0L	Y[7:0]
1	R0H	YF	LEAP	ACT	reserved	YS[2:0]			Y[8]
2	R1L	X[7:0]
3	R1H	XF	-	-	reserved	XS[2:0]			X[8]
4	R2L	TNUM[7:0]
5	R2H	SPAL[7:4]				TNUM[11:8]

Description:
X	X coordinate, byte + 1 bit (address can be between 0-511)
Y	Y coordinate, byte + 1 bit
XS	X size. (0=8, 1=16, ... 7=64 pixels)
YS	Y size. (0=8, 1=16, ... 7=64 lines)
XF	Horizontal sprite reflection on output. We note to ourselves that we already need to draw 2 times less sprites!
YF	Vertical sprite reflection on output
ACT	Display sprite (on/off)
LEAP	Bit defintes the transition to the next layer for subequent sprites. The thirs transition marks the end of the sprite descriptors.
TNUM	The number of the first tile of the sprite of the specified size. Bits 0-5 - X tile position in the sprite graphics page, bits 6-11 - Y position.
SPAL	Selects a palette. 4 bits - 16 palettes available.
\end{verbatim}

Let's look at an example of the descriptor: The

sprite will be 64x64 pixels in size, the palette for it is number 1,
located in memory from the address \#c000 - i.e. Start - Tile number
0:

\begin{verbatim}
sprite:
  db 0         ; Y position
  db %00111110
;; 0 - rotation (flip, reflection) on X off
;; 0 - go to the next layer of sprites off,
;; 1 - the sprite will be displayed
;; 1 - reserve for future use
;; 111 - three bits of size (64 points) in Y
;; 0 - the most significant bit of the position in Y
  db Xpos_logo ; X position
  db %00001110
;; 0 - rotation (flip) in Y off
;; 000 - not used
;; 111 - 64 points in X
;; 0 - the high bit of the X position
  db 0         ; tile number on sprite graphics card. countdown from the upper left corner, one tile 8x8 points.
  db %00010000 ; the highest tetrad is the number of the palette, the youngest is the highest bits of the tile number
sprite_end:
\end{verbatim}

After the data descriptor will be sent to the system, display it on
the screen.

\begin{verbatim}
  ld hl,logo_spr
  call set_ports
;; ...
logo_spr:
  defb #1a,low logo ; logo - all sprite descriptors, do not forget address alignment by 2
  defb #1b,high logo
  defb #1c,2
  defb #1d,0
  defb #1e,2        ; send to sprite memory
  defb #1f,0
  defb #26,(sprite_end-sprite)/2    ;transaction length
  defb #28,0
  db   #27,#85    ; DMA_RAM_SFILE
  db   #ff
\end{verbatim}

To do this, you must inform the system that the sprites will be
used. To do this, set the bit responsible for enabling the display of
sprites when outputting a TSConfig port, called S\_EN
(Sprite\_ENable):

\begin{verbatim}
    		ld bc,TSCONFIG
		ld a,TSU_SEN     ; 7th bit
		out (c),a
\end{verbatim}

Sprite prepared and included. We observe it on the screen. It is
indestructible, dog!

For that, he began to move - changing its position on the screen in
the handle - X and Y.

In order that-be looked sprite sprite (sprite image changed) - change
in the schedule of the tile number of the sprite on the right.

\textbf{Do not forget to send the changed data to the sprite memory by
  transfer. }

\emph{So, to render the sprite more than 64 points - use TWO sprites
  with the necessary coordinates;) }

\textbf{Questions and answers }

\emph{:? How to hide the sprite? }

! Turn off the Act bit, the sprite will not be displayed

\emph{? How to bring the sprite from the edge of the screen? }

! Specify the coordinates outside the screen: for example, -32

\emph{? My sprites are closed with tiles, why?}

Because you need to use the Leap bit to go to the desired layer. 2
times - and the sprite is on top of all layers.

Well, I went! ;)

\end{document}
