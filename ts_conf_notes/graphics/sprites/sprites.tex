\section{Sprites}

A prite is a graphic tile with a size ranging from 8 to 64 pixels,
which can be displayed in coordinates in the range 0-511 in both the X
and Y direction, with the possibility of possibility of being
reflected vertically and/or horizontally, and having transparency color.

A total of 85 sprites are available in each sprite layer.

The sprite needs to be placed in memory in exactly the same way as for
tiles (or the screen in 16 color mode).

The sprite palette is taken from the system palette.

After that - we load the description of the sprite into the system and
turn on the display.

Sprite becomes visible.

Let's sort it out in order: 

Well, firstly - the sprite itself. 

Sprite we can have a size of a multiple of 8 points in both X and Y,
from 8 to 64;

XS X Size. (0 = 8, 1 = 16, ... 7 = 64 pixels) 

YS Y Size. (0 = 8, 1 = 16, ... 7 = 64 lines)

those. - 16x64 - please, 48x8 - also, but 6x33 - alas, no. The size
will be a multiple of 8 points.

This is due to the fact that the sprites are almost the same tiles . 

The sprite does not fit even in the size of 64 points? No problem!
Solve, but a little later;)

You also need to consider that in TSconf for one sprite, one palette
is selected, which includes 15 colors of the sprite and a transparent
color. Transparency is the color number 0 in the palette chosen for
the sprite. The palette, of course, is loaded into the palette memory,
the number of this palette is noted.

Sprite must be placed in memory. To do this, select the page, the
number of which must be a multiple of 8 (0, 8, 192, 240, etc.).

Sprite data should not lie in a linear way, but as on the screen -
i.e. the next line of the sprite corresponds to the next line of the
screen.

To do this, you should use DMA and copy it to the page used to store
the graphics of the sprite, taking into account these requirements.

Well, we will assume that the sprite is correctly copied into memory. 

In order for us to see the sprite, it is necessary to transfer
information about it to the system. For the operation of the sprites
in the system there is a separate memory block, which is responsible
for storing the "descriptors" of sprites - information describing the
sprite, its size, position, etc.

A single sprite handle takes 6 bytes:

SFILE	Reg.16	7	6	5	4	        3    2    1        	0\\
0	R0L	Y[7:0]							\\
1	R0H	YF	LEAP	ACT	reserved	YS[2:0]			Y[8]\\
2	R1L	X[7:0]							
%3	R1H	XF	-	-	reserved	XS[2:0]			X[8]
%4	R2L	TNUM[7:0]							
%5	R2H	SPAL[7:4]				TNUM[11:8]			
%
%Описание:
%X	X координата, байт + 1 бит (адрес может быть в пределах 0-511)						
%Y	Y координата, байтом + 1 бит
%XS	X размер. (0=8, 1=16, ... 7=64 pixels)
%YS	Y размер. (0=8, 1=16, ... 7=64 lines)							
%XF	Горизонтальное отражение спрайта при выводе. <em>Отмечаем себе, что спрайтов рисовать уже нужно в 2 раза меньше!</em>
%YF	Вертикальное отражение спрайта при выводе
%ACT	Отображение спрайта (вкл/выкл)
%LEAP	Бит, определяющий переход на следующий слой для последующих спрайтов. Третий переход обозначает конец описателей спрайтов
%TNUM	Номер первого тайла спрайта указанного размера. Биты 0-5 - X позиция тайла в странице графики спрайтов, биты 6-11 - позиция по Y.
%SPAL	Выбор палитры. 4 бита - 16 палитр доступно.						
%
%
%Let's look at an example of the descriptor: The 
%sprite will be 64x64 pixels in size, the palette for it is number 1, located in memory from the address # c000 - i.e. Start - Tile number 0:
%
%sprite    	db 0    ; позиция по Y
%		db %00111110    ; биты: 0 - поворот (flip, отражение) по Х выкл,  0 -переход на следующий слой спрайтов выкл, 
%; 1 - спрайт будет отображаться,1 - резерв для будущего применения; три бита размера (111 - 64 точки) по Y, 
%; и последний бит - 0 - старший бит позиции по  Y
%		db Xpos_logo    ; позиция по Х
%		db %00001110    ; 0 - поворот (flip) по Y выкл, дальше биты не юзаются до битов размера, 111 - 64 точки по Х; последний бит - это старший бит позиции по Х
%		db 0            ; номер тайла на карте графики спрайта. отсчёт от левого верхнего угла, один тайл 8х8 точек.
%		db %00010000    ; старшая тетрада - это номер палитры, младшая - старшие биты номера тайла
%sprite_end
%
%
%After the data descriptor will be sent to the system, display it on the screen.
%
%		ld hl,logo_spr
%		call set_ports
%; ...
%logo_spr
%		defb    #1a,low logo ; logo - все дескрипторы спрайтов, не забываем выравнивание адреса по 2
%	        defb    #1b,high logo
%		defb    #1c,2
%	        defb    #1d,0
%	        defb    #1e,2    ; отправляем в спрайтовую память
%	        defb    #1f,0
%	        defb    #26,(sprite_end-sprite)/2    ;длина транзакции
%	        defb    #28,0
%		db 	#27,#85    ; DMA_RAM_SFILE
%		db #ff
%
%
%To do this, you must inform the system that the sprites will be used. To do this, set the bit responsible for enabling the display of sprites when outputting a TSConfig port, called S_EN (Sprite_ENable):
%
%    		ld bc,TSCONFIG
%		ld a,TSU_SEN     ; 7й бит
%		out (c),a
%
%
%Sprite prepared and included. We observe it on the screen. It is indestructible, dog! 
%
%For that, he began to move - changing its position on the screen in the handle - X and Y. 
%In order that-be looked sprite sprite (sprite image changed) - change in the schedule of the tile number of the sprite on the right. 
%Do not forget to send the changed data to the sprite memory by transfer. 
%
%So, to render the sprite more than 64 points - use TWO sprites with the necessary coordinates;) 
%
%Questions and answers 
%:? How to hide the sprite? 
%! Turn off the Act bit, the sprite will not be displayed 
%
%? How to bring the sprite from the edge of the screen? 
%! Specify the coordinates outside the screen: for example, -32 
%
%? My sprites are closed with tiles, why?
%Because you need to use the Leap bit to go to the desired layer. 2 times - and the sprite is on top of all layers. 
%
%Well, I went! ;)
